\section{Organization of the thesis\label{section:intro-organization}}

The thesis is structured as follows.
\begin{description}

\item[Chapter 2] sets the background picture for the thesis as a guided tour of the multi-view modeling framework considered. It describes the formalisms used for capturing scenarios, state machines, goals and process models. The chapter also states important consistency rules between these specific views.

\item[Chapter 3] describes our technique for vertical synthesis of labeled transition system from process models. A trace semantics for process models is defined together with algorithms for deriving lower-level models. The first algorithm maps process models into guarded LTS; the second algorithm maps the latter into pure event-based LTS.

\item[Chapter 4] describes our inductive techniques for horizontal synthesis of state machines from scenarios, based on interactive grammar induction. From a minimal problem statement, constraints on the synthesis process are gradually added and the algorithms are incrementally enhanced to meet these constraints.

\item[Chapter 5] discusses how our inductive synthesis technique has been evaluated. It covers both evaluations on case studies and evaluations on synthetic datasets.

\item[Chapter 6] discusses Stamina, our online platform for evaluating inductive synthesis techniques. Stamina first ran as a formal competition whose results are also briefly summarized.

\item[Chapter 7] discusses the toolset developed to support the proposed techniques, namely, a model checker for g-hMSC process models, an interactive synthesizer of state machines from scenarios, and a tool for modeling and analyzing process models for critical applications such as complex medical workflows.

\item[Chapter 8] reviews related work, discussing how other synthesis techniques complement and/or differ from those proposed in the thesis.

\item[Chapter 9] concludes the thesis and discusses open issues, perspectives, and future work. 

\end{description}
