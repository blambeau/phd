\section{Synthesis as a promising approach for multi-view model building\label{section:intro-synthesis}}

Model synthesis can be roughly defined as the systematic construction of models from various sources of knowledge about the target system. 

Common sources of knowledge for model building include early system descriptions, interviews with stakeholders, software prototypes, etc. For an in-depth description of requirements elicitation and system modeling from such sources see, e.g., \cite{VanLamsweerde:2009}.

Multi-view modeling frameworks offer new opportunities for effectively supporting the incremental building of models. In particular, the modeling languages often come with precise rules of consistency between the models. These rules are used for automating consistency checks. The same rules can also be used the other way round, to semi-automatically synthesize model fragments using the information already available in other views.

The use of a synthesis technique may be motivated by various needs. As a consequence, the term ``model synthesis'' may cover a broad spectrum of techniques and practices. In this thesis, we will focus on two particular and complementary forms of synthesis:
\begin{description}
\item[Horizontal synthesis] To complete a multi-view model of the target system. A synthesis technique is used here to build models missing from a multi-view framework or to complete existing ones.
\item[Vertical synthesis] To derive lower-level models from higher-level ones. This yields an operational semantics from the latter makes model-checking tools available to them.
\end{description}

The two forms of model synthesis will be seen to be fairly different in the thesis.
\begin{itemize}
\item In vertical synthesis, the synthesized models are defined at a lower level of abstraction than the source models they come from; this is not necessarily the case in horizontal synthesis.
\item The analyses enabled by vertical synthesis are performed on models having a shorter lifetime -- generally limited to the analysis itself. In horizontal synthesis, the resulting models are often kept as software documentation or requirements traceability. 
\item Vertical synthesis is derivational by nature whereas horizontal synthesis may be inductive.
\item In vertical synthesis, models are generally kept hidden from the user; the latter has a passive role in the synthesis process. In horizontal synthesis, the produced models must generally be validated by the end-user. The latter may even play an active role in the synthesis process itself.
\end{itemize}

The core chapters of this thesis will cover those two forms of model synthesis.
\begin{itemize}
\item A technique will be presented for synthesizing agent behavior models from scenarios. The language of Message Sequence Charts (MSCs) will be chosen for describing scenarios of agent interactions. Agent behaviors will be modeled with LTS state machines. The semantic links between MSC scenarios and LTS state machines is based on \cite{Uchitel:2003}.

The proposed synthesis will rely on grammar induction methods from \cite{Oncina:1992, Lang:1998}. Here, the inductive synthesis will take scenarios as examples and counterexamples of desired system behaviors; user interactions will be added to accept or reject scenarios generated by the technique. Moreover, domain knowledge such as goals or models of legacy components will be taken into account to ensure the consistency of the synthesized models while speeding-up the induction process and reducing the number of user interactions.

\item Guarded high-level Message Sequence Charts (guarded hMSC) are introduced in \cite{Damas:2010, Damas:2011} to model critical processes such as the ones involved in medical workflows. Such models take the form of flowcharts of tasks and decision nodes with outgoing guarded transitions labeled with Boolean expressions on process variables. 

In the thesis, a precise trace semantics for guarded hMSC will be defined in terms of labeled transition systems (LTS) \cite{Keller:1976, Magee:1999}. To achieve this, a guarded flavor of LTS is introduced as an intermediate model. Synthesis algorithms from guarded hMSC, to guarded LTS to LTS will be described. These algorithms make LTS-based model checking available to the process models though slight adaptations of a technique borrowed from \cite{Giannakopoulou:2003}.
\end{itemize}
