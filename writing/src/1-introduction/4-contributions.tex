\section{Overview of contributions\label{section:intro-contributions}}

This section summarizes the thesis contributions. Section \ref{subsection:intro-contib-supporting-process-models} lists them for the modeling and analysis of process models. Section \ref{subsection:intro-contrib-inductive-synthesis} focuses on those along our state machine synthesis technique. The latter reuses and extends algorithms from grammar induction to which specific contributions have be made. They are summarized in Section \ref{subsection:intro-contrib-grammar-induction}. 

%%%

\subsection{Effectively supporting formal process modeling\label{subsection:intro-contib-supporting-process-models}}

\begin{itemize}
\item The thesis defines a trace semantics for the process language of guarded hMSCs described in \cite{Damas:2011}. This trace semantics is defined in terms of guarded LTS and allows model checking process models with a trace-based model-checker similar to the one in LTSA \cite{Magee:1999}. 
\item Guarded LTS are an extension of LTS with guarded transitions defined on fluents. The thesis provides them with a declarative trace semantics. It also describes a composition operator for guarded LTS as well as an algorithm for capturing their semantics as a pure LTS.
\item The thesis describes the implementation of a compositional model-checker for guarded hMSC and guarded LTS. It also describes key points in the implementation of a tool for modeling and analyzing guarded processes. 
\end{itemize}

%%%

\subsection{Multi-view synthesis of state machines\label{subsection:intro-contrib-inductive-synthesis}}
\begin{itemize}
\item The thesis provides algorithms for synthesizing state machines from MSC scenarios and high-level MSCs. These algorithms rely on grammar induction for generalizing scenario behaviors. Knowledge from various sources, notably goals, can be incorporated to constraint the induction process so as to ensure multi-view model consistency.
\item A tool supporting our muti-view synthesis approach is also described. This tool has been successfully used for evaluating the approach on common case-studies from the requirement engineering literature.
\end{itemize}

%%%

\subsection{Contributions to grammar induction\label{subsection:intro-contrib-grammar-induction}}

\begin{itemize}
\item The thesis describes an induction algorithm called QSM. It is an interactive extension to state-of-the-art algorithms in grammar induction, namely RPNI and Blue-fringe \cite{Oncina:1992, Lang:1998}. QSM extends those algorithms with an oracle, typically an end-user. The oracle guides the induction by answering so-called ``membership queries'', that is, by classifying generated strings as positive or negative elements of the induced regular language.
\item The ASM and ASM$^*$ algorithms are other induction algorithm contributed in the thesis. They allow generalizing a positive \emph{language} in contrast to a positive \emph{sample}, that is, a finite set of induction strings. The generalization is made under the control of a negative sample from ASM, of a negative language for ASM$^*$.
\item The thesis describes a new protocol for evaluating induction algorithms. This protocol is an alternative to the Abbadingo procedure \cite{Lang:1998}; it has been designed for cases where samples are defined on large alphabets and obtained from the target state machine itself. 
\end{itemize}
