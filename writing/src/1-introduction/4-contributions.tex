\section{Overview of contributions\label{section:intro-contributions}}

This section summarizes the main contributions of the thesis. Section \ref{subsection:intro-contrib-grammar-induction} lists the thesis contributions to grammar induction techniques. Two contributed algorithms, QSM and ASM, provide a basis for the inductive synthesis of state machines from scenarios. The thesis contributions along this horizontal axis of model synthesis are summarized in Section \ref{subsection:intro-contrib-inductive-synthesis}. Section \ref{subsection:intro-contib-supporting-process-models} lists those related to vertical synthesis of behavior models from process models.

%%%

\subsection{Extended techniques for grammar induction\label{subsection:intro-contrib-grammar-induction}}

\begin{itemize}
\item The thesis presents a new induction algorithm called Query-Driven State Merging (QSM). This algorithm is an interactive extension of state-of-the-art algorithms in grammar induction, namely RPNI and Blue-fringe \cite{Oncina:1992, Lang:1998}. QSM extends those algorithms with an oracle, typically, an end-user. The oracle interacts with the induction process by answering so-called ``membership queries'', that is, by classifying generated strings as positive or negative elements of the induced regular language.
\item The thesis presents the Automaton State Merging (ASM) algorithm as further induction technique. This algorithm allows generalizing a positive \emph{language} as opposed to a positive \emph{sample} -- that is, a finite set of induction strings. The generalization process is made under the control of a negative sample.
\item The thesis describes a new protocol for evaluating grammar induction algorithms, called Stamina. This protocol is an alternative to the existing Abbadingo procedure \cite{Lang:1998}; it has been designed for samples defined on large alphabets and obtained from the target automaton itself. Samples and automata considered in Stamina are more representative of execution traces, scenarios and state machines encountered in software engineering contexts.
\end{itemize}

%%%

\subsection{Horizontal synthesis of behavior models\label{subsection:intro-contrib-inductive-synthesis}}
\begin{itemize}
\item The thesis provides algorithms for inductively synthesizing state machines models from MSC scenarios and high-level MSCs. These algorithms rely on grammar induction techniques to generalize scenario. Knowledge from various sources, notably goals, can be incorporated to constrain the induction process so as to ensure multi-view model consistency.
\item A tool supporting our inductive synthesis approach is also described. This tool has been successfully used for evaluating the approach on common case-studies from the requirement engineering literature.
\end{itemize}

%%%

\subsection{Vertical synthesis from formal process models\label{subsection:intro-contib-supporting-process-models}}

\begin{itemize}
\item The thesis provides a formal trace semantics for the process language of guarded hMSCs described in \cite{Damas:2011}. This trace semantics is defined in terms of guarded LTS; it supports the model checking of process models thanks to a a trace-based model-checker adapted from the one in LTSA \cite{Magee:1999}. 
\item The thesis extends LTS state machines with guarded transitions defined on fluents. It provides a declarative trace semantics for guarded LTS. It also proposes a composition operator for guarded LTS together with an algorithm for capturing their semantics as pure LTS.
\item The thesis describes the implementation of a compositional model-checker for guarded hMSC and guarded LTS. It also describes key aspects in the implementation of a tool for modeling and analyzing guarded processes. 
\end{itemize}
