\section{Organization of the thesis\label{section:intro-organization}}

The rest of this thesis is organized as follows:

\begin{description}

\item[Chapter 2] installs the background as a guided tour of the multi-view modeling framework. It describes the formalisms we use for scenarios, state machines, goals and process models. The chapter also states consistency rules between them.

\item[Chapter 3] describes our technique for synthesizing state machines from process models. A trace semantics for process models is defined and two algorithms described. The first one rewrites process models as guarded state machine; the second rewrites the latter as pure event-based state machines.

\item[Chapter 4] describes our technique for synthesizing state machines from scenarios using interactive grammar induction. Starting from a minimal problem statement, additional requirements are gradually incorporated and the algorithm enhanced to meet them.

\item[Chapter 5] discusses how our inductive synthesis technique has been evaluated and the results obtained. It covers qualitative evaluations on case studies and more quantative ones on synthetic datasets.

\item[Chapter 6] discusses Stamina, our online plateform for evaluating inductive state machine synthesis. Stamina has first ran as a formal competition, whose results are also briefly summarized.

\item[Chapter 7] presents an overview of the tool support: a model checker for guarded process models, an interactive synthesizer of state machines from scenarios and a tool for modeling and analyzing critical processes such as clinical pathways. 

\item[Chapter 8] discusses related work and discusses how other synthesis techniques complement and/or differ from those of the thesis.

\item[Chapter 9] concludes the thesis and states open issues, perspectives, and future work. 

\end{description}
