\section{Requirements on multi-view modeling languages\label{section:intro-requirements}}

Complex systems are better described with multiple views. In that case, different models focus on particular facets of the system, i.e. its intentional, structural, operational and behavioral dimensions \cite{VanLamsweerde:2009}. 

Scenarios, for example, illustrate typical interactions between agents whereas their state machines provide a complete description of their internal behavior. A process model might describe the operational decomposition of a complex work into smaller tasks whereas goals make the underlying objectives precise. And so on.

Reaching quality models in such multi-view approaches is far from an easy task. Adequate modeling languages, tools and techniques are therefore required to help building them. To play a significant role, a good multi-view modeling language should meet the following requirements \cite{VanLamsweerde:2009}:

\begin{itemize}
\item \emph{Multi-level}: the system should be captured at different levels of abstraction and precision to enable stepwise elaboration and validation;
\item \emph{Analyzable}: the modeling abstractions should be accurate enough to support useful forms of analysis.
\end{itemize}

In addition, such modeling language should support stakeholders and analysts in building quality models. A ``good'' model should meet the following requirements:

\begin{itemize}
\item \emph{Adequate}: the models should adequately represent the essence of the target system while abstracting unnecessary details;
\item \emph{Complete}: the models should capture all pertinent facets of the system along the \textsc{why-}, \textsc{what-} and \textsc{how-} dimensions;
\item \emph{Precise}: the models should be accurate enough to capture system descriptions with as less ambiguity as possible;
\item \emph{Consistent}: the models should agree on their overlapping descriptions of the system;
\item \emph{Comprehensible}: the models should be easy enough to understand by the people who need to use them;
\end{itemize}

Formal modeling approaches help meeting the requirements above. The semantics of a formal modeling language provides precise rules of interpretation that allow many of the problems with natural language to be overcome. Formal specifications may also be manipulated by automated tools for a wide variety of purposes. Among them, model \emph{analysis} and model \emph{synthesis} are intertwined activities that help reaching quality models, mostly along the completeness, consistency and precision requirements.

This thesis investigates model synthesis more specifically whereas its companion one, by Christophe Damas, investigates model analysis \cite{Damas:2011}.

