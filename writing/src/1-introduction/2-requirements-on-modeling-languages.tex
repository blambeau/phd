\section{Requirements on multi-view modeling languages\label{section:intro-requirements}}

Complex systems are better described through multiple views. Different models focus on different facets of the system along its intentional, structural, operational and behavioral dimensions \cite{VanLamsweerde:2009}. 

Scenarios, for example, may illustrate typical interactions among agent instances whereas state machines provide a complete description of agent behaviors. A process model may describe the operational decomposition of a complex work into smaller tasks. Goals may make the system objectives precise and articulate them from high-level concerns to fine-grained details.

Building high-quality models in such multi-view approaches is far from easy. Suitable modeling languages, tools and techniques are therefore required to help in this task. To play a significant role, a good multi-view modeling language should meet the following requirements \cite{VanLamsweerde:2009}:

\begin{itemize}
\item \emph{Multi-level}: the system should be captured at different levels of abstraction and precision to enable stepwise elaboration and validation;
\item \emph{Analyzable}: the modeling abstractions should be accurate enough to support useful forms of analysis.
\end{itemize}

In addition, a modeling language should support stakeholders and analysts in building high-quality models. A ``good'' model should meet the following requirements:

\begin{itemize}
\item \emph{Adequate}: the models should adequately represent the essence of the target system while abstracting from unnecessary details;
\item \emph{Complete}: the models should capture all pertinent facets of the system along the \textsc{why-}, \textsc{what-} and \textsc{how-} dimensions;
\item \emph{Precise}: the models should be accurate enough to capture system descriptions with as little ambiguity as possible;
\item \emph{Consistent}: the models should agree on their overlapping descriptions of the system;
\item \emph{Comprehensible}: the models should be easy enough to understand by the people who need to use them;
\end{itemize}

Formal modeling approaches help meeting the above requirements. The semantics of a formal modeling language provides precise rules of interpretation that allow many of the problems with natural language to be overcome. Formal specifications may also be manipulated by automated tools for a wide variety of purposes. 

In a disciplined approach, model \emph{analysis} and model \emph{synthesis} are intertwined activities aimed at reaching high-quality models, especially towards increased completeness, consistency and precision.

This thesis investigates model synthesis. A companion thesis by Christophe Damas investigates model analysis more specifically \cite{Damas:2011}.
