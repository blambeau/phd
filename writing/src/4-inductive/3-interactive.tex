\section{Interactive LTS synthesis from MSC collections\label{section:lts-induction-from-mscs}}

Algorithm~\ref{QSM} gives the pseudo-code of the \textsc{QSM} algorithm, a Query driven State-Merging DFA induction technique. \textsc{QSM} takes a scenario collection as input and produces a consistent DFA as output. The completion of the initial scenario collection with classified scenarios that are generated during learning is another output of the algorithm. The input collection must contain at least one positive scenario. The generated DFA covers all positive scenarios in the final collection and excludes all negative ones. 

The induction process can be described as follows: it starts by constructing an initial DFA covering all positive scenarios only. The induced DFA is then successively generalized under the control of the available negative scenarios and newly generated scenarios classified by the end-user. This generalization is carried out by successively merging well-selected state pairs of the initial automaton. The induction process is such that, at any step, the current DFA covers all positive scenarios and excludes all negative ones, including the interactively classified ones. In the sequel, a DFA is said to be \textsl{compatible} with respect to a set of scenarios if it covers all positive scenarios in that set and excludes all negative ones. By extension, two states are said compatible for merging (resp. incompatible) if the quotient DFA which results from their merging is compatible (resp. incompatible) with the current set of scenarios.

\begin{algorithm}[H]
{
\textbf{Algorithm} \textsc{QSM}\\
\KwIn{A non-empty initial scenario collection $(S_+, S_-)$}
\KwOut{A DFA $A$ consistent with an extended collection $(S_+, S_-)$}
$A \leftarrow $ {\tt Initialize($S_+, S_-$)}\\
\While{$(q,q') \leftarrow $ {\tt ChooseStatePairs($A$)}}{
$A_{new} \leftarrow$ {\tt Merge$(A,q,q')$}\\
\If{{\tt Compatible$(A_{new},S_+,S_-)$}}{
 \While{$Query \leftarrow $ {\tt GenerateQuery($A,A_{new}$)}}{
   \If{{\tt CheckWithEndUser($Query$)}}{
     $S_+ \leftarrow S_+ \cup Query$
   }\Else{
     $S_- \leftarrow S_- \cup Query$\\
     \Return{\textsc{QSM}$(S_+, S_-)$}
   }
 }
 $A \leftarrow A_{new}$
}
}
\Return{$A,(S_+, S_-)$}
}
\caption{\textsc{QSM}, an interactive state-merging algorithm with membership queries\label{QSM}}.
\end{algorithm}

The \texttt{Initialize} function of \textsc{QSM} returns an initial candidate automaton built from $S_+$ and\footnote{As explained in section~\ref{prelim:scenarios} any proper prefix of a negative scenario is a positive scenario. Hence the initial PTA also stores all the proper prefixes of the negative scenarios. Besides, an augmented $PTA(S_+,S_-)$ with positive and negative strings is considered in section~\ref{BlueFringe}.} $S_-$. Next, pairs of states are iteratively chosen from the current solution according to the \texttt{ChooseStatePairs} function\footnote{The assignment in the corresponding \textbf{while} loop is assumed to be \textit{true} whenever a valid state pairs is returned by \texttt{ChooseStatePairs}. When no more state pairs are considered for merging, \texttt{ChooseStatePairs} returns \textit{(nil,nil)} and the assignment is evaluated to \textit{false.} This abuse of notation in the pseudo-code allows to keep it more concise. A similar remark also applies to the inner \textbf{while} loop of the \textsc{QSM} algorithm.}. The quotient automaton obtained by merging such states, and possibly some additional states, is computed by the \texttt{Merge} function. The compatibility of this quotient automaton with the learning sample is then checked by the \texttt{Compatible} function using available negative scenarios. When compatible, new scenarios are generated through the \texttt{GenerateQuery} function and submitted to the end-user for classification (see section~\ref{QSM:query}). Scenarios classified as positive or negative are added to the initial collection with their respective labels. If all generated scenarios are classified as positive, the quotient automaton becomes the current candidate solution. The process is iterated until no more pair of states can be considered for merging. When a generated scenario is classified as negative, the algorithm is recursively called on the extended scenario collection.

The original RPNI algorithm can be seen as a particular instance of \textsc{QSM} when no query is generated or, equivalently, without the inner \textbf{while} loop. The advantage of \textsc{QSM} is that a finer control of the generalization offered by the state-merging operations can be obtained by validating these generalizations with an oracle. Section~\ref{QSM:merging} describes the general process of merging compatible state pairs while section~\ref{QSM:query} focuses on the generation of queries submitted to the end-user. Section~\ref{BlueFringe} discusses the adaptation of \textsc{QSM} to the Blue-Fringe strategy.

\subsection{Merging compatible state pairs\label{QSM:merging}}

\begin{figure}[H]
\centering
\scalebox{.45}{\includegraphics*{src/4-inductive/images/InitScenA}}
\scalebox{.45}{\includegraphics*{src/4-inductive/images/InitScenB}}
\scalebox{.45}{\includegraphics*{src/4-inductive/images/InitScenC}}
\scalebox{.45}{\includegraphics*{src/4-inductive/images/InitScenD}}
\caption{Initial positive and negative scenarios for a train system\label{Fig:init:scen}.}
\end{figure}

The various functions which control how merging is performed from an initial automaton are described below. 

\begin{description}
\item[Initialize] The \texttt{Initialize} function returns the prefix tree acceptor built from $S_+$ and the proper prefixes from $S_-$. The PTA built from the initial scenario collection in Figure~\ref{Fig:init:scen} is shown on top of Figure~\ref{Fig:algo:steps}. According to the modeling hypothesis discussed in section~\ref{prelim:LTS} (all states in a LTS are accepting states), all PTA states are labeled as accepting. This is a specificity of the application domain considered here since any prefix of a positive scenario is also a positive scenario. Hence a single positive scenario defines several positive examples. The algorithm proposed here would actually work for arbitrary positive samples not satisfying this property.

\item[ChooseStatePairs] The candidate solution is refined by merging well-selected state pairs. The \texttt{ChooseStatePairs} function determines which pairs to consider for such merging. It relies on the standard order $<$ on strings. Each state of $PTA(S_+)$ can be labeled by its unique prefix from the initial state. Since prefixes can be sorted according to that order, the states can be ranked accordingly. For example, the PTA states in Fig.~\ref{Fig:algo:steps} are labeled by their rank according to this order. The algorithm considers states $q$ of $PTA(S_+)$ in increasing order. The state pairs considered for merging only involve such state $q$ and any state $q'$ of lower rank. The $q'$ states are considered in increasing order as well. This particular ordering is specific to the original RPNI algorithm.

\item[Merge] The \texttt{Merge} function merges the two states $(q, q')$ selected in order to compute a quotient automaton, that is, to generalize the current set of positive examples. In the example of Fig.~\ref{Fig:algo:steps}, we assume that states 0, 1, and 2 were previously determined not to be compatible for merging (through negative scenarios initially submitted or generated scenarios that were rejected by the user). Merging a candidate state pair may produce a non-deterministic automaton (NFA). For example, after having merged $q = 3$ and $q' = 0$ in the upper part of Fig.~\ref{Fig:algo:steps}, two transitions labeled \texttt{start} from state 0 lead to states 2 and 6, respectively. In such a case, the \texttt{Merge} function merges states 2 and 6 and, recursively, any further pair of states that introduces non-determinism. 

We call \textsl{merging for determinization} this recursive operation of removing non-determinism. This operation guarantees that the current solution at any step is a DFA. It produces an automaton which may accept a more general language than the NFA it starts from and, as such, it is not equivalent to the standard algorithm to transform a NFA into a DFA accepting the same language~\cite{Hopcroft:1979}. 
The time complexity of merging for determinization is a linear function of the number of states of the NFA it starts from. 
When two states are merged, the rank of the resulting state is defined as the lowest rank of the pair; in particular, the rank of the merged state when merging $q$ and $q'$ is defined as the rank of $q'$ by construction. If no compatible merging can be found between $q$ and any of its predecessor states according to $<$, state $q$ is said to be \textsl{consolidated} (in the example, states 0, 1, and 2 are consolidated).

\item[Compatible] The \texttt{Compatible} function checks whether the automaton $A_{new}$ correctly rejects all negative scenarios. As seen in Algorithm.~\ref{QSM}, the quotient automaton is discarded by \textsc{QSM} when it is detected not to be compatible with the negative sample.

\begin{figure}[H]
\hspace*{-1cm}
\begin{center}
\scalebox{.63}{\includegraphics*{src/4-inductive/images/AlgoSteps}}
\end{center}
\caption{A typical induction step of the \textsc{QSM} algorithm\label{Fig:algo:steps}.}
\end{figure}

\end{description}

\subsection{Generating queries submitted to the end-user\label{QSM:query}}

This section describes how membership queries are generated in the \textsc{QSM} algorithm and how the answers provided by the end-user are processed. A complexity analysis of this algorithm is provided afterwards.
   
\begin{description}
\item[GenerateQuery] When an intermediate solution is compatible with the available scenarios, new scenarios are generated for classification by the end-user as positive or negative. The aim is to avoid poor generalizations of the learned language. The notion of characteristic sample drives the identification of which new scenarios should be generated as queries. Recall from section~\ref{Characteristic} that a sample is characteristic of a language $L$, called here the target language, if it contains enough positive and negative information. On the one hand, the required positive information is the set of short prefixes $Sp(L)$ which form the shortest histories leading to each target DFA state. This positive information must also include all elements of the kernel $N(L)$ which represents all system transitions, that is, all shortest histories followed by any admissible event. If such positive information is available, the target machine can always be derived from the PTA by an appropriate set of merging operations. On the other hand, the negative scenarios provide the necessary information to make incompatible the merging of states which should be kept distinct. A negative scenario which excludes the merging of a state pair $(q, q')$ can be simply made of the shortest history leading to $q'$ followed by any suffix, \textit{i.e.} any valid continuation, from state $q$ as detailed below.

Consider the current solution of our induction algorithm when a pair of states $(q, q')$ is selected for merging. By construction, $q'$ is always a consolidated state at this step of the algorithm; that is, $q'$ is considered to be in $Sp(L)$. State $q$ is always both the root of a tree and the child of a consolidated state. In other words, $q$ is situated at one letter of a consolidated state, that is, $q$ is considered to be in $N(L)$. States $q$ and $q'$ are compatible according to the available negative scenarios; they would be merged by the standard RPNI algorithm. In our extension, the tool will first confirm or infirm the compatibility of $q$ and $q'$ by generating scenarios to be classified by the end-user. The generated scenarios are constructed as follows.

\begin{figure}[H]
\centering
\scalebox{.5}{\includegraphics*{src/4-inductive/images/GenerateQuestion_bw}}
\caption{A new scenario to be classified by the end-user\label{Fig:generated:question}.}
\end{figure}

Let $A$ denote the current solution, $L(A)$ the language generated by $A$, and $A_{new}$ the quotient automaton computed by the \texttt{Merge} function at some given step. Let $x \in Sp(L)$ and $y \in N(L)$ denote the short prefixes of $q'$ and $q$ in A, respectively. Let $u \in L(A)/y$ denote a suffix of $q$ in $A$. A generated scenario is a string $xu$ such that $xu \in L(A_{new})\setminus L(A)$. This string can be further decomposed as $xvw$ such that $xv \in L(A)$. A generated scenario $xu$ is thus constructed as the short prefix of $q'$ concatenated with a suffix of $q$ in the current solution, provided the entire behavior is not yet accepted by $A$. Such scenario is made of two parts: the first part $xv$ is an already accepted behavior whereas the second part $w$ provides a continuation to be checked for acceptance by the end-user. When submitted to the end-user, the generated scenario can always be rephrased as a question: after having executed the first episode ($xv$), can the system continue with the second episode ($w$)? Consider the example in Fig.~\ref{Fig:algo:steps} with selected state pair $q=3, q'=0$. As $q'$ is the root of the PTA, its short prefix is the empty string. The suffixes of $q$ here yield one generated question (Fig.~\ref{Fig:generated:question}), which can be rephrased as follows: when having started and stopped the train, can the controller restart it? One can see that the first episode of this scenario in Fig.~\ref{Fig:algo:steps} is already accepted by $A$ whereas the entire behavior is accepted in $A_{new}$. The suffixes selected by our tool for generating queries are always the entire branches of the tree rooted at $q$. The aim is to help the end-user to determine more easily whether the generated scenario should be rejected. The boundary between the first ($xv$) and second ($w$) episodes of this scenario can be easily determined by comparing $A$ and $A_{new}$ as a side product of the merging for determinization implemented in the \texttt{Merge} function.

\item[CheckWithEndUser] When a new scenario is generated, it is submitted as a membership query to the end-user. If the end-user classifies the $Query$ as positive, it is added to the collection of positive scenarios. This addition changes the search space as it extends $S_+$ and consequently $PTA(S_+)$. However, this extension is implicit as the new solution $A_{new}$ is, by construction, also a quotient automaton of this extended PTA. When the $Query$ is classified as negative the induction process is recursively started\footnote{A more sophisticated strategy to update the current solution in such a case is mentioned in section~\ref{related}.} on the extended scenario collection.
\end{description}

The QSM algorithm has a polynomial time complexity in the size of the learning sample. An upper bound on the time complexity can be derived as follows.

Let $n\in \cal{O}(||S_+||+||S_-||)$ denote the number of states of the PTA built from the initial collection of scenarios. For a fixed collection of scenarios, there are ${\cal O}(n^2)$ state pairs which are considered for merging. The \texttt{Merge} and \texttt{Compatible} functions have a time complexity linear in $n$. The \texttt{GenerateQuery} is a side product of the \texttt{Merge} function and does not change its complexity. The function \texttt{CheckWithEndUser} is assumed to run in constant time. Hence, for a fixed scenario collection, the time complexity is the same as for the RPNI algorithm and is upper bounded by ${\cal O}(n^3)$. This bound is obviously not very tight. It assumes that all pairs of states considered by \texttt{ChooseStatePairs} appears to be incompatible, which is a very pessimistic assumption. Practical experiments often show that the actual complexity is much closer to the lower bound $\Omega(n)$. 

The global complexity of QSM depends on the number of recursive calls, that is, the number of times a new scenario submitted to the end-user is classified as negative. The way new scenarios are generated by the \texttt{GenerateQuery} function guarantees that the PTA built from the extended scenario collection has at most ${\cal O}(n^2)$ states.During the whole incremental learning process, there is at most one query for each transition in this tree. Consequently, the number of queries is bounded by ${\cal O}(n^2)$.

When QSM received a characteristic sample in the initial scenario collection (or any scenario collection considered when calling it recursively), it is guaranteed that no additional scenario can be classified as negative. It follows that QSM will not be called recursively anymore and stops by returning the target model. Note that the size of such a characteristic sample is not necessarily reduced by the fact that any prefix of a positive scenario is also a positive scenario since the number of negative examples it must contain is not affected by this property. An experimental study of the actual sample size required to observe the convergence of \textsc{QSM} and the number of queries submitted to the end-user is detailed in section~\ref{artificial:data}.

\subsection{Reducing the number of queries with blue-fringe\label{BlueFringe}}

The order in which states are considered for merging by the \texttt{ChooseStatePairs} function described in section~\ref{QSM:merging} follows from the implicit assumption that the current sample is characteristic. Consequently, two states are considered compatible for merging if there is no suffix to distinguish among them. This can lead to a significant number of scenarios being generated to the end-user, to avoid poor generalizations, when the initial sample is sparse and actually not characteristic for the target global LTS. To overcome this problem, our tool implements an optimized strategy known as Blue-Fringe~\cite{Lang:1998}. The difference lies in the way state pairs are considered for merging. The general idea is to early detect incompatible state pairs and, subsequently, first consider state pairs for which compatibility has the highest chance to be confirmed by the user through positive classification. The resulting "please confirm" interaction may also appear more appealing to the user.

\begin{figure}[H]
\hspace*{-1cm}
\scalebox{.7}{\includegraphics*{src/4-inductive/images/BlueFringe_bw}}
\vspace*{-.5cm}
\caption{Consolidated states (red) and states on the fringe (blue) in a temporary solution\label{Fig:BlueFringe}.}
\end{figure}

\noindent
 Fig.~\ref{Fig:BlueFringe} gives a typical example of a temporary solution produced by the original algorithm. Three state classes can be distinguished in this DFA. The red states are the consolidated ones (0, 1 and 2 in this example). Outgoing transitions from red states lead to blue states unless the latter have already been labeled as red. Blue states (4 and 5 in this case) form the blue fringe. All other states are white states. 

The original \texttt{ChooseStatePairs} function described in section~\ref{QSM:merging} considers the lowest-rank blue state first (state 4 here) for merge with the lowest-rank red state (0). When this choice leads to a compatible quotient automaton, generated scenarios are submitted to the end-user (in this case, a scenario equivalent to the string \texttt{alarm propagated, emergency stop, emergency open}). The above strategy may lead to multiple queries being generated to avoid poor generalization. Moreover, such queries may be non-intuitive for the user, \textit{e.g.} the \texttt{alarm propagated} event is sent to the train controller without having been fired by the \texttt{alarm pressed} event to the sensor.

To select a state pair for merging, the Blue-Fringe strategy evaluates all (red, blue) state pairs first. The \texttt{ChooseStatePairs} function now calls the \texttt{Merge} and \texttt{Compatible} functions before selecting the next state pair. If a blue state is found to be incompatible with all current red states, it is immediately promoted to red; the blue fringe is updated accordingly and the process of evaluating all (red, blue) pairs is iterated. When no blue state is found to be incompatible with red states, the most compatible (red, blue) pair is selected for merging. Note that the \texttt{Initialize} function now returns an augmented prefix tree acceptor $PTA(S_+, S_-)$. It stores the prefixes of all positive and negative strings, with accepting states being labeled either as positive or negative. The \texttt{Compatible} function now returns a compatibility score instead of a boolean value. The score is defined as $-\infty$ when, in the merging process for determinization, merging the current (red, blue) pair requires some positive accepting state to be merged with some negative accepting state; this score indicates an incompatible merging. Otherwise, the compatibility score measures how many accepting states in this process share the same label (either + or -). The (red, blue) pair with the highest compatibility score is considered first.  The above strategy can be further refined with a compatibility threshold $\alpha$ as additional input parameter. Two states are considered to be compatible if their compatibility score is above that threshold. This additional parameter controls the level of generalization since increasing $\alpha$ decreases the number of state pairs that are considered compatible for merging; it thus decreases the number of generated queries.

On the simple train example of this paper, the QSM algorithm with the original RPNI state-merging order learns the global LTS correctly by submitting 20 scenarios to the end-user (17 should be rejected and only 3 should be accepted). With the Blue-Fringe strategy, the same LTS is synthesized with only 3 scenarios being submitted (one to be rejected and two to be accepted). Further comparative results are detailed in section~\ref{results}.
