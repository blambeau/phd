\section{Achieving multi-view consistency\label{section:inductive-mutliview-consistency}}

The interactive QSM algorithm described in section~\ref{section:lts-induction-from-mscs} always provides a System LTS consistent with the available positive and negative scenarios. The Blue-Fringe strategy can also be applied to reduce the number of additional scenarios submitted to the end-user. The latter strategy relies on two equivalence classes partitioning the states of an augmented PTA. These classes correspond to the accepting states and the error states, respectively. All states belonging to the same class are not necessarily merged in the final solution; however, the \texttt{Compatible} function guarantees that only states belonging to the same class \emph{can} be merged.

This approach can be extended to achieve multi-view consistency, by incorporating various sources of information. This information refines the equivalence partition and further constrains the compatible merging operations. The resulting approach has many advantages; it speeds up the search; it provides strong consistency of the System LTS with other views; it reduces the number of scenario queries in the interactive setting.

Section \ref{subsection:induction-pruning-with-domain-knowledge} shows how to incorporate descriptive knowledge from fluent definitions, lts models of legacy components and domain properties. Section \ref{subsection:induction-pruning-with-goals} shows how to inject prescriptive knowledge from \textsl{goals}.

The optimization techniques detailed hereafter are based on various equivalence relations on system states. We use the term equivalence relation here in its usual mathematical sense, that is, a symmetric, reflexive, and transitive binary relation over states. The general principle underlying our techniques is the following:
\begin{quote}
\emph{Two states will be considered for merging if they agree according to all considered equivalence relations.}
\end{quote}
The equivalence relations considered in the following sections are all invariant under state merging; a state derived by merging some states simply inherits their relation. This allows each relation to be computed only once on the initial PTA. The results of such pre-processing are kept as annotations on PTA states. The above principle for state merging is very general. It could therefore be further instantiated to other equivalence relations not considered here.


\subsection{Injecting domain knowledge in the synthesis process\label{subsection:induction-pruning-with-domain-knowledge}}

\subsubsection*{Propagating fluents}

The notion of fluent has been introduced in~\cite{Giannakopoulou:2003}.

\begin{definition}[Fluent]
A \emph{fluent} 
is a proposition defined by a set $Init_{Fl}$ of initiating events, a set $Term_{Fl}$ of 
terminating events, with $Init_{Fl}\cap Term_{Fl} = \emptyset$, and an initial boolean value.
\end{definition}

\noindent
For example, the fluent \emph{DoorsClosed} describes states of the train doors as being either \\
\centerline{\texttt{closed} ($DoorsClosed = true$) or \texttt{open} ($DoorsClosed = false$),} \\
and describes which event is responsible for which state change:

\hspace*{-.5cm}
\begin{small}
\begin{tabular}{lr}
\emph{DoorsClosed} $=\langle\{$\texttt{close doors}$\},$ &$\{$\texttt{open doors, emergency open}$\}\rangle$\\
&initially $true$.\\
\end{tabular}
\end{small}

The value of every fluent can be computed on each PTA state by symbolic execution, starting from the initial state associated with the initial value for each fluent. The PTA states are then decorated with the conjunction of such values. Two states in $PTA(S_+, S_-)$ belong to the same equivalence class if they have the same value for every fluent. The decoration of the merged states is simply inherited from the states being merged. Figure~\ref{Fig:fluents} shows the result of propagating the values of the fluent \emph{DoorsClosed} along the augmented PTA built from the scenarios described in Figure~\ref{Fig:init:scen}.

\begin{figure}[H]
\centering
\scalebox{.6}{\includegraphics*{src/4-inductive/images/FluentPropagation}}
\caption{Propagating fluents\label{Fig:fluents}. {\small dc is a shorthand for \emph{DoorsClosed}; \texttt{a.pres}, \texttt{a.prop}, \texttt{e.open} stand for \texttt{alarm pressed, alarm propagated, emergency open} respectively}.}
\end{figure}

\subsubsection*{Unfolding models of external components}

Quite often the components being modeled need to interact with other components in their environment - \textit{e.g.}, legacy components in a bigger existing system, foreign components in an open system, etc. In such cases the behavior of external components is generally known - typically, through some behavioral model \cite{Hall:2004}. Here we assume that external components are known by their LTS model. For example, Figure~\ref{Fig:AlarmSensor} shows the LTS for a legacy alarm sensor in our train system. When the alarm button is pressed by a passenger, this component propagates a corresponding signal to the train controller. 

The PTA states can also be decorated with state labels from this external LTS by unfolding the latter on the PTA. Such decoration is performed by jointly visiting the PTA and the external LTS. The latter synchronizes on shared events and stays in its current state on other events. Figure~\ref{Fig:AlarmSensor:Unfolding} shows the result of unfolding the alarm sensor LTS from Fig.~\ref{Fig:AlarmSensor} on the augmented PTA built from the scenarios described in Fig.~\ref{Fig:init:scen}. Each state in Fig.~\ref{Fig:AlarmSensor:Unfolding} is labeled with the number of the corresponding state in the alarm sensor LTS. Two states belong to the same equivalence class if they have the same external LTS state label.

\begin{figure}[H]
\centering
\scalebox{.5}{\includegraphics*{src/4-inductive/images/AlarmSensor}}
\caption{Alarm sensor LTS\label{Fig:AlarmSensor}.}
\end{figure}

\begin{figure}[H]
\centering
\scalebox{.6}{\includegraphics*{src/4-inductive/images/AlarmSensorUnfold}}
\caption{Unfolding the Alarm sensor model\label{Fig:AlarmSensor:Unfolding}.}
\end{figure}

\subsection{Injecting goals in the synthesis process\label{subsection:induction-pruning-with-goals}}

Descriptive statements about the domain, called \textsl{domain properties}, or prescriptive statements of intent about the target system, called \textsl{goals}, can be expressed declaratively with fluents in \textsl{Linear Temporal Logic} (LTL)~\cite{Giannakopoulou:2003}. LTL assertions use standard operators for temporal referencing such as:

\begin{center}
$\square$ (always in the future), $\diamond$ (some time in the future),\\
$\rightarrow$ (implies in the current state), $\Rightarrow$ (always implies).\\
\end{center}

For example, in an extended version of our running example, the statement

\begin{center}
	$\square (HighSpeed \rightarrow Moving)$
\end{center}

expresses a physical law stating that all scenarios for which a train is running at high speed while not moving should be considered as negative. 

A goal requiring train doors to remain closed while the train is moving is formalized as 
\begin{center}
$ DoorsClosedWhileMoving = \square (Moving \rightarrow DoorsClosed)$
\end{center}

We restrict here our attention to statements that can be formalized as LTL safety properties. For such properties, a tester can be automatically generated \cite{Giannakopoulou:2003}. A tester for a property is a LTS extended with a negative accepting state such that every path leading to this state violates the property. The tester LTS for the goal $DoorsClosedWhileMoving$ is represented in Figure~\ref{Fig:Goals:Injection} (the black state is the negative accepting state). Any event sequence leading to the black state from the initial state corresponds to an undesired system behavior. In particular, the event sequence \texttt{start, open} corresponds to the initial negative scenario in our running example (see Fig.\ref{Fig:init:scen}). The tester of Figure~\ref{Fig:Goals:Injection} provides many more negative scenarios, actually an infinite number of negative scenarios due to the cyclic nature of such LTS. The property tester is used to constrain the induction process in the same way as an external component LTS. The PTA and the tester are traversed jointly in order to decorate each PTA state with the corresponding tester state. Two states belong in the same equivalence class if they have the same property tester state. This technique has the additional benefit of ensuring that the synthesized global LTS satisfies the considered goal or domain property. 

\begin{figure}[H]
\centering
\scalebox{.5}{\includegraphics*{src/4-inductive/images/GoalsInjection}}
\caption{Tester LTS for the goal $DoorsClosedWhileMoving$\label{Fig:Goals:Injection}. {\small \texttt{e.stop}, \texttt{e.open} stand for \texttt{emergency stop, emergency open} respectively}.}
\end{figure}
