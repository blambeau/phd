\section{State machines as labeled transition systems\label{section:background-state-machines}}

In our framework, the behavior of an agent will be modeled by a specific kind of finite state machine, called \emph{labeled transition system} (LTS). This formalism, initially introduced by Keller for reasoning about parallel programs~\cite{Keller:1976}, has been intensively used for specifying and analyzing concurrent systems, e.g. in~\cite{Milner:1989, Clarke:1989, Magee:1997}. 

A LTS is made of a set of states and a set of transitions between them (see Fig.~\ref{image:framework-start-stop}). Each transition has an \emph{event} label -- sometimes called \emph{symbol} or \emph{action} label. A state is labeled with one number to distinguish if from other states. An \emph{initial state} is designated graphically by an incoming arrow with no source state (e.g. state 0 in the figure). 

\vspace{0.5cm}
\begin{figure}[H]
\centering\scalebox{0.60}{
  \includegraphics*[clip]{src/2-framework/images/start-stop}}
  \caption{A Labeled Transition System for an \artifact{Engine} agent\label{image:framework-start-stop}.}
\end{figure}

Mathematically, a LTS is defined as a 4-tuple $(Q,\Sigma,\delta,q_{init})$ where 

\begin{itemize}
\item $Q$ is a finite set of states,
\item $\Sigma$ is a set of labels called its \emph{alphabet}, 
\item $\delta$ is a transition relation $Q \times \Sigma\cup\{\tau\} \times Q$,
\item $q_{init} \in Q$ is the initial state. 
\end{itemize}

The \emph{alphabet} $\Sigma$ captures the notion of \emph{agent interface}, as a set of event labels that an agent recognizes. In other words, the events in which the agent \emph{engages} in synchronous communications with its environment. For example, the LTS of Fig.~\ref{image:framework-start-stop} has an alphabet \artifact{$\Sigma=\{start, stop\}$}. 

Note that labeled transition systems do not distinguish between \emph{sent} and \emph{received} events. Such distinction may be required when connecting LTS and scenarios; in such cases, we assume this structural information to be available elsewhere, typically, from a context or architecture diagram~\cite{Ward:1985, Magee:1995}. We will assume that an event label uniquely determines the interacting agents; an event may however be received by more than one agent. 

In the thesis $\Sigma^{*}$ denotes the set of finite sequences over an alphabet $\Sigma$. A \emph{trace} is an element of $\Sigma^{*}$, that is, a finite sequence of event labels \artifact{$w = \textless l_0,\ldots,l_{n} \textgreater$} with $l_i \in \Sigma$. We will sometimes use the notation $vw$ to denote the concatenation of a trace $v$ with another trace $w$.

The label $\tau$ is used to denote so-called \emph{non-observable} transitions. Such transitions allow us to model agent state changes that cannot be monitored by its environment. We will denote $\Sigma\cup\{\tau\}$ by $\Sigma_{\tau}$ (an alphabet augmented with the $\tau$ label).

A LTS is \emph{deterministic} if any trace uniquely determines the reached state; otherwise it is \emph{non-deterministic}. Therefore, a deterministic LTS may not have $\tau$ transitions; neither does it have a state with two outgoing transitions sharing the same label, that is, 

\begin{center}$(q,l,q_1) \in \delta \wedge (q,l,q_2) \in \delta \implies q_1 = q_2$\end{center}

A state is \emph{terminating} if it has no outgoing transition; otherwise it is \emph{non-terminating}. A \emph{terminating} LTS has at least one terminating state; otherwise it is \emph{non-terminating}. 

Note that, the above definition does not allow distinguishing between terminating states that model successful termination (an agent stops running intentionally) and non-successful termination (e.g. an agent composed from finer-grained agents \emph{deadlocks} unintentionally). While relevant for system analysis, this thesis do not consider deadlocks.

A finite LTS \emph{execution} is a finite sequence of states separated by labels: 

\begin{center}
\artifact{$w = \textless q_0,l_0,\ldots,q_{n-1},l_{n-1},q_n \textgreater$} 
\end{center}

\noindent with $q_i \in Q$ and $l_i \in \Sigma_{\tau}$. 

An execution is \emph{valid} for a LTS if it denotes a path from the initial state in the corresponding graph:

\begin{center}
$q_0 = q_{init}$ and $(q_i,l_i,q_{i+1}) \in \delta$ for $0 \leq i < n$. 
\end{center}

The \emph{projection} of an execution $w$ over an alphabet $\Sigma$, denoted by $w|_{\Sigma}$, is the result of keeping, from $w$ only those event labels that belong to $\Sigma$ (in other words, eliminating $q_i$ states and $\tau$ labels). In other words, the projection of an execution defines a trace. 

A trace $t$ is \emph{accepted} by a LTS if there exists a valid execution $w$ such that $w|_{\Sigma} = t$. In other words, a trace is accepted by a LTS if it denotes an existing path in the corresponding graph from the initial state, possibly with silent moves offered by $\tau$ transitions in the non-deterministic case (in this case a trace may actually denote more that one existing path). As a consequence of this definition, a prefix of an accepted trace is an accepted trace as well; the empty trace $\lambda$ is therefore always accepted. For example, the LTS of Fig.~\ref{image:framework-start-stop} accepts the trace \artifact{<start stop start>}, the trace \artifact{<start stop>} as well, but not \artifact{<start start>}. 

The set of traces accepted by a LTS $P$ is called its \emph{language}; it is denoted by $\mathcal{L}(P)$. For example, the  \emph{language} of the LTS shown in Fig.~\ref{image:framework-start-stop} is $\mathcal{L}(\artifact{Engine})=\{\lambda$, \artifact{<start>}, \artifact{<start stop>}, \artifact{<start stop start>}, \ldots $\}$. 

\emph{Behavioral equivalence} is an important notion when designing and analyzing concurrent systems. It allows answering questions such as ``\emph{are agents $Ag_1$ and $Ag_2$ the same in terms of behavior?}''. Many different notions of behavioral equivalence exist in the literature, like \emph{strong} and \emph{observational}  equivalences~\cite{Milner:1989}, bisimilarity~\cite{Park:1981}, or failure equivalence~\cite{Hoare:1985}. Assuming \emph{determinate} agents, as explained in Section~\ref{subsection:background-multiple-views}, allows us to stick to the weakest, yet simplest, notion of LTS equivalence: trace equivalence~\cite{Hoare:1985, Engelfriet:1985}. Two LTS $P$ and $Q$ are \emph{trace-equivalent} if they accept the same set of traces:

\begin{equation}
P \equiv_{tr} Q \mbox{~if and only if~} \mathcal{L}(P) = \mathcal{L}(Q)
\end{equation}

The next section discusses some important properties inherited by LTSs from their connection with regular languages. 

\subsection{Labeled transition systems and regular languages\label{section:background-lts-and-regular-languages}}

Labeled transition systems are actually a subclass of standard automata \cite{Hopcroft:1979}. A standard automaton is a 5-tuple $(Q,\Sigma,\delta,q_{init},F)$. The only difference with a LTS is that a subset $F \subseteq Q$ of its states are \emph{accepting} states. They are depicted with double circles to distinguish them from non-accepting states. Only traces that end in an accepting state are accepted by a standard automaton. 

A LTS is a standard automaton in which all states are accepting: $F = Q$. Many results from standard automata theory therefore apply to LTS while preserving trace equivalence.

Standard automata, both deterministic and non-deterministic ones, capture the well-studied class of \emph{regular} languages~\cite{Hopcroft:1979}. As they only have accepting states, LTS capture the class of \emph{prefix-closed} regular languages. The term ``prefix-closed'' means that all (non necessarily proper) prefixes of accepted traces are also accepted traces. Mathematically, $prefixes(\mathcal{L}(P)) = \mathcal{L(P)}$.

For any regular language $\mathcal{L}$, there exists a canonical automaton $A(\mathcal{L})$. This automaton is the minimal deterministic automaton accepting $\mathcal{L}$; it is known to be unique up to state renumbering~\cite{Hopcroft:1979}. 

Without loss of generality, we may therefore assume that the behavior of any agent is modeled by a canonical LTS. Moreover, we may use non-deterministic constructions ($\tau$ transitions in particular) without contradicting the assumption of \emph{determinate} agents. Given a LTS $P$, deterministic or not, we will denote by $P^{\Delta}$ its canonical equivalent, where $P$ can actually be a LTS \emph{expression}, that is, a LTS ``computed'' by applying LTS operators introduced in the next sections. Standard automaton algorithms from~\cite{Hopcroft:1979} can be used to compute $P^\Delta$. This typically involves removing $\tau$ transitions, determinizing and minimizing the LTS under trace equivalence.

As languages are sets of traces, it is sometimes natural to reason in terms of standard operators on sets. In the following sections, we will often make use of notations for the union of two languages ($\cup$), their intersection ($\cap$), subset ($\subseteq$), proper subset ($\subset$) and equality ($=$). Techniques and algorithms for implementing these operators for the general case of standard automata can be found in~\cite{Hopcroft:1979, Aho:1986}. For a given language $\mathcal{L}$, $mt(\mathcal{L})$ denotes the set of \emph{maximal} traces of $\mathcal{L}$, that is, traces that cannot be extended by a suffix within the same language. In the case of a deterministic LTS, they simply correspond to traces reaching a terminating state.

The next two sections define two additional operators, \emph{composition} and \emph{hiding}, that support reasoning about behaviors in presence of multiple agents with different alphabets.

\subsection{Systems as agent compositions\label{subsection:lts-composition}}

A system is composed of active agents whose behavior is explicitly modeled by a LTS. The behavior of the system itself is defined through parallel composition~\cite{Hoare:1985}. In this setting agents execute asynchronously but synchronize on shared events. A system made of $n$ agents is defined as:

\begin{equation}
System = Ag_1 \parallel \ldots \parallel Ag_n
\label{equation:parallel-composition}
\end{equation}

As we are mostly interested in agent \emph{behaviors}, for parallel composition we will use the binary composition operator $\parallel$ defined on LTS \cite{Giannakopoulou:1999, Magee:1999}. This operator computes the interleaving of all traces accepted by the two LTS under the constraint that they synchronize on shared labels. It is both commutative and associative, allowing us to write~(\ref{equation:parallel-composition}) without ambiguity. 

Let $P = (S_1,\Sigma_1,\delta_1,q_{1})$ and $Q = (S_2,\Sigma_2,\delta_2,q_{2})$ denote two LTS. Their \emph{composition} is another LTS 
\begin{equation}
P \parallel Q = (S_1 \times S_2,\Sigma_1\cup\Sigma_2,\delta,(q_1,q_2))
\end{equation}
\noindent where $\delta$ is the smallest relation satisfying the following rules~\cite{Giannakopoulou:1999}:

\begin{center}
\begin{tabular}{cc}
$\frac{\displaystyle P \stackrel{l}{\longrightarrow} P'}{\displaystyle P \parallel Q \stackrel{l}{\longrightarrow} P' \parallel Q}~~l \notin \Sigma_2$ &
$\frac{\displaystyle Q \stackrel{l}{\longrightarrow} Q'}{\displaystyle P \parallel Q \stackrel{l}{\longrightarrow} P \parallel Q'}~~l \notin \Sigma_1$ \\
 & \\
\multicolumn{2}{c}{$\frac{\displaystyle P \stackrel{l}{\longrightarrow} P',~Q \stackrel{l}{\longrightarrow} Q'}{\displaystyle P \parallel Q \stackrel{l}{\longrightarrow} P' \parallel Q'}~~l \neq \tau$} \\
\end{tabular}
\end{center}

In those rules, the notation $X \stackrel{l}{\longrightarrow} X'$ means that the LTS $X = (S,\Sigma,\delta,q_0)$ may \emph{transit} into another LTS $X' = (S,\Sigma,\delta,q_1)$ through the event label $l$, provided that $(q_0,l,q_1) \in \delta$. 

$P \parallel Q$ is thus defined on the Cartesian product of the sets of states of $P$ and $Q$; its initial state is $\{q_1,q_2\}$. The rules above define the possible transitions from such a state. 

\begin{itemize}
\item The first two rules are symmetric; they encode the fact that, on non-shared labels, one LTS may transit while the other stays in its previous state. Those rules allow individual LTS to move along $\tau$ transitions. 
\item The last rule forces the two LTS to transit together on all shared labels but $\tau$.
\end{itemize}

A composed LTS can easily be computed constructively by exploring the state space from its initial state until no new state pair is discovered. 

When the two LTS operands share the same alphabet, the composition operator computes the intersection of accepted traces:

\begin{equation}
\mathcal{L}(P \parallel Q) = \mathcal{L}(P) \cap \mathcal{L}(Q) \mbox{~~if~~} \Sigma_{1}=\Sigma_{2}
\end{equation}

The trace semantics of a system composed of $n$ agents whose behavior is modeled with LTSs $Ag_1 \ldots Ag_n$ is captured by:

\begin{equation}
\mathcal{L}(System) = \mathcal{L}(Ag_1 \parallel \ldots \parallel Ag_n)
\label{equation:system-composition}
\end{equation}


\subsection{Black-box behavior through event \emph{hiding}\label{subsection:lts-hiding}}

It is sometimes useful to consider the composition of a subset of agents that together define an interesting boundary in the system considered. 

Consider the agents depicted in the scenario of Fig.~\ref{image:train-scenario-all-agents}, for example. The machine boundary can simply be modeled as follows:

\vspace{-0.8cm}
\begin{align}
Machine &= Controller \parallel Actuators \parallel Sensors \\
World   &= Passenger \parallel Doors \parallel Engine \\
System  &= Machine \parallel World
\end{align}
\vspace{-0.8cm}

As seen before, the $Machine$ agent has an interface in terms of the set of events in which it engages. This interface might be too large. A black-box version of the $Machine$ behavior might be suitable; it is a LTS whose interface is restricted to those events crossing the depicted boundary. 

To support this, we introduce a simple operator called hiding. The \emph{hiding} of a set of labels $I$ in a LTS $P = (Q,\Sigma,\delta,q_{init})$ defines the LTS

\begin{equation}
P \setminus I = (Q,\Sigma \setminus I,\delta_{hidden},q_{init})
\end{equation}

\noindent where $\delta_{hidden}$ is the smallest relation satisfying the following rules~\cite{Giannakopoulou:1999}:

\begin{center}
\begin{tabular}{cc}
$\frac{\displaystyle P \stackrel{l}{\longrightarrow} P'}{\displaystyle P \setminus I \stackrel{l}{\longrightarrow} P' \setminus I}~~l \notin I$ & 
$\frac{\displaystyle P \stackrel{l}{\longrightarrow} P'}{\displaystyle P \setminus I \stackrel{\tau}{\longrightarrow} P' \setminus I}~~l \in I$ \\
\end{tabular}
\end{center}

The hiding operator thus simply makes a set of labels invisible to the environment by replacing them by $\tau$ transitions. The resulting LTS is non-deterministic; however, a minimal and deterministic equivalent exists (as discussed in the previous section). In our example, the LTS of the black-box machine we are actually looking for is:

\vspace{-0.8cm}
\begin{align}
Machine' &= (Machine \setminus Internals)^\Delta
\end{align}
\vspace{-0.8cm}

\noindent where $Machine$ is the composition between the controller, actuators and sensors given previously and $Internals$ is the set of events internal to the machine, here: $\{\artifact{start-signal}, \artifact{stop-signal}, \artifact{open-signal}, \artifact{alarm-signal}, \ldots\}$ (see Fig.~\ref{image:train-scenario-all-agents}).

Given a LTS $P = (Q,\Sigma,\delta,q_{init})$ and a set of labels $I$, the relation between the languages of $P$ and $P \setminus I$ is defined as follows:

\begin{equation}
\mathcal{L}(P \setminus I) = \{ t'~|~\exists t \in \mathcal{L}(P)~such~that~t' = t|_{\Sigma \setminus I}\}
\end{equation}

