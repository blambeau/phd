\section{State Machines as Labeled Transition Systems\label{section:background-state-machines}}

In our framework, the behavior of an agent (and the system) is modeled by a specific kind of finite state machine, called \emph{labeled transition system} (LTS). This formalism, initially introduced by Keller for reasoning about parallel programs~\cite{Keller:1976}, has since been intensively used for specifying and analyzing concurrent systems, e.g. in~\cite{Milner:1989, Clarke:1989, Magee:1997}. A LTS is made of a set of states and a set of transitions between them (see Fig.~\ref{image:framework-start-stop}). Each transition is depicted with an \emph{event} label -- sometimes called a \emph{symbol} or an \emph{action} label; also, a specific state is the \emph{initial state}, designated graphically by an empty arrow in front of it (state 0 in the figure). 

\vspace{0.5cm}
\begin{figure}[H]
\centering\scalebox{0.60}{
  \includegraphics*[clip]{src/2-framework/images/start-stop}}
  \caption{A Labeled Transition System for an \artifact{Engine} agent\label{image:framework-start-stop}.}
\end{figure}

Mathematically, a LTS is defined as a 4-tuple $(Q,\Sigma,\delta,q_{init})$ where $Q$ is a finite set of states, $\Sigma$ is a set of labels called its \emph{alphabet}, $\delta$ is a transition relation $Q \times \Sigma\cup\{\tau\} \times Q$ and $q_{init} \in Q$ is the initial state. The label $\tau$ is used to denote the so-called \emph{non-observable} transitions, which allow modeling an agent changing state without such change being monitored by its environment. 

A \emph{deterministic} LTS does not have $\tau$ transitions and has no state with two outgoing transitions having the same label (that is, $(q,l,q_1) \in \delta \wedge (q,l,q_2) \in \delta \implies q_1 = q_2$); otherwise it is \emph{non-deterministic}.

A \emph{terminating} state is one with no outgoing transition; otherwise it is \emph{non-terminating}. A \emph{terminating} LTS has at least one terminating state; otherwise it is \emph{non-terminating}. Note that, as such, the LTS definition does not allow distinguishing between terminating states that model successful termination -- an agent stops running intentionally -- and non-successful termination -- an agent, more often the system as a \emph{composed} agent (see next section), \emph{deadlocks} unintentionally.

The \emph{alphabet} $\Sigma$ captures the notion of \emph{agent interface}~\cite{Feather:1987}, as a set of event labels that an agent recognizes or, said otherwise, in which the agent \emph{engages} in synchronous communications with its environment. For example, the LTS of Fig.~\ref{image:framework-start-stop} has an alphabet \artifact{$\Sigma=\{start, stop\}$}. Note that labeled transition systems do not distinguish between \emph{sent} and \emph{received} events. This distinction being required when playing with scenarios, we assume this structural information is available elsewhere, typically from a context or architecture diagram~\cite{Ward:1985, Magee:1995}. As already stated, we also assume that an event label uniquely determines the interacting agents but allow an event to be received by more than one agents. By simplicity in the sequel, we denote $\Sigma\cup\{\tau\}$ by $\Sigma_{\tau}$ (an alphabet augmented with the $\tau$ label).

An finite LTS \emph{execution} is a finite sequence of its states separated by labels, i.e. \artifact{$w = \textless q_0,l_0,\ldots,q_{n-1},l_{n-1},q_n \textgreater$} with $q_i \in Q$ and $l_i \in \Sigma_{\tau}$. An execution is valid for a LTS if it denotes an existing path, from the initial state, in the corresponding graph; mathematically, $q_0 = q_{init}$ and $(q_i,l_i,q_{i+1}) \in \delta$ for $0 \leq i < n$. The projection of an execution $w$ over an alphabet $\Sigma$ is denoted by $w|_{\Sigma}$ and is the result of keeping, from $w$, only event labels that belong to $\Sigma$ (in other words, eliminating $q_i$ states and $\tau$ labels). Such a projection is also called a \emph{trace}, that we define now.

A \emph{trace} denotes an element of $\Sigma^*$, that is a finite sequence of event labels \artifact{$t= \textless l_0,\ldots,l_{n} \textgreater$} with $l_i \in \Sigma$. Unlike an execution, a trace never contains $\tau$ labels. A trace $t$ is accepted by a LTS if there exists a valid execution $w$ such that $w|_{\Sigma} = t$. In other words, a trace is \emph{accepted} by a LTS if it denotes an existing path in the corresponding graph from the initial path, but allowing ``in the middle'' silent moves offered by $\tau$ transitions in the non-deterministic case (and hence, possibly, more than one path). Note that, by this definition, a prefix of an accepted trace is also an accepted trace; the empty trace $\lambda$ is therefore always accepted. For example, the LTS of Fig.~\ref{image:framework-start-stop} accepts the trace \artifact{<start stop start>}, and hence \artifact{<start stop>}, but not \artifact{<start start>}. We sometimes use a dot notation $w.l$ to denote the concatenation of a trace $w$ with a label or another trace $l$.

An important notion when designing and analyzing concurrent systems is the one of \emph{behavior equivalence} that permits answering questions like ``\emph{are agents $Ag_1$ and $Ag_2$ the same in terms of behavior?}''. Many different notions of behavioral equivalence exist in the literature, like \emph{strong} and \emph{observational}  equivalences~\cite{Milner:1989}, bisimilarity~\cite{Park:1981}, or failure equivalence~\cite{Hoare:1985}. Their introduction (mostly in process algebra) is motivated by the need to distinguish between special process cases as well as being able to reason about their correctness (in terms of \emph{deadlock}, for instance), especially when dealing with non-determinism. For details, see e.g.~\cite[chap. 3]{Hoare:1985}, \cite[chap. 4 \& 5]{Milner:1989} or the overview given in~\cite{Fernandez:1991}. Our hypotheses, especially the one of \emph{determinate} agents, allows us to stick with the weakest, yet simplest, notion of LTS equivalence: \emph{trace equivalence}~\cite{Hoare:1985, Engelfriet:1985}. Under the latter, two LTS $P$ and $Q$ are equivalent, denoted by $P \equiv_{tr} Q$, if they accept the same set of traces, in other words, if they define the same \emph{language} $\mathcal{L}(P) = \mathcal{L}(Q)$. This naturally leads us to the next section, which discusses some important properties inherited by LTSs from their connection with regular languages. 

\subsection{Labeled Transition Systems and Regular Languages\label{section:background-lts-and-regular-languages}}

Labeled transition systems are actually a subclass of standard automata \cite{Hopcroft:1979}. The only difference is that the latter make an additional distinction between \emph{accepting} and \emph{non-accepting} states. In other words, a LTS can be seen as a standard automaton in which all states are accepting states. Many results from standard automata theory apply to LTS while preserving trace equivalence, that is, observable behaviors.

First, the (maybe infinite) set of traces accepted by a LTS, say $P$, is called its \emph{language} and denoted by $\mathcal{L}(P)$. For example, the  \emph{language} of the LTS shown in Fig.~\ref{image:framework-start-stop} is $\mathcal{L}(\artifact{Engine})=\{\lambda$, \artifact{<start>}, \artifact{<start stop>}, \artifact{<start stop start>}, \ldots $\}$. Standard automata, both deterministic and non-deterministic ones, capture the well-studied class of \emph{regular} languages~\cite{Hopcroft:1979} while LTS, because they have all accepting states, capture the class of \emph{prefix-closed} regular languages. The term ``prefix-closed'' means that all prefixes of accepted traces are also accepted traces, mathematically $prefixes(\mathcal{L}(P)) = \mathcal{L(P)}$.

Another notable result is the existence, for any regular language $\mathcal{L}$, of a canonical automaton $A(\mathcal{L})$. This canonical automaton is the minimal deterministic automaton accepting $\mathcal{L}$, and is known to be unique up to state renumbering~\cite{Gold:1978}. Without loss of generality, we can therefore assume that the behavior of any agent is modeled by a canonical LTS. Moreover, these results enables the use of non-deterministic constructions -- $\tau$ transitions in particular -- without being in contradiction with our assumption of \emph{determinate} agents. Given any LTS $P$, being deterministic or not, we denote by $P^{\Delta}$ its canonical equivalent, where $P$ can actually be a LTS \emph{expression} (that is a LTS ``computed'' by applying LTS operators introduced in next sections). Standard automaton algorithms from~\cite{Hopcroft:1979} can be used to compute $P^\Delta$, that is, to remove $\tau$ transitions, determinize and/or minimize LTSs under our assumptions.

As languages are sets of traces, it is sometimes natural to reason in terms of standard operators on sets. In the following sections, we often make use of notations for the union of two languages ($\cup$), their intersection ($\cap$), subset ($\subseteq$), proper subset ($\subset$) and equality ($=$). This is especially true in the definition of inter-model consistency rules. Techniques and algorithms for implementing these operators for the general case of standard automata can be found in~\cite{Hopcroft:1979, Aho:1986}, providing an operational way of testing such rules. Also, given a language $\mathcal{L}$, we sometimes denote by $mt(\mathcal{L})$ the set of \emph{maximal} traces of $\mathcal{L}$, that is, traces that cannot be extended by a suffix within the same language. Given a deterministic LTS, they simply correspond to traces reaching a terminating state. 

In the next two sections, we define two additional operators, \emph{composition} and \emph{hiding}, that allow reasoning about behaviors in presence of multiple agents with different alphabets.

\subsection{System as Agent composition}

If a system is composed of active agents and the behavior of each of these agents is explicitly modeled with an LTS, one can ask what is the behavior of the system itself. We define it through parallel composition~\cite{Hoare:1985}, a setting where agents execute asynchronously but synchronize on shared events. Given a system made of $n$ agents, and the composition operator denoted by~$\parallel$, the system is defined as:

\begin{equation}
System = Ag_1 \parallel \ldots \parallel Ag_n
\end{equation}

As we are mostly interested in agent \emph{behaviors}, we use the binary composition operator $\parallel$ defined on LTS, see e.g.~\cite{Giannakopoulou:1999, Magee:1999}. The operator, which is both commutative and associative (allowing our writing above without ambiguity), computes the interleaving of the traces accepted by the two LTS, under the constraint that they synchronize on shared labels. Let $P = (S_1,\Sigma_1,\delta_1,q_{1})$ and $Q = (S_2,\Sigma_2,\delta_2,q_{2})$ denote two LTS. Then, their composition $P \parallel Q$ is another LTS $(S_1 \times S_2,\Sigma_1\cup\Sigma_2,\delta,(q_1,q_2))$, where $\delta$ is the smallest relation satisfying the following rules:

\begin{center}
\begin{tabular}{cc}
$\frac{\displaystyle P \stackrel{l}{\longrightarrow} P'}{\displaystyle P \parallel Q \stackrel{l}{\longrightarrow} P' \parallel Q}~~l \notin \Sigma_2$ &
$\frac{\displaystyle Q \stackrel{l}{\longrightarrow} Q'}{\displaystyle P \parallel Q \stackrel{l}{\longrightarrow} P \parallel Q'}~~l \notin \Sigma_1$ \\
 & \\
\multicolumn{2}{c}{$\frac{\displaystyle P \stackrel{l}{\longrightarrow} P',~Q \stackrel{l}{\longrightarrow} Q'}{\displaystyle P \parallel Q \stackrel{l}{\longrightarrow} P' \parallel Q'}~~l \neq \tau$} \\
\end{tabular}
\end{center}

In these rules, the notation $X \stackrel{l}{\longrightarrow} X'$ simply denotes the fact that the LTS $X = (S,\Sigma,\delta,q_0)$ may \emph{transit} into another LTS $X' = (S,\Sigma,\delta,q_1)$ with the event label $l$, provided that $(q_0,l,q_1) \in \delta$. 

As one can see, $P \parallel Q$ is defined on the Cartesian product of the states of $P$ and $Q$, and has its initial state simply defined as $(q_1,q_2)$ in this state space. Rules above define possible transitions from such a state. The first two rules are symmetric and encode the fact that, on non shared labels, one LTS may transit while the other stays in its previous state. As stated, those rules allow individual LTS to move along $\tau$ transitions. The last rule forces the two LTS to transit together on all shared labels but $\tau$. A composed LTS can easily be computed constructively by exploring the state space from its initial state until no new state pair is discovered. 

The trace semantics of a system composed of $n$ agents whose behavior is modeled with LTSs $Ag_1$ to $Ag_n$ is captured as:

\begin{equation}
\mathcal{L}(System) = \mathcal{L}(Ag_1 \parallel \ldots \parallel Ag_n)
\label{equation:system-composition}
\end{equation}

\subsection{Black-box behavior through \emph{hiding}}

If the notion of agent composition gives a sound interpretation to the notion of \emph{system} and its behavior, it is, in fact, of slightly more general use. Indeed, as explained in the introduction of this section, it makes sense to consider not only the composition of all agents but sometimes the composition of a subset of them that, together, define an interesting boundary in the system considered. Consider the agents depicted in the scenario of Fig.~\ref{image:train-scenario-all-agents} for example. The ``machine vs. world'' boundary can simply be modeled as follows:

\vspace{-0.8cm}
\begin{align*}
Machine &= Controller \parallel Actuators \parallel Sensors \\
World   &= Passenger \parallel Doors \parallel Engine \\
System  &= Machine \parallel World
\end{align*}
\vspace{-0.8cm}

However, as defined above, the $Machine$ agent has an interface -- in terms of the set of events in which it engages -- which is actually too large. One would like to have a black-box version of the $Machine$ behavior, that is a LTS with events of the new interface only, those crossing the depicted boundary. For this, LTSs come with a simple operator called \emph{hiding}. Hiding of a set of labels $I$ in a LTS $P = (Q,\Sigma,\delta,q_{init})$ is denoted $P \setminus I$ and defines the LTS $(Q,\Sigma \setminus I,\delta_{hidden},q_{init})$ where $\delta_{hidden}$ is the smallest relation satisfying the rules:

\begin{center}
\begin{tabular}{cc}
$\frac{\displaystyle P \stackrel{l}{\longrightarrow} P'}{\displaystyle P \setminus I \stackrel{l}{\longrightarrow} P' \setminus I}~~l \notin I$ & 
$\frac{\displaystyle P \stackrel{l}{\longrightarrow} P'}{\displaystyle P \setminus I \stackrel{\tau}{\longrightarrow} P' \setminus I}~~l \in I$ \\
\end{tabular}
\end{center}

As one can see, the operator simply makes a set of labels invisible from the environment by replacing them by $\tau$. The resulting LTS is non-deterministic, but results from previous sections ensure that a minimal and deterministic equivalent exists. That is, the LTS of the black-box machine we are actually looking for is the following:

\vspace{-0.8cm}
\begin{align*}
Machine' &= (Machine \setminus Internals)^\Delta
\end{align*}
\vspace{-0.8cm}

\noindent where $Machine$ is the composition between the controller, actuators and sensors given previously and $Internals$ is the set of events internal to the machine $\{\artifact{start-signal}, \artifact{stop-signal}, \artifact{open-signal}, \artifact{alarm-signal}, \ldots\}$. 

Given a LTS $P = (Q,\Sigma,\delta,q_{init})$ and a set of labels $I$, the relation between the languages of $P$ and $P \setminus I$ is defined as follows:

\begin{center}
$\mathcal{L}(P \setminus I) = \{ t'~|~\exists t \in \mathcal{L}(P)~such~that~t' = t|_{\Sigma \setminus I}\}$
\end{center}

