\section{Framework overview\label{section:background-multi-agent-systems-and-behavior-modeling}}

As stated in the introduction, we are concerned in this thesis with the modeling of software systems. Making so call for rich models that cover the structural, intentional and behavioral dimensions of the system~\cite{VanLamsweerde:2000}. The present framework approaches these co-related dimensions, with an emphasis on the behavioral one. Let briefly discuss each of them in turn.

\noindent \textbf{Structural} -- A system is commonly seen as being made of active components, called \emph{agents}, that behave and interact so as to fulfill system goals while restricting their behavior to ensure constraints they are assigned to~\cite{Feather:1987}. Some of them are human agents (the passenger), others are physical or electronic devices (e.g. the doors, the actuators), still others are software components (the automated controller).

In addition to the notion of \emph{system}, that encompass all agents, the literature makes use of specific terms to distinguish between certain agents and/or agent aggregations. In~\cite{VanLamsweerde:2009} for example, the \emph{software-to-be} denotes software agent(s) that need to be developed (the automated controller, for example), while other agents compose its \emph{environment}. Another boundary consists in distinguishing the software together with its input and output devices from other agents. This boundary, depicted with a dashed line in Fig.~\ref{image:train-scenario-all-agents}, corresponds to the distinction made by Jackson between the \emph{world} and the \emph{machine}~\cite{Jackson:1995}.

In this thesis, we do not cover specific artifacts for modeling and playing with agent boundaries and interfaces. Nevertheless, abstracting completely from the structural dimension is not possible. Indeed, we are concerned with modeling not only agent- but also \emph{system} behaviors, as the composition of the former. However, we can stick here with intuitive notions for the structural dimension, as already illustrated in Fig.~\ref{image:train-scenario-all-agents}. That is, surrounding a set of agents allows drawing a separation between them and the rest of the world. This can be seen as aggregating those agents as a new one of more coarse granularity. Therefore, messages can be partitioned between internal and external (including crossing) messages to the surrounding box. The latter can be treated as a white or a black-box according to whether you show or hide internal messages. Composition and hiding operators on state machines, that we present later, support these structural mechanisms on the behavioral side. For a more precise description of structural models that nicely fit the present framework the reader can refer to~\cite{Magee:1995}.

\noindent \textbf{Behavioral} -- The behavioral dimension is the one on which we put the strongest emphasis. Behaviors here capture interactions among the agents forming a system, and are modeled as \emph{events} synchronously sent and received by them. Typical examples and counterexamples of system behaviors are illustrated with positive and negative scenarios using Message Sequence Charts~\cite{ITU:1996}, like the one in Fig.~\ref{image:train-scenario-all-agents}.

In addition to partial and multi-agent behavior illustrations offered by scenarios, the complete behavior of each agent is modeled with a kind of state machine called labeled transition systems (LTS)~\cite{Keller:1976, Milner:1989}. Agent LTSs can be composed for obtaining the behavior of a complete system~\cite{Hoare:1985}, or projected the other way around, therefore supporting the zoom-in/zoom-out structural mechanism aforementioned.

We restrict our attention here to \emph{determinate} agents~\cite{Engelfriet:1985}, that is, agents whose observable behavior can be captured with the sole use of \emph{deterministic} transition systems (see Section~\ref{section:background-state-machines}). Such an assumption leads to a simple and intuitive framework, an important aspect for accessibility to stakeholders involved in an early-design phase of software system. In particular, this allows formalizing behaviors with standard \emph{trace theory}~\cite{Hoare:1985} and sticking to the simplest notion of behavior equivalence, namely \emph{trace equivalence}~\cite{Engelfriet:1985}. Last, under such hypotheses, agent and system behaviors are captured by the class of \emph{prefix-closed} regular languages, a subclass of the well-studied \emph{regular} languages~\cite{Hopcroft:1979, Aho:1986}. In addition to enabling reuse of standard results from automaton theory, this paves the way to using grammar inference~\cite{Gold:1978} for behavior model synthesis (see chapter~\ref{chapter:inductive-synthesis}). 
