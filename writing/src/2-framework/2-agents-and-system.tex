\section{Multi-Agent Systems and Behavior Modeling\label{section:background-agents-system-behaviors}}

A system is commonly admitted to be made of active components, called \emph{agents}, that behave and interact so as to fulfill system goals while restricting their behavior to ensure constraints they are assigned to~\cite{Feather:1987}. Some of them are human agents (the passenger), others are physical or electronic devices (e.g. the doors, the actuators), still others are software components (the automated controller). In addition to the notion of \emph{system}, that encompass all agents, the literature makes use of specific terms to distinguish between certain agents and/or agent aggregations. In~\cite{VanLamsweerde:2009} for example, the \emph{software-to-be} denotes software agent(s) that need to be developed (the automated controller, for example), while other agents compose its \emph{environment}. Another boundary consists in distinguishing the software together with its input and output devices from other agents. This boundary, depicted with a dashed line in Fig.~\ref{image:train-scenario-all-agents}, corresponds to the distinction made by Jackson between the \emph{world} and the \emph{machine}~\cite{Jackson:1995}. In this thesis, we focus on the behavior of a single agent as observed by the other agents with which it interacts (as opposed to its internal implementation). From the former agent perspective, then, the \emph{environment} is made of all these latter agents. 

In the light of the previous paragraph, we clearly need tools to capture single agent behaviors while being able to play with agent boundaries in a flexible manner -- for instance, for ``computing'' the behavior of agent aggregations like the \emph{software environment}, the \emph{machine} or simply, the \emph{system}. For this, we choose to model behaviors and interactions in an event-based framework, where agents communicate via messages that are sent and received simultaneously. Such kind of communication, called \emph{synchronous communication} or simply, \emph{message passing}, is motivated by its simplicity, an important aspect for accessibility to stakeholders involved during the early-design phase of a system. The next section introduces  labeled transition systems (LTS), the kind of model we use to capture agent behaviors. The next one presents operators for composing (and decomposing) them under a \emph{synchronous communication} hypothesis, a manner of capturing the behaviors of multiple interacting agents.

