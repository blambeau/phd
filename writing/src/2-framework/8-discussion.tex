\section{Model synthesis opportunities\label{section:background-discussion}}

The next two chapters present synthesis techniques that can be applied in our multi-model framework. 

\begin{itemize}
\item Chapter~\ref{chapter:deductive} defines a trace semantics for process models specified as g-hMSCs. In order to model-check such models, it is convenient to capture process traces through LTS; the chapter presents a synthesis technique for this. The implementation of the model-checker is discussed in Chapter~\ref{chapter:tool-support}.
\item Chapter~\ref{chapter:inductive-synthesis} presents an inductive technique for synthesizing LTS state machines from scenario collections and hMSCs. This technique relies on grammar induction~\cite{Gold:1978}, which can be used to learn LTS from traces. The generalization process is guided by the end-user who is requested to classify additional scenarios as positive or negative system behaviors. The information offered by the other models (fluents, goals, etc.) can also be used to guarantee inter-model consistency. 
\end{itemize}

Other synthesis techniques are briefly discussed in Chapter~\ref{chapter:discussion}. Some of them can be used inside our framework whereas others are defined on other formalisms. The same chapter identifies further techniques that might be investigated.
