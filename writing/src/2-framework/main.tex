\chapter{A multi-view modeling framework\label{chapter:framework}}

This chapter installs the necessary background about the formalisms we use and sets up our multi-view modeling framework. We've decided to present the technical background \emph{in situ}. Indeed, strictly speaking, the multi-view framework is a contribution of this thesis, but almost all its technical pieces, that is, formalisms, algorithms and techniques, are not. 

The chapter is organized as follows. The running example used throughout this thesis is quickly presented in the next section. Section~\ref{section:background-multi-agent-systems-and-behavior-modeling} then provides a general overview of the modeling framework and how multiple models fit together. These models and their semantics is then detailed in subsequent sections. State machines are presented in Section~\ref{section:background-state-machines}. Scenarios and related constructions in Section~\ref{section:background-scenarios}. We then turn to state based abstractions by describing fluents and their use for capturing state invariants in Section~\ref{section:background-fluents}, and goals and domain properties in Section~\ref{section:background-goals}. We close this chapter with Section~\ref{section:background-discussion} that provides a general discussion about model synthesis in the framework.

\section{Running example: a simple train controller system\label{section:background-running-example}}

We use a simple train system fragment as running example for illustrating concepts and techniques throughout this thesis. The system is composed of an automated train controller, actuators for doors and the engine as well as the latter themselves, sensors, and a passenger. Via the actuators, the controller typically controls operations like starting or stopping the train, opening or closing the doors, and so on. A safety goal requires train doors to remain closed while the train is moving. If the train is not moving and the passenger presses the alarm button, the controller must open the doors immediately. If the train is moving and the passenger presses the alarm button, then the controller must stop the train first and then open the doors. Typical agent interactions for the latter case are depicted in Fig.~\ref{image:train-scenario-all-agents}. The precise semantics of such a scenario is made clear in the following sections.

\begin{figure}\centering
\scalebox{0.75}{
  \includegraphics[trim=2mm 2mm 2mm 2mm, clip]{src/2-framework/images/train-scenario-all-agents}
}
\caption{A scenario illustrating a train system stopping in emergency when an alarm is pressed.\label{image:train-scenario-all-agents}}
\end{figure}


\section{Framework overview\label{section:background-multi-agent-systems-and-behavior-modeling}}

As stated in the introduction, we are concerned in this thesis with the modeling of software systems. Making so call for rich models that cover the structural, intentional and behavioral dimensions of the system~\cite{VanLamsweerde:2000}. The present framework approaches these co-related dimensions, with an emphasis on the behavioral one. Let briefly discuss each of them in turn.

\noindent \textbf{Structural} -- A system is commonly seen as being made of active components, called \emph{agents}, that behave and interact so as to fulfill system goals while restricting their behavior to ensure constraints they are assigned to~\cite{Feather:1987}. Some of them are human agents (the passenger), others are physical or electronic devices (e.g. the doors, the actuators), still others are software components (the automated controller).

In addition to the notion of \emph{system}, that encompass all agents, the literature makes use of specific terms to distinguish between certain agents and/or agent aggregations. In~\cite{VanLamsweerde:2009} for example, the \emph{software-to-be} denotes software agent(s) that need to be developed (the automated controller, for example), while other agents compose its \emph{environment}. Another boundary consists in distinguishing the software together with its input and output devices from other agents. This boundary, depicted with a dashed line in Fig.~\ref{image:train-scenario-all-agents}, corresponds to the distinction made by Jackson between the \emph{world} and the \emph{machine}~\cite{Jackson:1995}.

In this thesis, we stick with very basic notions of structural modeling. We consider the system as made of agents, whose interface consist in the set of messages, call \emph{events} here, that it can send or receive (see the behavioral dimension below). By simplicity we assume that an event label uniquely determines the interacting agents. We do not cover specific artifacts for modeling and playing with agent interfaces and boundaries, like context diagrams~\cite{Feather:1987}, and stay with the informal understanding already illustrated in Fig.~\ref{image:train-scenario-all-agents}. That is, surrounding a set of agents allows drawing a separation between them and the rest of the world. This can be seen as aggregating those agents as a new one of coarser granularity. Therefore, events can be partitioned between internal, crossing and external events. The crossing events form the new agent's interface. The surrounding box can be seen as a white or a black-box, according to whether you show or hide internal messages. Composition and hiding operators on state machines, that we present later, support these structural mechanisms on the behavioral side. For a more precise description of structural models that nicely fit the present framework the reader can refer to~\cite{Magee:1995}.

\noindent \textbf{Behavioral} -- The behavioral dimension is the one on which we put the strongest emphasis. Behaviors here capture interactions among the agents forming a system, and are modeled as sequences of events. We consider that events are synchronously sent and received by agents. Also, we allow the same event to be received by several agents at once, that is we support a form of \emph{broadcasting}. Typical examples and counterexamples of system behaviors are illustrated with positive and negative scenarios using Message Sequence Charts~\cite{ITU:1996}, like the one in Fig.~\ref{image:train-scenario-all-agents}. Higher-level scenarios allow introducing sequences and loops in such system descriptions.

In addition to the multi-agent, yet partial, behavior description offered by scenarios, the complete behavior of each agent is modeled with a kind of state machine called labeled transition systems (LTS)~\cite{Keller:1976, Milner:1989}. The behavior of the system itself is obtained through parallel composition~\cite{Hoare:1985} of agent LTSs. Behavior  projection on specific agents is also supported, therefore providing the zoom-in/zoom-out structural mechanism aforementioned.

Nevertheless, we restrict our attention here to \emph{determinate} agents~\cite{Engelfriet:1985}, that is, agents whose observable behavior can be captured with the sole use of \emph{deterministic} transition systems (see Section~\ref{section:background-state-machines}). Such an assumption leads to a simple and intuitive framework, an important aspect for accessibility to stakeholders involved in an early-design phase of software system. In particular, this allows formalizing behaviors with standard \emph{trace theory}~\cite{Hoare:1985} and sticking to the simplest notion of behavior equivalence, namely \emph{trace equivalence}~\cite{Engelfriet:1985}. Last, under such hypotheses, agent and system behaviors are captured by the class of \emph{prefix-closed} regular languages, a subclass of the well-studied \emph{regular} languages~\cite{Hopcroft:1979, Aho:1986}. In addition to enabling reuse of standard results from automaton theory, this paves the way to using grammar inference~\cite{Gold:1978} for behavior model synthesis (see chapter~\ref{chapter:inductive-synthesis}). 

\begin{figure}\centering
  \scalebox{0.50}{\includegraphics[trim=2mm 2mm 3mm 2mm, clip]{src/2-framework/images/framework}}
  \caption{Formal framework overview.\label{image:framework}}
\end{figure}

\noindent \textbf{Intentional} -- The intentional dimension aims at capturing \emph{why} is the system needed, through goal models. A goal corresponds to an objective the system should achieve through cooperation of agents in the software-to-be and in the environment. Unlike goals, \emph{domain properties} are descriptive statement about the environment -- such as physical laws, organizational rules, etc. Goal models are AND/OR graphs that capture how functional and non-functional goals contribute positively or negatively to each other~\cite{VanLamsweerde:2000, VanLamsweerde:2004}.

In this thesis, we restrict our attention to \emph{behavioral} goals, which unlike \emph{soft} goals can be established in a clear-sense (see~\cite{VanLamsweerde:2009} for a taxonomy of goals). Moreover, regarding the connection with model synthesis, we only consider goals that can be formalized as \emph{safety} properties in linear temporal logic (LTL)~\cite{Manna:1992}. A safety property stipulates that some ``bad thing'' does not happen~\cite{Alpern:1986}. If such a ``bad thing'' happens in an infinite sequence, then it must also do so after some finite prefix and must be irremediable. The class of system behaviors respecting safety properties can then be captured with labeled transition system, and fit the present behavior framework without much change.

However, goals are best captured with state-based abstractions (the train is \emph{moving} then the doors are \emph{closed}) while the framework so far is event-based (the train controller starts and stops the engine). We use \emph{fluents}~\cite{Miller:2002} as an effective way for reconciling these paradigms; fluents indeed capture state-based propositions in terms of the occurrence of events. Then, we formalize goals in Fluent Linear Temporal Logic (FLTL)~\cite{Giannakopoulou:2003}, a flavor of linear temporal logic where atomic propositions are fluents. We do not consider neither the structuration of goals in goal graphs nor their assignment to agents in the present thesis.





\section{State Machines as Labeled Transition Systems\label{section:background-state-machines}}

In our framework, the behavior of an agent (and the system) is modeled by a specific kind of finite state machine, called \emph{labeled transition system} (LTS). This formalism, initially introduced by Keller for reasoning about parallel programs~\cite{Keller:1976}, has since been intensively used for specifying and analyzing concurrent systems, e.g. in~\cite{Milner:1989, Clarke:1989, Magee:1997}. A LTS is made of a set of states and a set of transitions between them (see Fig.~\ref{image:framework-start-stop}). Each transition is depicted with an \emph{event} label -- sometimes called a \emph{symbol} or an \emph{action} label; also, a specific state is the \emph{initial state}, designated graphically by an empty arrow in front of it (state 0 in the figure). 

\vspace{0.5cm}
\begin{figure}[H]
\centering\scalebox{0.60}{
  \includegraphics*[clip]{src/2-framework/images/start-stop}}
  \caption{A Labeled Transition System for an \artifact{Engine} agent\label{image:framework-start-stop}.}
\end{figure}

Mathematically, a LTS is defined as a 4-tuple $(Q,\Sigma,\delta,q_{init})$ where $Q$ is a finite set of states, $\Sigma$ is a set of labels called its \emph{alphabet}, $\delta$ is a transition relation $Q \times \Sigma\cup\{\tau\} \times Q$ and $q_{init} \in Q$ is the initial state. The label $\tau$ is used to denote the so-called \emph{non-observable} transitions, which allow modeling an agent changing state without such change being monitored by its environment. 

A \emph{deterministic} LTS does not have $\tau$ transitions and has no state with two outgoing transitions having the same label (that is, $(q,l,q_1) \in \delta \wedge (q,l,q_2) \in \delta \implies q_1 = q_2$); otherwise it is \emph{non-deterministic}.

A \emph{terminating} state is one with no outgoing transition; otherwise it is \emph{non-terminating}. A \emph{terminating} LTS has at least one terminating state; otherwise it is \emph{non-terminating}. Note that, as such, the LTS definition does not allow distinguishing between terminating states that model successful termination -- an agent stops running intentionally -- and non-successful termination -- an agent, more often the system as a \emph{composed} agent (see next section), \emph{deadlocks} unintentionally.

The \emph{alphabet} $\Sigma$ captures the notion of \emph{agent interface}~\cite{Feather:1987}, as a set of event labels that an agent recognizes or, said otherwise, in which the agent \emph{engages} in synchronous communications with its environment. For example, the LTS of Fig.~\ref{image:framework-start-stop} has an alphabet \artifact{$\Sigma=\{start, stop\}$}. Note that labeled transition systems do not distinguish between \emph{sent} and \emph{received} events. This distinction being required when playing with scenarios, we assume this structural information is available elsewhere, typically from a context or architecture diagram~\cite{Ward:1985, Magee:1995}. As already stated, we also assume that an event label uniquely determines the interacting agents but allow an event to be received by more than one agents. By simplicity in the sequel, we denote $\Sigma\cup\{\tau\}$ by $\Sigma_{\tau}$ (an alphabet augmented with the $\tau$ label).

An finite LTS \emph{execution} is a finite sequence of its states separated by labels, i.e. \artifact{$w = \textless q_0,l_0,\ldots,q_{n-1},l_{n-1},q_n \textgreater$} with $q_i \in Q$ and $l_i \in \Sigma_{\tau}$. An execution is valid for a LTS if it denotes an existing path, from the initial state, in the corresponding graph; mathematically, $q_0 = q_{init}$ and $(q_i,l_i,q_{i+1}) \in \delta$ for $0 \leq i < n$. The projection of an execution $w$ over an alphabet $\Sigma$ is denoted by $w|_{\Sigma}$ and is the result of keeping, from $w$, only event labels that belong to $\Sigma$ (in other words, eliminating $q_i$ states and $\tau$ labels). Such a projection is also called a \emph{trace}, that we define now.

A \emph{trace} denotes an element of $\Sigma^*$, that is a finite sequence of event labels \artifact{$t= \textless l_0,\ldots,l_{n} \textgreater$} with $l_i \in \Sigma$. Unlike an execution, a trace never contains $\tau$ labels. A trace $t$ is accepted by a LTS if there exists a valid execution $w$ such that $w|_{\Sigma} = t$. In other words, a trace is \emph{accepted} by a LTS if it denotes an existing path in the corresponding graph from the initial path, but allowing ``in the middle'' silent moves offered by $\tau$ transitions in the non-deterministic case (and hence, possibly, more than one path). Note that, by this definition, a prefix of an accepted trace is also an accepted trace; the empty trace $\lambda$ is therefore always accepted. For example, the LTS of Fig.~\ref{image:framework-start-stop} accepts the trace \artifact{<start stop start>}, and hence \artifact{<start stop>}, but not \artifact{<start start>}. We sometimes use a dot notation $w.l$ to denote the concatenation of a trace $w$ with a label or another trace $l$.

An important notion when designing and analyzing concurrent systems is the one of \emph{behavior equivalence} that permits answering questions like ``\emph{are agents $Ag_1$ and $Ag_2$ the same in terms of behavior?}''. Many different notions of behavioral equivalence exist in the literature, like \emph{strong} and \emph{observational}  equivalences~\cite{Milner:1989}, bisimilarity~\cite{Park:1981}, or failure equivalence~\cite{Hoare:1985}. Their introduction (mostly in process algebra) is motivated by the need to distinguish between special process cases as well as being able to reason about their correctness (in terms of \emph{deadlock}, for instance), especially when dealing with non-determinism. For details, see e.g.~\cite[chap. 3]{Hoare:1985}, \cite[chap. 4 \& 5]{Milner:1989} or the overview given in~\cite{Fernandez:1991}. Our hypotheses, especially the one of \emph{determinate} agents, allows us to stick with the weakest, yet simplest, notion of LTS equivalence: \emph{trace equivalence}~\cite{Hoare:1985, Engelfriet:1985}. Under the latter, two LTS $P$ and $Q$ are equivalent, denoted by $P \equiv_{tr} Q$, if they accept the same set of traces, in other words, if they define the same \emph{language} $\mathcal{L}(P) = \mathcal{L}(Q)$. This naturally leads us to the next section, which discusses some important properties inherited by LTSs from their connection with regular languages. 

\subsection{Labeled Transition Systems and Regular Languages\label{section:background-lts-and-regular-languages}}

Labeled transition systems are actually a subclass of standard automata \cite{Hopcroft:1979}. The only difference is that the latter make an additional distinction between \emph{accepting} and \emph{non-accepting} states. In other words, a LTS can be seen as a standard automaton in which all states are accepting states. Many results from standard automata theory apply to LTS while preserving trace equivalence, that is, observable behaviors.

First, the (maybe infinite) set of traces accepted by a LTS, say $P$, is called its \emph{language} and denoted by $\mathcal{L}(P)$. For example, the  \emph{language} of the LTS shown in Fig.~\ref{image:framework-start-stop} is $\mathcal{L}(\artifact{Engine})=\{\lambda$, \artifact{<start>}, \artifact{<start stop>}, \artifact{<start stop start>}, \ldots $\}$. Standard automata, both deterministic and non-deterministic ones, capture the well-studied class of \emph{regular} languages~\cite{Hopcroft:1979} while LTS, because they have all accepting states, capture the class of \emph{prefix-closed} regular languages. The term ``prefix-closed'' means that all prefixes of accepted traces are also accepted traces, mathematically $prefixes(\mathcal{L}(P)) = \mathcal{L(P)}$.

Another notable result is the existence, for any regular language $\mathcal{L}$, of a canonical automaton $A(\mathcal{L})$. This canonical automaton is the minimal deterministic automaton accepting $\mathcal{L}$, and is known to be unique up to state renumbering~\cite{Gold:1978}. Without loss of generality, we can therefore assume that the behavior of any agent is modeled by a canonical LTS. Moreover, these results enables the use of non-deterministic constructions -- $\tau$ transitions in particular -- without being in contradiction with our assumption of \emph{determinate} agents. Given any LTS $P$, being deterministic or not, we denote by $P^{\Delta}$ its canonical equivalent, where $P$ can actually be a LTS \emph{expression} (that is a LTS ``computed'' by applying LTS operators introduced in next sections). Standard automaton algorithms from~\cite{Hopcroft:1979} can be used to compute $P^\Delta$, that is, to remove $\tau$ transitions, determinize and/or minimize LTSs under our assumptions.

As languages are sets of traces, it is sometimes natural to reason in terms of standard operators on sets. In the following sections, we often make use of notations for the union of two languages ($\cup$), their intersection ($\cap$), subset ($\subseteq$), proper subset ($\subset$) and equality ($=$). This is especially true in the definition of inter-model consistency rules. Techniques and algorithms for implementing these operators for the general case of standard automata can be found in~\cite{Hopcroft:1979, Aho:1986}, providing an operational way of testing such rules. Also, given a language $\mathcal{L}$, we sometimes denote by $mt(\mathcal{L})$ the set of \emph{maximal} traces of $\mathcal{L}$, that is, traces that cannot be extended by a suffix within the same language. Given a deterministic LTS, they simply correspond to traces reaching a terminating state. 

In the next two sections, we define two additional operators, \emph{composition} and \emph{hiding}, that allow reasoning about behaviors in presence of multiple agents with different alphabets.

\subsection{System as Agent composition}

If a system is composed of active agents and the behavior of each of these agents is explicitly modeled with an LTS, one can ask what is the behavior of the system itself. We define it through parallel composition~\cite{Hoare:1985}, a setting where agents execute asynchronously but synchronize on shared events. Given a system made of $n$ agents, and the composition operator denoted by~$\parallel$, the system is defined as:

\begin{equation}
System = Ag_1 \parallel \ldots \parallel Ag_n
\end{equation}

As we are mostly interested in agent \emph{behaviors}, we use the binary composition operator $\parallel$ defined on LTS, see e.g.~\cite{Giannakopoulou:1999, Magee:1999}. The operator, which is both commutative and associative (allowing our writing above without ambiguity), computes the interleaving of the traces accepted by the two LTS, under the constraint that they synchronize on shared labels. Let $P = (S_1,\Sigma_1,\delta_1,q_{1})$ and $Q = (S_2,\Sigma_2,\delta_2,q_{2})$ denote two LTS. Then, their composition $P \parallel Q$ is another LTS $(S_1 \times S_2,\Sigma_1\cup\Sigma_2,\delta,(q_1,q_2))$, where $\delta$ is the smallest relation satisfying the following rules:

\begin{center}
\begin{tabular}{cc}
$\frac{\displaystyle P \stackrel{l}{\longrightarrow} P'}{\displaystyle P \parallel Q \stackrel{l}{\longrightarrow} P' \parallel Q}~~l \notin \Sigma_2$ &
$\frac{\displaystyle Q \stackrel{l}{\longrightarrow} Q'}{\displaystyle P \parallel Q \stackrel{l}{\longrightarrow} P \parallel Q'}~~l \notin \Sigma_1$ \\
 & \\
\multicolumn{2}{c}{$\frac{\displaystyle P \stackrel{l}{\longrightarrow} P',~Q \stackrel{l}{\longrightarrow} Q'}{\displaystyle P \parallel Q \stackrel{l}{\longrightarrow} P' \parallel Q'}~~l \neq \tau$} \\
\end{tabular}
\end{center}

In these rules, the notation $X \stackrel{l}{\longrightarrow} X'$ simply denotes the fact that the LTS $X = (S,\Sigma,\delta,q_0)$ may \emph{transit} into another LTS $X' = (S,\Sigma,\delta,q_1)$ with the event label $l$, provided that $(q_0,l,q_1) \in \delta$. 

As one can see, $P \parallel Q$ is defined on the Cartesian product of the states of $P$ and $Q$, and has its initial state simply defined as $(q_1,q_2)$ in this state space. Rules above define possible transitions from such a state. The first two rules are symmetric and encode the fact that, on non shared labels, one LTS may transit while the other stays in its previous state. As stated, those rules allow individual LTS to move along $\tau$ transitions. The last rule forces the two LTS to transit together on all shared labels but $\tau$. A composed LTS can easily be computed constructively by exploring the state space from its initial state until no new state pair is discovered. 

The trace semantics of a system composed of $n$ agents whose behavior is modeled with LTSs $Ag_1$ to $Ag_n$ is captured as:

\begin{equation}
\mathcal{L}(System) = \mathcal{L}(Ag_1 \parallel \ldots \parallel Ag_n)
\label{equation:system-composition}
\end{equation}

\subsection{Black-box behavior through \emph{hiding}}

If the notion of agent composition gives a sound interpretation to the notion of \emph{system} and its behavior, it is, in fact, of slightly more general use. Indeed, as explained in the introduction of this section, it makes sense to consider not only the composition of all agents but sometimes the composition of a subset of them that, together, define an interesting boundary in the system considered. Consider the agents depicted in the scenario of Fig.~\ref{image:train-scenario-all-agents} for example. The ``machine vs. world'' boundary can simply be modeled as follows:

\vspace{-0.8cm}
\begin{align*}
Machine &= Controller \parallel Actuators \parallel Sensors \\
World   &= Passenger \parallel Doors \parallel Engine \\
System  &= Machine \parallel World
\end{align*}
\vspace{-0.8cm}

However, as defined above, the $Machine$ agent has an interface -- in terms of the set of events in which it engages -- which is actually too large. One would like to have a black-box version of the $Machine$ behavior, that is a LTS with events of the new interface only, those crossing the depicted boundary. For this, LTSs come with a simple operator called \emph{hiding}. Hiding of a set of labels $I$ in a LTS $P = (Q,\Sigma,\delta,q_{init})$ is denoted $P \setminus I$ and defines the LTS $(Q,\Sigma \setminus I,\delta_{hidden},q_{init})$ where $\delta_{hidden}$ is the smallest relation satisfying the rules:

\begin{center}
\begin{tabular}{cc}
$\frac{\displaystyle P \stackrel{l}{\longrightarrow} P'}{\displaystyle P \setminus I \stackrel{l}{\longrightarrow} P' \setminus I}~~l \notin I$ & 
$\frac{\displaystyle P \stackrel{l}{\longrightarrow} P'}{\displaystyle P \setminus I \stackrel{\tau}{\longrightarrow} P' \setminus I}~~l \in I$ \\
\end{tabular}
\end{center}

As one can see, the operator simply makes a set of labels invisible from the environment by replacing them by $\tau$. The resulting LTS is non-deterministic, but results from previous sections ensure that a minimal and deterministic equivalent exists. That is, the LTS of the black-box machine we are actually looking for is the following:

\vspace{-0.8cm}
\begin{align*}
Machine' &= (Machine \setminus Internals)^\Delta
\end{align*}
\vspace{-0.8cm}

\noindent where $Machine$ is the composition between the controller, actuators and sensors given previously and $Internals$ is the set of events internal to the machine $\{\artifact{start-signal}, \artifact{stop-signal}, \artifact{open-signal}, \artifact{alarm-signal}, \ldots\}$. 

Given a LTS $P = (Q,\Sigma,\delta,q_{init})$ and a set of labels $I$, the relation between the languages of $P$ and $P \setminus I$ is defined as follows:

\begin{center}
$\mathcal{L}(P \setminus I) = \{ t'~|~\exists t \in \mathcal{L}(P)~such~that~t' = t|_{\Sigma \setminus I}\}$
\end{center}


\section{Scenarios as Message Sequence Charts\label{section:background-scenarios}}

Unlike labeled transition systems, that intrinsically model the behavior of a single agent (even if a composed one), scenarios explicitly illustrate interactions among multiple agents. The scenarios we use in this thesis are a syntactic subset of Message Sequence Charts (MSC)~\cite{ITU:1996} (see Fig.~\ref{image:train-scenario-all-agents} for example). To keep the models usable by end-users, however, we use only a small subset of their features. In its simplest form, a MSC is composed of vertical lines representing timelines associated with agents and horizontal arrows representing interactions among agents, also called \emph{events}. Following the modeling of agents of the previous section, events are synchronously sent and received by interacting agents (we also use the terms \emph{controlled} and \emph{monitored} events, respectively). As already stated in previous section, we assume that an event label uniquely determines the latter agents. 

We consider \emph{positive} scenarios, as examples of behaviors that the system should exhibit, in the next section and \emph{negative} scenarios, behaviors that the system must avoid, in the following one. Subsequent sections will then discuss ways of managing multiple positive and negative scenarios. 

\subsection{Positive scenarios\label{subsection:background-positive-scenarios}}

MSCs in this thesis are given a trace semantics, following~\cite{Uchitel:2004}. That is, we consider that MSCs define sets of traces, and capture the later with LTSs, as for agents in the previous section. Given a MSC, two kinds of traces can actually be considered: those from the local perspective of a single timeline, and those from the global perspective of the complete MSC. We discuss each of these views in turn.

As time in a MSC is represented top-down, the order in which events are sent and received along a particular timeline defines a total order. Therefore, from the perspective of a single agent, a MSC defines only one trace; precisely, one \emph{maximal} trace, that is with all events to which the agent participates. That trace, and all its suffixes then, can be captured by a LTS. For example, the traces defined by the timeline of the \artifact{Controller} in the MSC of Fig.~\ref{image:train-scenario-all-agents} are precisely captured by the LTS of Fig.~\ref{image:local-traces-lts}. Given a MSC $M$ and an agent $Ag$, we denote such an LTS by $M_{\downarrow Ag}$.

\vspace{0.5cm}
\begin{figure}[H]\centering
\scalebox{0.45}{\includegraphics{src/2-framework/images/local-trace}}
\caption{LTS capturing the traces of the MSC of Fig.~\ref{image:train-scenario-all-agents} from the local point of view of the \artifact{Controller}.\label{image:local-traces-lts}}
\end{figure}

When considering traces from the perspective of the whole MSC, two interpretations co-exist. In the former, a reasoning similar to the previous one is applied. Indeed, one can consider that a MSC defines a \emph{maximal} trace, with all events in a (graphical) top-down ordering. In the example at hand, such a trace would be \artifact{<start-signal, start, alarm-pressed, \ldots, open>}. Making so amounts to consider that a MSC defines a total order among all events; this leads to a simple and straightforward, yet limited, trace semantics of MSCs.

However, when considering concurrent systems, a partial ordering among events seems more adequate~\cite{ITU:1996, Uchitel:2003}. Consider for example the events \artifact{start-signal} and \artifact{alarm-pressed} at beginning of the same MSC shown in Fig.~\ref{image:train-scenario-all-agents}. These two events model unrelated message passing between different agents, and can therefore hardly be considered timely ordered (e.g. maybe does the passenger push the alarm button when the \artifact{start-signal} is already sent but before \artifact{start} has been propagated? or even before \artifact{start-signal}?, and so on.). To take such cases into account, one has to consider that the traces defined by a MSC are \emph{linearizations} of the partial order among events. In other words, linearizations capture all possible sequences of events that respect the total ordering defined by timelines. We do not formalize the structure of Message Sequence Charts and their linearizations here, and refer the reader to~\cite{Uchitel:2003} for such a mathematical characterization. However, we state the relations that hold between the set of traces obtained with the local and global perspectives discussed so far:

\vspace{-0.4cm}
\begin{equation}
\label{equation:msc-composition}
\mathcal{L}_{total}(M) \subseteq \mathcal{L}_{partial}(M) = \mathcal{L}(M_{\downarrow Ag_1} \parallel \ldots \parallel M_{\downarrow Ag_n})
\end{equation}

The left part simply states that traces under a partial ordering certainly include those under a total ordering, which is expected. Indeed, the model with partial ordering is more general than the one with total ordering. Among others, it means that everything that is true for the former is certainly true for the latter as well. From now on, and unless stated otherwise, we therefore assume the general model with partial ordering. In particular, we do not make the $partial$ and $total$ subscripts explicit when stating other language relations in the following sections. A consistent use of one of them renders all formula valid.

The right part provides a simple way of computing all linearizations of a MSC as an (acyclic) transition system, through LTS composition. Such a LTS is illustrated in Fig.~\ref{image:msc-linearizations} for the MSC of Fig.\ref{image:train-scenario-all-agents}. As shown, the latter accepts six different linearizations, due to the possible interleaving of its first four events. By construction, such a LTS has only one initial state (the leftmost one) and only one terminating state (the rightmost one). This allows us to loosely refer to \emph{the} terminating state of an MSC (more precisely, of the LTS capturing its traces) with a precise underlying meaning.

\aside{The reader wondering whether certain linearizations are not \emph{implied} scenarios~\cite{Uchitel:2004} or whether our initial scenario should not be regarded flawed is referred to Section~\ref{section:background-discussion} where those issues are further examined.}

\begin{figure}\centering
\scalebox{0.31}{\includegraphics{src/2-framework/images/linearizations}}
\caption{LTS capturing all event linearizations of the MSC of Fig.~\ref{image:train-scenario-all-agents}. Here, \artifact{alarm-pressed} is abbreviated as \artifact{a-pressed} and \artifact{sig.} stands for \artifact{signal}. \label{image:msc-linearizations}}
\end{figure}

\subsubsection*{Multi-view model consistency}

The trace semantics of a single MSC can now be explicitly related to the notions of agent and system behaviors, also captured with LTSs in Section~\ref{section:background-state-machines}. This amounts to define consistency rules between them, and explains the similitude between definitions \ref{equation:system-composition} and \ref{equation:msc-composition}.

Consider a system composed of $n$ agents whose behavior is modeled as $\system$. Let $M$ denote a MSC illustrating interactions between them. We say that $M$ is \emph{consistent} with $S$ -- or, more accurately, that $M$ and $S$ \emph{are} consistent -- if the following conditions hold:

\begin{itemize}
\item $M$ and $S$ are \emph{architecturally} consistent,
\item $\mathcal{L}(M_{\downarrow Ag_i}) \subseteq \mathcal{L}(Ag_i)$ for each agent $Ag_i$, and
\item $\mathcal{L}(M) = \mathcal{L}(M_{\downarrow Ag_1} \parallel \ldots \parallel M_{\downarrow Ag_n}) \subseteq \mathcal{L}(Ag_1 \parallel \ldots \parallel Ag_n) = \mathcal{L}(S)$
\end{itemize}

The first condition is not formalized but simply requires the MSC and the system to agree on the set of agents (a MSC may actually illustrate interactions among a proper subset of system agents) and their respective interfaces (labels along a timeline are a subset of the alphabet of the corresponding agent). The second condition states that the traces defined by a timeline in the MSC must be traces accepted in the LTS modeling the behavior of the corresponding agent. The third condition states that all linearizations of the MSC must be accepted traces of the LTS modeling the behavior of the composed system. Note that, under architectural consistency, the second condition implies the third one~\cite{Uchitel:2003}. Last, but not least, stated conditions restrict consistent MSCs to those starting in the system initial state.

\subsection{Negative scenarios}

While positive MSCs model examples of behavior that the system is expected to exhibit, it is often convenient to be able to model the counterpart, that is examples of behavior that the system is expected (even required) \emph{not} to exhibit. Such proscribed behaviors are illustrated with negative MSCs~\cite{Uchitel:2004}. A negative MSC is simply a scenario whose last event is proscribed, as depicted by a crossed arrow below a dashed line. An example is given in Fig.~\ref{image:train-negative-scenario}, where the \artifact{Controller} may not open doors immediately after having started the train.

\begin{figure}\centering
\scalebox{0.75}{
  \includegraphics[trim=2mm 2mm 2mm 2mm, clip]{src/2-framework/images/train-negative-scenario}
}
\caption{A negative scenario illustrating that the controller may not open doors after having started.\label{image:train-negative-scenario}}
\end{figure}

More precisely, a negative MSC is a pair $(P,e)$ where $P$ is a positive MSC and $e$ is a single event (given our simplifying assumption, the latter event can simply be denoted by its label $l = label(e)$). The positive scenario $P$ is called the $precondition$ and $e$ the \emph{proscribed event}. The intuitive semantics is that, once the precondition has occurred from the system initial state, $e$ may not be the (very) \emph{next} event in the system. We make this semantics fully precise now. First, if the precondition of a negative MSC $N = (P,e)$ is a positive MSC, it certainly defines sets of positive traces:

\vspace{-0.2cm}
\begin{equation*}
\mathcal{L}^{+}(N) = \mathcal{L}(P)
\end{equation*}

\noindent In addition, it defines the following set of negative traces:

\vspace{-0.2cm}
\begin{equation*}
\mathcal{L}^{-}(N) = \{~w.l \mid w \in mt(\mathcal{L}(P)) \wedge l = label(e)~\}
\end{equation*}

That is, negative traces are maximal traces of the precondition concatenated with the label of the proscribed event. Note that such a definition implies that the precondition must occur completely for the proscribed event to be taken into account. In other words, partial orderings between the proscribed event and those in the precondition are never considered. This is the intended meaning of the dashed line separating them~\cite{Uchitel:2004}. 

Last, observe that the set of negative traces cannot be captured with the sole mean of a LTS. This is because $\mathcal{L}^{-}(N)$ is not prefix-closed (see Section~\ref{section:background-lts-and-regular-languages}). Negative scenarios are sometimes captured by a LTS extended with an error state~\cite{Uchitel:2004}. Another way consists in using a standard automaton, which makes a distinction between accepting and non-accepting states. 

\subsubsection*{Multi-view model consistency}

Similarly to positive MSCs, we state conditions for a negative MSC $N = (P,e)$ and a system $\system$ to be consistent, as follows:

\begin{itemize}
\item $N$ and $S$ are \emph{architecturally} consistent,
\item $P$ and $S$ are consistent, and
\item $\mathcal{L}^{-}(N) \not\subseteq \mathcal{L}(Ag_1 \parallel \ldots \parallel Ag_n) = \mathcal{L}(S)$
\end{itemize}

The first condition is similar to what has been said previously for positive MSCs. The second enforces the precondition to be a consistent (positive) MSC, implicitly requiring positive traces to be accepted. The last one states that the system may not exhibit any negative trace captured by the negative MSC.

\subsection{Scenario collections}

Systems are generally illustrated with multiple positive and negative scenarios. The most straightforward way of doing so is through a scenario collection $Sc = (S^+,S^-)$ where $S^+$ and $S^-$ are finite (but possibly empty) sets of positive MSCs and negative MSCs, respectively. It is straightforward, yet useful, to extend the notions of language and consistency of scenarios to collections of them. 

The positive and negative languages defined by a scenario collection $Sc = (S^+,S^-)$ are simply defined via the union on languages, but taking into account the fact that negative scenarios define positive traces in addition to negative ones:

\vspace{-0.5cm}
\begin{align*}
\mathcal{L}^+(Sc) &= \bigcup_{P \in S^+} \mathcal{L}(P)~~\cup~~\bigcup_{N \in S^{-}} \mathcal{L}^{+}(N) \\
\mathcal{L}^-(Sc) &= \bigcup_{N \in S^-} \mathcal{L}^{-}(N)
\end{align*}

Also, a scenario collection and a system are said to be consistent if and only if each positive and each negative MSC of the collection is itself consistent with the system. We extend this to the consistence of the collection of scenarios itself as follows: two sets of positive and negative scenarios, $S^+$ and $S^-$, are consistent with each other if there exists a system which is consistent with them taken as a collection $Sc = (S^+,S^-)$. A necessary condition for a scenario collection to be consistent (but not sufficient, because architectural consistency is not taken into account here) is the disjointness of positive and negative traces:

\begin{center}
$Sc = (S^+,S^-)$ is consistent only if $\mathcal{L}^+(Sc) \cap \mathcal{L}^-(Sc) = \emptyset$
\end{center}

Note that, by definition, a collection cannot be consistent with a system unless all scenarios start in its initial state. Also, since scenarios and scenario collections are both finite, a collection is hardly complete in practice because most systems accept an infinite number of traces, through loops. Last, the fact that all scenarios must start in the same initial state may imply a lot of redundancy in descriptions of large systems. Among others, this can render consistency difficult to guarantee unless costly refactoring steps are conducted on scenarios. High-level Message Sequence Charts (hMSCs), introduced in the next section, provide a mean to tackle these problems.

\subsection[Flowcharting scenarios in high-level MSCs]{Flowcharting scenarios in high-level Message Sequence Charts\label{section:background-hmsc}}

A High Level Message Sequence Chart is a directed graph where each node refers to a MSC (named \emph{basic} MSC, bMSC for short) or a finer grained hMSC~\cite{ITU:1996}. We ignore this later possibility until further notice. Outgoing edges of a node capture its possible continuations, allowing the user to introduce sequences, alternatives and loops, to reuse small MSC fragments, and so on. An hMSC also has an initial starting point that indicates the initial system state. An example of hMSC is given in Fig.~\ref{image:train-hmsc}. 

\vspace{0.4cm}
\begin{figure}[H]\centering
\scalebox{0.66}{
  \includegraphics[trim=2mm 2mm 2mm 2mm, clip]{src/2-framework/images/train-hmsc}
}
\caption{A high-level Message Sequence Chart for the train system.\label{image:train-hmsc}}
\end{figure}

A trace semantics can of course be given to hMSCs, which amounts to answering the question \emph{what traces are captured by a hMSC?} This apparently simple question has, however, a more complex answer. The reasons are actually manifold; we summarize here, and slightly extend, the characterization given in~\cite{Uchitel:2004}.

First of all, some hMSCs do not define regular languages~\cite{Henriksen:2000}, and therefore have sets of traces that cannot be modeled with LTS. Therefore, we must restrict our framework to regular hMSCs. Under total ordering of events inside bMSCs, hMSCs are regular. Under partial ordering, a sufficient condition for being regular is that the hMSC does not contain a cycle in which two disjoint sets of agents interact independently of each other. We assume such \emph{bounded} hMSCs in the sequel. 

Also, in addition to the partial or total ordering of events in bMSCs, two interpretations co-exist about how a system evolves from a bMSC to another inside a hMSC. The first one, called \emph{strong sequential composition}, is to assume that all agents wait until all events of a bMSC have occurred before moving on to the next one. This implies that there is an implicit synchronization scheme used by agents to know when a scenario has been completed. The other one, \emph{weak sequential composition}, does not make such an assumption and allows an agent to move from a bMSC to another one without having to synchronize with the other agents. Given the independence between assumptions of partial/total event ordering and weak/strong sequential composition, four combinations actually exist. To keep things simple enough, and because they lead to strange concurrency models\footnote{in particular, one can check that such models are very sensitive to bMSC concatenation and decomposition}, we do not consider partial (resp. total) ordering with strong (resp. weak) sequential composition. In the sequel, therefore, strong (resp. weak) sequential composition implies total ordering (resp. partial) of MSC events.

Under these assumptions, the trace analysis for hMSCs is very similar to what has been done for MSCs, and therefore leads to language relations similar to the ones given for the latter in Section~\ref{subsection:background-positive-scenarios} (see equation~\ref{equation:msc-composition}):

\begin{equation}
\mathcal{L}_{strong}(H) \subseteq \mathcal{L}_{weak}(H) \subseteq \mathcal{L}(H_{\downarrow Ag_1} \parallel \ldots \parallel H_{\downarrow Ag_n})
\label{equation:hsmc-traces-by-agent-composition}
\end{equation}

The set of traces $\mathcal{L}_{strong}(H)$ and $\mathcal{L}_{weak}(H)$ are easily explained by considering scenarios ``built'' by the hMSC. Consider a finite path in the hMSC. Concatenating the bMSCs along this path ``builds'' a single MSC. For example, concatenating \artifact{Starting train}, \artifact{Pressing alarm} and \artifact{Stopping \& Opening the doors} in the hMSC of Fig.~\ref{image:train-hmsc} leads to the MSC of Fig.~\ref{image:train-scenario-all-agents}. From results in Section~\ref{subsection:background-positive-scenarios}, such a MSC $M$ defines the set of traces $\mathcal{L}_{total}(M)$ and $\mathcal{L}_{partial}(M)$. To define traces defines by a hMSC $H$, we have to consider all such possible MSCs: 

\vspace{-0.5cm}
\begin{align*}
\mathcal{L}_{strong}(H) &= \bigcup_{M \in H} \mathcal{L}_{total}(M) \\
\mathcal{L}_{weak}(H) &= \bigcup_{M \in H} \mathcal{L}_{partial}(M)
\end{align*}

%\vspace{-0.8cm}
\noindent where $M \in H$ means ``the MSC $M$ can be built by $H$'', with the obvious meaning in terms of possible paths in $H$. The language $\mathcal{L}_{weak}$ corresponds to the notion of \emph{trace model} in~\cite{Uchitel:2004}.

Now, one can actually consider another way to compute the set of traces defined by a hMSC. In this alternative way, the local perspective of the agent timelines is taken into account, in a way similar to what has been done for MSCs. This leads to the rightmost part of the relations between hMSC languages in~\ref{equation:hsmc-traces-by-agent-composition}, that computes hMSC traces as the composition of local agent traces $H_{\downarrow Ag_i}$. The composition of such a LTS for each agent correspond to the notion of \emph{minimal architecture model} in~\cite{Uchitel:2004} and will sometimes be denoted by $\mathcal{L}_{arch}(H)$.

An algorithm can be found in~\cite{Uchitel:2004} for synthesizing the LTS modeling $\mathcal{L}_{arch}(H)$. For a given agent $Ag_{i}$, the LTS $H_{\downarrow Ag_i}$ is synthesized by connecting the LTSs $M_{\downarrow Ag_i}$ corresponding to each bMSC $M$ with $\tau$ transitions, according to their possible continuations given by hMSC edges. This construction is illustrated in Fig.~\ref{image:train-controller-synthesis} for the hMSC of Fig.~\ref{image:train-hmsc} and the \artifact{Controller} agent. The resulting LTS can be further simplified by removing $\tau$ transitions and minimizing the LTS.  Note that, thanks to the simplicity of the model with strong sequential composition, $\mathcal{L}_{strong}(H)$ can be computed in a very similar way. In contrast, $\mathcal{L}_{weak}(H)$ is much harder to model as a LTS; a synthesis algorithm can however be found in~\cite{Uchitel:2004}.

\vspace{0.4cm}
\begin{figure}[H]\centering
\scalebox{0.85}{
  \includegraphics[trim=0mm 0mm 0mm 0mm, clip]{src/2-framework/images/train-controller-synthesis}
}
\caption{Synthesis of the \emph{Controller} LTS from the hMSC of Fig.~\ref{image:train-hmsc}.\label{image:train-controller-synthesis}}
\end{figure}

\aside{One can note an important difference between the relations between MSC languages in~\ref{equation:msc-composition} and those between hMSC languages in~\ref{equation:hsmc-traces-by-agent-composition}. Indeed, the former denotes an equality between $\mathcal{L}_{partial}(M)$ and $\mathcal{L}(M_{\downarrow Ag_1}\parallel\ldots\parallel M_{\downarrow Ag_n})$ while the latter defines a set inclusion between $\mathcal{L}_{weak}(H)$ and $\mathcal{L}(H_{\downarrow Ag_1}\parallel\ldots\parallel H_{\downarrow Ag_n})$. In fact, traces in $\mathcal{L}_{arch}(H) \setminus \mathcal{L}_{weak}(H)$ capture the set of \emph{implied} scenarios of a hMSC specification~\cite{Alur:2000, Uchitel:2004}. Implied scenarios occur when a system is designed globally (the hMSC) while implemented component-wise (the composition of agent LTSs). In other words, implied scenarios capture traces that follow different paths in the hMSC when projected on individual agent timelines. We come back to them in Section~\ref{section:background-discussion}.}

\subsubsection*{Finer-grained hMSC}

As stated previously, a hMSC node can refer to a finer-grained hMSC instead of a bMSC. Taken them into account in the trace semantics given previously requires deciding how a sub-hMSC has to be ``connected'' with its father. To keep a consistent framework in terms of synchronization hypotheses, we require such a sub-hMSC to have, in addition to its initial state, a terminating state to which at least one node is connected. For simplicity, we also forbid nodes with no outgoing transition. Under such assumptions, it is relatively easy to ``unfold'' a hMSC as another one with all nodes refined as basic MSCs. The trace semantics remains unchanged and is defined in terms of the latter ``flat'' hMSC. 

In practice, this flat hMSC must not be explicitly constructed when synthesizing the LTS for $\mathcal{L}_{arch}$. Indeed, the LTSs $H_{\downarrow Ag_i}$ of finer-grained hMSCs respecting our constraints present only one initial state and only one terminating state, allowing them to be connected with $\tau$ transitions in the same way than LTSs $M_{\downarrow Ag_i}$. Similar argument applies for $\mathcal{L}_{strong}$, but not for $\mathcal{L}_{weak}$.

\subsubsection*{Multi-view model consistency}
 
Based on the trace semantics defined in the previous section, we can now define additional consistency rules in our framework. We say that a hMSC $H$ and a system $\system$ are consistent if the following condition holds:

\begin{itemize}
\item $H$ and $S$ are \emph{architecturally} consistent,
\item $\mathcal{L}(H_{\downarrow Ag_i}) \subseteq \mathcal{L}(Ag_i)$ for each agent $Ag_i$, and
\item $\mathcal{L}_{arch}(H) = \mathcal{L}(H_{\downarrow Ag_1} \parallel \ldots \parallel H_{\downarrow Ag_n}) \subseteq \mathcal{L}(Ag_1 \parallel \ldots \parallel Ag_n) = \mathcal{L}(S)$
\end{itemize}

These conditions are the counterpart of those given previously for MSCs and have a similar interpretation, \emph{mutatis mutandis}. In addition, however, it is convenient to distinguish between a hMSC that describes all behaviors of a system and one that only illustrates a subset of them. This leads to the notion of hMSC completeness. A hMSC $H$ is \emph{complete} for a system $\system$ if it is consistent with it and defines the same language, that is, if $\mathcal{L}_{arch}(H) = \mathcal{L}(S).$

\subsection{Explicit state annotation}

A last construction allowed in MSCs is worth discussing that helps managing multiple scenarios. Indeed, many authors allow MSCs to be annotated with state-based information (see, e.g.,~\cite{Kruger:2000, Whittle:2000}). 

TODO: complete this description, discussing the difference between identification and state assertions (aka fluent-based decorations). The former info uniquely identifies an agent state, so that they can be merged with mandatory merge constraints presented in section~\ref{section:inductive-from-hMSC}, while the later does not, but helps pruning search space as usual with fluents.

\section{State-based invariants through Fluents\label{section:background-fluents}}

Miller and Shanahan define fluents as ``\emph{time-varying properties of the world that are true at particular time-points if they have been initiated by an event occurrence at some earlier time-point, and not terminated by another event occurrence in the meantime. Similarly, a fluent is false at a particular time-point if it has been previously terminated and not initiated in the meantime}''~\cite{Miller:2002}.

Mathematically, a fluent $Fl$ is a proposition defined by a set $Init_{Fl}$ of initiating events, a set $Term_{Fl}$ of terminating events, and an initial value $Initially_{Fl}$ that can be true or false. The sets of initiating and terminating events must be disjoint. The concrete syntax for fluent definition is the following~\cite{Giannakopoulou:2003}:

\begin{center}
fluent $Fl = \textless Init_{Fl}, Term_{Fl} \textgreater $ initially $Initially_{Fl}$
\end{center}

In our train example, the safety goal ``\emph{\texttt{Doors shall remain closed while the train is moving}}'' suggest two fluents defined as follows:

\begin{center}
fluent $Moving = \textless \{start\}, \{stop\} \textgreater $ initially $false$ \\
fluent $DoorsClosed = \textless \{close\}, \{open\} \textgreater $ initially $true$ \\
\end{center}

\subsection{Fluent values along single traces}

Fluents nicely interface with trace semantics as follows. A fluent $Fl$ is said to be $true$ after a finite trace $s$ if and only if one of the following conditions hold~\cite{Giannakopoulou:2003}:

\begin{enumerate}
\item $Fl$ holds initially and no terminating event has occurred in $s$.
\item Some initiating event has occurred in $s$ with no terminating event occurring since then.
\end{enumerate}

Given this definition, for example, the fluent $Moving$ is $true$ after the trace \artifact{<start stop start>}, but not after the empty trace $\lambda$ nor after \artifact{<start stop>}. Note that, as initiating and terminating events are disjoint, the value of a fluent after a given trace is deterministic. Also, taking trace prefixes into account, it is straightforward to think in terms of fluent values \emph{along} traces by rephrasing the above definition as a recursive version~\cite{Damas:2005}. This is illustrated in Fig.~\ref{image:fluent-values-along-a-trace} where the values of the two fluents given above are shown in the states of a LTS modeling a typical event trace for the train system. 

\begin{figure}[H]\centering
\scalebox{0.45}{\includegraphics{src/2-framework/images/decorating-trace}}
\caption{Fluent values along a single trace, captured with a LTS. Here \artifact{DoorsClosed} is abbreviated as \artifact{Closed}.\label{image:fluent-values-along-a-trace}}
\end{figure}

As the example suggests, given a set of fluents $\Phi$, a trace (and any of its prefixes) defines an interpretation over $2^\Phi$, that is the assignment of a boolean value to each fluent in $\Phi$. We call them \emph{fluent value assignments} throughout this thesis. For example, the (maximal) trace in the figure defines the assignment $\{Moving \rightarrow false,~DoorsClosed \rightarrow false\}$. 

\subsection{Fluent values along multiple traces}

Interestingly, previous results can be generalized for annotating states of any LTS. The generalization consists in considering that a LTS state may be reached by a set of traces instead of a single one. Therefore annotations become \emph{sets} of fluent assignments, that is elements of $\mathcal{P}(2^\Phi)$. In particular, a state could be reached by a trace rendering a fluent $true$, while another trace reaching it would render the same fluent $false$. Remark however that even in presence of a possibly infinite number of traces, the set of all possible annotations is itself finite. Moreover, $\mathcal{P}(2^\Phi)$ actually coincides with the set of propositional formula over fluents. In fact, it is convenient to interpret state annotations as such formula, where $false$ then corresponds to an empty set of fluent assignments (i.e. an unreachable state) and \artifact{true} corresponds to the set of all possible assignments. We call those state annotations \emph{fluent invariants}, as they encode assertions that always hold when the LTS state is visited. An example is given in Fig.~\ref{image:fluent-values-along-multiple-traces}.

\begin{figure}[H]\centering
\scalebox{0.45}{\includegraphics{src/2-framework/images/decorating-lts}}
\caption{Fluent values along multiple traces. Here \artifact{DoorsClosed} is abbreviated as \artifact{Closed}.\label{image:fluent-values-along-multiple-traces}}
\end{figure}

A fix-point algorithm for decorating a LTS with fluent assignments first appeared in~\cite{Damas:2005}; it is generalized for handling state invariants in \cite{Damas:2009}. In short, it consist in annotating the LTS initial state with an initial invariant, according to fluent initial values. Invariants are propagated along LTS transitions, according to fluent definitions and accumulated in states through boolean disjunction until a fix point is reached. One can profit from binary decision diagrams~\cite{Bryant:1986} for concisely encoding and efficiently manipulating boolean formula in an implementation. This algorithm is generalized once again in~\cite{Damas:2011} to work on guarded LTS (see chapter~\ref{chapter:deductive}) as well as for handling other kinds of decorations than state invariants.

\subsection{Integrating fluents in multi-view models}

Fluents, as described in the previous section, provide a general mechanism for interfacing event-based models with state-based abstractions. In this section, we restrict our own use of this mechanism so as to cleanly interface with agents, their state machines, and scenarios. The next section naturally extends this discussion by introducing fluent-based goals and domain properties in the framework.

First of all, we restrict our use of fluents to those \emph{monitored} and \emph{controlled} by the agents forming the system. A fluent is monitored by an agent if all its initiating and terminating events are either sent or received by this agent; it is controlled if all initiating and terminating events are sent by the agent. Considering only monitored and controlled fluents is motivated by the need for state machines and goals to be realizable by their agents~\cite{Letier:2002} (see also next section).

We have seen in the previous section that the value of a fluent is not necessarily known in a LTS state because some traces could render it $true$ while others could render it $false$. While there is no strong agreement for viewing this as problematic in general, we consider a good practice not to fall in such a situation between the fluents monitored and controlled by an agent and the state machine modeling its behavior. The reason is that LTSs are not considered a structured form of state machines. Therefore, a LTS state should not be viewed as a class of agent states, that is, it should correspond to a unique assignment of agent state variables; fluents are such variables. The LTS state machine shown in Fig.~\ref{image:fluent-values-along-multiple-traces} respects this sane property. In particular, one can check that propositional formula shown in states are satisfiable by only one interpretation (fluent invariants ``degenerate'' to fluent value assignments). In such a case, an agent state uniquely defines the value of its monitored and controlled fluents (this property is conserved under LTS composition by construction). The contrapositive of this property is not true: as illustrated in the same figure, fluent value assignments do not necessarily identify LTS states.

Following previous assumptions and results, MSC timelines can easily be annotated with agent invariants on monitored and controlled fluents. However note that, as with LTSs, these invariants do not necessarily identify agent states in a unique manner.

\section{Declarative goals and domain properties\label{section:background-goals}}

A \emph{goal} is a prescriptive statement of intent, whose satisfaction requires the collaboration of agents forming the system. Unlike goals, \emph{domain properties} are descriptive statement about the environment -- such as physical laws, organizational rules, etc. Goals are structured in AND/OR refinement graphs showing how they contribute to each other~\cite{VanLamsweerde:2000}.

Goals and domain properties can be formalized in Linear Temporal Logic (LTL), that allows specifying admissible and/or proscribed system histories in a declarative and implicit way (that is, without requiring an explicit time parameter)~\cite{VanLamsweerde:2009}. In addition to the usual propositional constructs (we ignore first-order constructs here), LTL provides operators for temporal referencing: $\circ$ (at the next smallest time unit), $\diamond$ (some time in the future), $\square$ (always in the future), $\rightarrow$ (implies in the current state), $\Rightarrow$ (always implies), $\mathcal{U}$ (always in the future until), $\mathcal{W}$ (always in the future unless), see~\cite{Manna:1992}.

However, a system history is commonly viewed in LTL as a temporal sequence of system states. Also, atomic propositions of LTL formula often refer to state variables (e.g. in the SPIN model-checker~\cite{Holzmann:1997}). In contrast, a system history is seen as a trace here, that is, a temporal sequence of events. Capitalizing on fluents for reconciling state-based and event-based paradigms (see previous section), we use a flavor of LTL known as Fluent Linear Temporal Logic (FLTL), a linear temporal logic in which atomic propositions are fluents~\cite{Giannakopoulou:2003}. FLTL proves a convenient way for specifying state-based temporal logic properties over the event-based operational model given by our scenarios and state machines. For example, the safety goal ``\emph{Doors shall remain closed while the train is moving}'' of our running example can be formalized in terms of the $Moving$ and $DoorsClosed$ fluents defined in the previous section, as follows:

\begin{center}
\artifact{Maintain[DoorsClosed While Moving]} = $\square(Moving \rightarrow DoorsClosed)$
\end{center}

Properties formalized in temporal logic are often classified as \emph{liveness} or \emph{safety} properties~\cite{Alpern:1986}. The difference between both is generally illustrated with the distinction between ``\emph{something good will eventually happen}'' (liveness) and  ``\emph{something bad never happens}'' (safety). We restrict our attention to safety properties in this thesis. Indeed, reasoning about liveness properties requires considering the acceptance of infinite execution traces, which it ouside the expressiveness of LTSs and regular languages. In contrast, if ``something bad'' happens it must do so after a finite sequence of events, and is irremediable~\cite{Alpern:1986, Giannakopoulou:1999}. Strictly speaking, as with negative scenarios, this is just outside the expressiveness of LTSs, but not of regular languages. Let us explain this by stating the consistency rules first.

\subsubsection*{Multi-view model consistency}

Consider a safety property $G$ and a system $\system$. Let denote by $\mathcal{L}^{-}(G)$ the set of traces that violates the property. We say that the system respects the property, that is that they are consistent if and only if the following condition holds:

\begin{center}
$\mathcal{L}^{-}(G) \cap \mathcal{L}(Ag_1 \parallel \ldots \parallel Ag_n) = \emptyset$
\end{center}

\subsection{Tester automaton}


\section{Discussion\label{section:inductive-synthesis-discussion}}

Let us step back a little before closing this chapter and ask two questions:
\begin{itemize}
\item \emph{What problem are we trying to solve?}
\item \emph{Why do we solve it this way?}
\end{itemize}
In addition to revisiting our synthesis technique and discussing some perspectives, answers to those questions allow us to compare our approach with other ones. Such comparisons will take place in Chapter~\ref{chapter:related-work}.

\subsection{What problem are we trying to solve?}

This chapter discussed an horizontal synthesis technique. Remember from Section~\ref{section:intro-synthesis} that the underlying objectives are to elaborate requirements and explore system designs. In this context, horizontal synthesis aims at building models missing from a multi-view framework or completing existing ones.

In the light of these objectives, our technique can be seen as tackling the following problem:
\begin{itemize}
\item There is an expected target system, composed of agents $Ag_1, \ldots, Ag_n$ whose behavior can be modeled through state machines. The behavior of the system itself is defined through agent composition.
\item A complete description of the system behavior is not available. In particular, agent state machines are unknown or only partially known; the same applies to the underlying behavioral goals.
\item However, behavior model fragments are available; those fragments take various forms:
\begin{itemize}
\item Scenarios may describe examples and counterexamples of desired system behaviors.
\item The behavior of some agents may be partially or completely captured through definitions of agent state variables and/or known state machines.
\item Known behavioral goals may define restrictions on system behaviors so as to avoid undesired ones.
\end{itemize}
\item We are interested in specifying the system behaviors more completely. This roughly means completing all models in a consistent way.
\end{itemize}

The problem above presuppose incremental approaches to be solved. Within iterations, system descriptions gets more complete and behavior models gets richer. Additional system features are modeled when successful iterations have completed; refactoring steps are conducted when problems appear, such as undesired implied scenarios.

Our inductive synthesis technique provides a building block for an integrated approach along this line. Observe that the synthesis of state machines from scenarios can be seen as a ``pretext''. In the multi-view vision above, state machine synthesis is seen as equally important as the completion of the scenario collection with answers to scenario questions; the same applies to the identification of behavior goals triggered when some of these scenarios are rejected. 

The ISIS tool goes one step further as it integrates QSM with other synthesis techniques, such as the inference of behavioral goals from scenarios (see Section~\ref{section:tool-support-isis}).

\subsection{Why do we solve it this way?}

Some standpoints in the problem formulation above are debatable. One could ask whether this incremental process has an end or argue that the ultimate goal is the elaboration of requirements in particular. 

Whatever the answers, we state two reasonable expectations on approaches supporting the incremental building of a multi-view behavior model:
\begin{quotation}
\emph{Every time a successful iteration completes, the available models should provide a consistent view of system behaviors.}
\end{quotation}
\begin{quotation}
\emph{If failing to complete, the elaboration process should at least provide a reasonable convergence criteria.}
\end{quotation}

Section~\ref{section:inductive-correctness} proposed an detailed discussion about the consistency guarantees that one can expect with our approach.

The convergence of the elaboration process also deserves some attention. To keep things simple enough, we leave aside structural aspects hidden behind fluents state variables, our use legacy components, the structure of scenarios, the decomposition step of our approach, and implied scenarios. In other words, we focus here on pure behavioral aspects of scenarios and goals.

In such case, our problem can be stated once again as follows \cite{Uchitel:2007}:
\begin{itemize}
\item There is an expected target system, denoted by $S$. Let denote its expected behaviors by $\mathcal{L}(S)$.
\item Positive scenarios capture a lower bound on system behaviors, that is, behaviors that the system \emph{must} provide. Let denote these behaviors by $\mathcal{L}^+(S)$.
\item Negative scenarios and goals capture an upper bound on system behaviors, that is, they capture behaviors that the system \emph{may not} exhibit. Let denote these behaviors by $\mathcal{L}^-(S)$.
\item We are interested in specifying the system behaviors more completely. This roughly means completing scenario and goal descriptions in a consistent way.
\end{itemize}

In addition, the two expectations on the elaboration process can be stated as follows:
\begin{itemize}
\item Every time a successful iteration completes, scenarios and goals should provide a consistent view of system behaviors:
\begin{align}
&\mathcal{L}^+(S) \subseteq \mathcal{L}(S)\\
&\mathcal{L}^-(S) \cap \mathcal{L}(S) = \emptyset
\end{align}
\item If failing to complete, the elaboration process should at least provide a reasonable convergence criteria. That is, it should ultimately lead to a situation where scenarios and goals possibly describe the same system:
\begin{align}
\mathcal{L}^+(S) = \mathcal{L}(S) = \overline{\mathcal{L}^-(S)}
\end{align}
\end{itemize}

%\subsection{Rationale behind the generalization step}

%Our synthesis requirements include ``behavior generalization''. Looking at relation (\ref{relation:inductive-language-refinements}) we might ask ourselves what drives the generalization process and until when. By construction, the first automaton $A_0$ already meets the consistent system view conditions (\ref{relation:inductive-invariant}) and (\ref{relation:inductive-invariant-II}). Therefore, $A_0$ is a valid, yet trivial, solution. Why not simply use it?

%The answer is to be found in the Occam's principle stating that ``among all models explaining the world equally well, the simplest should be preferred''. Using grammar induction this amounts to searching for the \emph{smallest} automaton consistent with the positive and negative scenarios, also called the \emph{input sample}. The initial automaton $A_0$ is rarely the simplest model according to this criteria. 

%Looking for the smallest automaton consistent with an input sample is known to be NP-hard \cite{Gold:1978, Angluin:1978}. The RPNI algorithm offers a consistent approximated solution in polynomial time; this solution is the smallest consistent deterministic automaton when the input sample is rich enough, in particular when it forms a so-called \emph{characteristic sample} (see later) \cite{Oncina:1992}.

%\subsection{Future directions}

%From a grammar induction point of view, the ASM algorithm can be seen as generalizing any positive regular language $\mathcal{L}^+$ under the control of a negative sample $S^-$. As such, RPNI is thus a special case where the positive language forms a sample $S^+$, that is a finite set of strings.

%Moreover, goals and domain properties can still be used to prune the ASM search space with the technique discussed in Section~\ref{subsection:induction-pruning-with-goals}. As goals actually capture negative languages through their tester automaton, this amounts to consider a generalization of ASM to generalize a positive language $\mathcal{L}^+$ under the control of a negative one $\mathcal{L}^-$. This generalization is called ASM$^*$ and briefly discussed in \cite{Lambeau:2008}.

