\section{Running examples\label{section:background-running-examples}}

This section introduces the two running examples that will be used throughout the thesis. Section \ref{subsection:background-train-system} introduces a simple train control system. Section \ref{subsection:background-meeting-scheduler} introduces a meeting scheduler system \cite{Feather:1997}.

\subsection{A simple train control system\label{subsection:background-train-system}}

A simplified train control system will be used as a running example for illustrating behavioral model concepts and techniques throughout the thesis. 

The system is composed of a software train controller, actuators for doors and train acceleration, sensors and passengers. Through the actuators, the software controller controls operations like starting or stopping the train, opening or closing doors, and so on. A safety goal requires train doors to remain closed while the train is moving. If the train is not moving and the passenger presses the alarm button, the controller must open the doors immediately. If the train is moving and the passenger presses the alarm button, the controller must stop the train first and then open the doors. 

The latter situation is illustrated with a scenario in Fig.~\ref{image:train-scenario-all-agents}, where horizontal arrows denote interaction events among agent instances. The precise semantics of such scenario will be made clear in the following sections.

\begin{figure}[H]\centering
\scalebox{0.75}{
  \includegraphics[trim=0mm 0mm 0mm 0mm, clip]{src/2-framework/images/train-scenario-all-agents}
}
\caption{A scenario illustrating train stopping in emergency when an alarm is pressed.\label{image:train-scenario-all-agents}}
\end{figure}

\subsection{The meeting scheduler\label{subsection:background-meeting-scheduler}}

For illustrating process models and associated techniques, we will focus on the following episode of a meeting scheduling process \cite{Feather:1997}. 

A meeting initiator issues a meeting request, specifying the expected participants and the date range within which the meeting should take place. The scheduler then sends an electronic invitation to each participant, requesting them to provide their date constraints. 

A date conflict occurs when no date can be found that fits all participant constraints. In such case, the initiator may extend the date range or request some participants to weaken their constraints; a new scheduling cycle is then required. Otherwise, the meeting is planned at a date meeting all constraints.

A soft goal requires meetings to be scheduled as quickly as possible once initiated; another one requires interactions with participants to be minimized. In the simplified version considered here, only two scheduling cycles are allowed; the meeting is automatically planned after that. In such case, we will assume that the best date is chosen so as to maximize the number of participants attending. We also ignore features like meeting cancellations, meeting locations, and so on.

\begin{figure}[H]\centering
\scalebox{0.60}{
  \includegraphics[trim=2mm 2mm 2mm 2mm, clip]{src/2-framework/images/scheduler-ghmsc}
}
\caption{A process model for a meeting scheduling process.\label{image:scheduler-ghmsc}}
\end{figure}

This meeting scheduling process is illustrated in Fig.~\ref{image:scheduler-ghmsc}, where boxes denote tasks. The precise semantics of such model will be detailed in Chapter~\ref{chapter:deductive}.

