\chapter{Derivation of State Machine Models from Process Models\label{chapter:deductive}}

This chapter shows how state machines can be synthesized from process models. Section \ref{section:deductive-motivation} motivates such derivation and outlines the approach. Section \ref{section:deductive-glts} introduces guarded labeled transition systems (g-LTS) as an intermediate formalism between guarded hMSCs and LTS state machines. Algorithms for transforming process models into guarded and non-guarded transition systems are detailed in Section \ref{section:deductive-glts-to-lts}.

Models are increasingly recognized as an effective means for elaborating requirements and exploring designs of software systems. A model is an abstract representation of the target system, where key features are highlighted, specified and inter-related to each other \cite{VanLamsweerde:2009}. Models are widely used at different steps of the software development process and for different software-related purposes.

%%%

\emph{Requirements engineering} for example, aims at deciding precisely what the system should do. It has been claimed that this is the hardest part of software development \cite{Brooks:1987}. Using models for exploring and stating requirements has multiple benefits:

\begin{itemize}
\item They force stakeholders and analysts to be more precise in their system description.
\item They allow them to abstract from multiple details in order to focus on key system aspects.
\item They provide a basis for early detection and fixing of errors.
\end{itemize}

Model-oriented approaches to system requirements include KAOS \cite{VanLamsweerde:2009} and NFR/i$^*$ \cite{Mylopoulos:1992, Yu:1993}. Models explain how system objectives may be refined in lower-level objectives, also called goals. KAOS emphasizes more on semi-formal and formal reasoning about behavioural goals, i.e. goals that prescribe system behaviors declaratively. In NFR/i$^*$, the emphasis is more on qualitative reasoning on soft goals, i.e. goals that cannot be established in a clear-cut sense.

%%%

\emph{Software design} may be captured by Model Driven Engineering (MDE) approaches and driven by UML models \cite{OMG:2004} and Domain Specific Languages \cite{VanDeursen:2000, Fowler:2010}. Such approaches model the application domain rather than the computing or algorithmic concepts. The use of models here has several advantages:

\begin{itemize}
\item They enforce the separation between the abstract domain concepts and their concrete implementation. This allows focusing on each aspect at the adequate level of abstraction.
\item Code fragments can often be generated from models, which is less error-prone and time consuming than manually translating models to code. In the case of domain specific languages, models may even be truly executable already. 
\item Design models also yield a natural documentation for the software and the underlying design decisions. 
\end{itemize}

%%%

\emph{Work processes} may be represented by different graphical models. Such models take different forms: flowchart style, e.g. Activity Diagrams \cite{OMG:2004} or Business Process Modeling Notation (BPMN) \cite{OMG:2008}, task trees, e.g. Little-Jil \cite{Clarke:2008}. Such models may be limited to graphical representations, e.g. Activity Diagrams and BPMN or have a strong formal semantics, e.g. Little-Jil and YAWL \cite{Vanderaalst:2005}. These approaches allow:

\begin{itemize}
\item Analyzing and verifying models before translating them to work organization on the field. This allows early detecting errors and bottlenecks, among others.
\item Partly or fully triggering the execution of necessary tasks, through reminders to process actors or automated processing for example.
\end{itemize}

%%%

Complex systems are better described with multiple views, each of them being captured through a specific kind of model. Such models focus on particular facets of the system, i.e. its intentional, structural, operational and behavioral dimensions. Scenarios, for example, illustrate interactions between agents forming the system whereas goals make the underlying objectives precise. Models often overlap on specific aspects of their system descriptions. For example, the agents forming the system appear both in scenarios and in a goal model.

Reaching quality models in such multi-view approaches is far from an easy task. Adequate modeling languages, tools and techniques are therefore required to help building them. To play a significant role, a good multi-view modeling language should meet the following requirements \cite{VanLamsweerde:2009}:

\begin{itemize}
\item \emph{Multi-level}: the system should be captured at different levels of abstraction and precision to enable stepwise elaboration and validation;
\item \emph{Analyzable}: the modeling abstractions should be accurate enough to support useful forms of analysis.
\end{itemize}

In addition, such modeling language should support stakeholders and analysts in building quality models. A ``good'' model should meet the following requirements \cite{VanLamsweerde:2009}:

\begin{itemize}
\item \emph{Adequate}: the models should adequately represent the essence of the target system while abstracting unnecessary details;
\item \emph{Complete}: the models should capture all pertinent facets of the system along the \textsc{why-}, \textsc{what-} and \textsc{how-} dimensions;
\item \emph{Precise}: the models should be accurate enough to capture system descriptions with as less ambiguity as possible;
\item \emph{Consistent}: the models should agree on their overlapping descriptions of the system;
\item \emph{Comprehensible}: the models should be easy enough to understand by the people who need to use them;
\end{itemize}

Formal modeling approaches help meeting the requirements above. The semantics of a formal modeling language provides precise rules of interpretation that allow many of the problems with natural language to be overcome. Formal specifications may also be manipulated by automated tools for a wide variety of purposes. Among them, model \emph{analysis} and model \emph{synthesis} are intertwined activities that help reaching quality models, mostly along completeness, consistency  and precision.

\begin{description}

\item[Model analysis] may be done at surface level, e.g. by query-based checks on model \cite{VanLamsweerde:2009}, or more deeply according to different types of approaches, such as model-checking \cite{Clarke:1986, Queille:1982}. For example, model analysis may be used:

\begin{itemize}
\item to detect inconsistencies between different system views;
\item to complete system descriptions, using examples and counter-example returned by queries and checks to trigger new model construction;
\item to gain modeling precision and accuracy by incrementally refining models as the result of analysis feedback;
\end{itemize}

\item[Model synthesis] helps creating and completing models thanks to multiple sources of knowledge. Among others, model synthesis may be used:

\begin{itemize}
\item to complete a multi-view description by synthesizing missing models or completing existing ones;
\item to define the operational semantics of high-level models through their rewriting as lower-level ones;
\item to make analyses available at a high-level of system description that were available at those lower-levels;
\end{itemize}

\end{description}

In practice, the boundaries between model synthesis and model analysis are a bit fuzzy. As stated above, model synthesis helps making model analysis available; on the other hand, feedback of analyses may actually yield models and can thus be seen as a kind of model synthesis. Be that as it may, the thesis investigates model synthesis more specifically; its companion by Christophe Damas \cite{Damas:2011} investigates model analysis. Both report research efforts on the same kind of models: scenarios, state machines, goals and processes.



\section{Introducing guarded transition systems\label{section:deductive-glts}}

This section introduces guarded LTS (g-LTS) as an intermediate formalism between guarded hMSCs and LTS. This formalism provides a convenient milestone on the way from a guarded hMSC to the corresponding LTS, in particular, for determining the set of traces accepted by the guarded hMSC. 

\begin{definition}[Guarded LTS]
\noindent A guarded LTS is defined as a structure $(Q,\Sigma,\Phi,\delta,q_{0},C_{0})$ where 
\begin{itemize}
\item $Q$ is a finite set of states,
\item $\Sigma$ is a set of event labels, 
\item $\Phi$ is a set of fluents defined on $\Sigma$,
\item $\delta$ is a transition relation $Q \times L \times Q$,
\item $q_{0} \in Q$ is the initial state,
\item $C_{0} \in \mathcal{P}(2^\Phi)$ is an initial condition on fluents.
\end{itemize}
where the set of transition labels is defined as $L = \Sigma \cup \{\tau\} \cup \mathcal{P}(2^\Phi)$ 
\end{definition}

In a guarded LTS, transitions are labeled either by a guard, an event, or $\tau$. As with LTS, $\tau$-transitions are non-observable from the environment (see Section \ref{section:background-state-machines}). In the sequel, $\Sigma_{\tau}$ will denote $\Sigma \cup \{\tau\}$.

A guard is a propositional formula over fluents in $\Phi$. The set of such formula is denoted by $\mathcal{P}(2^\Phi)$, as in Section \ref{section:background-fluents}. Intuitively, the guard must evaluate to true for its transition to be activated. This is made more precise is the next section.

The condition $C_0$ plays the same role as in guarded hMSCs. It is a Boolean formula that constrains the acceptable initial values of fluents. 

\subsection{Trace semantics\label{subsection:glts-trace-semantics}} 

The semantics of g-LTS is defined in terms of event traces involving no guards at all.

\begin{definition}[g-LTS execution]
An \emph{execution} of a g-LTS $G = (Q,\Sigma,\Phi,\\\delta,q_{0},C_{0})$ from $q_0$ is a pair $(Init, \textless l_0, \ldots,l_n\textgreater)$, where 
\begin{itemize}
\item $Init \in 2^\Phi$ is an initial fluent value assignment mapping every fluent in $\Phi$ to true or false,
\item $\textless l_0,\ldots,l_n \textgreater$ is an finite sequence of labels $l_i \in \Sigma_{\tau}\cup\mathcal{P}(2^\Phi)$, each of them being either an event label, $\tau$, or a guard.
\end{itemize}
\label{definition:glts-execution}
\end{definition}

Due to guards, only certain executions are to be considered valid from the initial state of the g-LTS. This is captured by the following definition.

\begin{definition}[Valid g-LTS execution]
An execution $S = (Init,\textless l_0,\ldots,\\l_n \textgreater)$ of a g-LTS $(Q,\Sigma,\Phi,\delta,q_{0},C_{0})$ is \emph{valid} from its initial state $q_0$ iff the following \emph{acceptance conditions} are met for every $i$ $(0 \leqslant i < n)$:\\
\vspace{-0.8cm}
\begin{tabbing}
\indent trace inclusion:~~~~~~~\= $\exists q_{i+1} \in Q$ such that $(q_i,l_i,q_{i+1}) \in \delta$\\
\indent admissible start:      \> $Init \models C_0$ \\
\indent guard satisfaction:    \> $S_i \models l_i$ if $l_i \in \mathcal{P}(2^\Phi)$\\
\end{tabbing}
\vspace{-0.8cm}
where $S_i$ is the fluent value assignment after the i-th event in the trace, with $S_0 = Init$.
\label{definition:valid-glts-execution}
\end{definition}

\begin{description}
\item[The trace inclusion condition] states that the label sequence must denote an existing path in the automaton.
\item[The admissible start condition] states that the initial fluent value assignment must meet the initial condition $C_0$.
\item[The guard satisfaction condition] ensures that all guards are met along the sequence. The fluent value assignment $S_i$ is determined by fluent definitions, as explained in Section \ref{subsection:background-fluents-single-traces}. 
\end{description}

As with LTS, the trace semantics of g-LTS is defined in terms of accepted \emph{event} traces. Roughly, it consists of all valid executions where $\tau$ labels as well as guards are removed. This is precisely captured by the the following definitions.

\begin{definition}[g-LTS trace]
A g-LTS trace is a pair $(Init,\textless e_0,\ldots,e_n\textgreater)$ where 
\begin{itemize}
\item $Init \in 2^\Phi$ is an initial fluent value assignment mapping every fluent in $\Phi$ to true or false,
\item $\textless e_0,\ldots,e_n \textgreater$ is an finite sequence of event labels, with $e_i \in \Sigma$.
\end{itemize}
\end{definition}

Unlike an execution, a g-LTS trace only contains event labels, but no $\tau$ nor guards. Such trace still includes an fluent assignment $Init$ that associates a truth-value to each fluent in the initial state.

\begin{definition}[Valid g-LTS traces]
The set of g-LTS traces accepted by a g-LTS $G = (Q,\Sigma,\Phi,\delta,q_{0},C_{0})$ is defined as:
\begin{align*}
\{~(Init, s) \mid \exists~(Init, w) \mbox{~a valid execution of $G$ such that~} s = w|_{\Sigma}~\}
\end{align*}
where $w|_{\Sigma}$ denotes the projection of $w$ over the alphabet $\Sigma$ (see Section \ref{section:background-state-machines}).
\label{definition:valid-glts-traces}
\end{definition}

In other words, the set of valid g-LTS traces corresponds to valid executions where $\tau$ and guards have been removed. 

The traces considered in the definition above are g-LTS traces, that is, they explicitly define an initial fluent assignment \emph{Init}. The set of pure event traces can be defined through existential quantification over such a fluent assignment:

\begin{definition}[Trace semantics of g-LTS]
The set of pure event traces accepted by a g-LTS $G$ is defined as:
\begin{align*}
\{~ s \mid \exists~Init \mbox{~such that~} (Init,s) \mbox{~is a valid g-LTS trace for G}~\}
\end{align*}
\label{definition:glts-trace-semantics}
\end{definition}

In the definitions above, g-LTS traces are used as a convenient milestone between g-LTS executions and pure event traces. Indeed, the explicit initial fluent assignment $Init$ proves useful for user feedback when model-checking g-hMSC and g-LTS (see Section \ref{section:tool-model-checker}). 

The next section provides g-LTS with hiding and composition operators similar to those defined on LTS in Section \ref{section:background-state-machines}.

\subsection{Hiding and composition in the presence of guards\label{subsection:glts-operators}}

The hiding operator on g-LTS is very similar to the one on LTS defined in Section \ref{subsection:lts-hiding}. 

\begin{definition}[g-LTS hiding]
The \emph{hiding} of a set of labels $L$ in a g-LTS $G = (Q,\Sigma,\Phi,\delta,q_{0},C_{0})$ yields the g-LTS
\begin{equation*}
G \setminus L = (Q,\Sigma \setminus L,\Phi,\delta_{hidden},q_{0},C_0)
\end{equation*}
\noindent where $\delta_{hidden}$ is the smallest relation satisfying the following rules:
\begin{center}
\begin{tabular}{cc}
$\frac{\displaystyle G \stackrel{l}{\longrightarrow} G'}{\displaystyle G \setminus L \stackrel{l}{\longrightarrow} G' \setminus L}~~l \notin L$ & 
$\frac{\displaystyle G \stackrel{l}{\longrightarrow} G'}{\displaystyle G \setminus L \stackrel{\tau}{\longrightarrow} G' \setminus L}~~l \in L$ \\
\end{tabular}
\end{center}
where $G \stackrel{l}{\longrightarrow} G'$ denotes that the g-LTS $G = (Q,\Sigma,\Phi,\delta,q_{0},C_{0})$ may transits into a g-LTS $G' = (Q,\Sigma,\Phi,\delta,q_{1},C_{0})$ through a transition labeled $l$ from its initial state, provided that $(q_0,l,q_1) \in \delta$. 
\label{definition:glts-hiding}
\end{definition}

As with LTS, the hiding operator makes a set of labels invisible by replacing them by $\tau$ transitions. Note that the definition here allows hiding both guards and events. 

The special case where $L = \mathcal{P}(2^\Phi)$ yields a g-LTS with no guard remaining. We will then consider that the hiding operator actually returns the LTS defined as follows:
\begin{equation*}
G \setminus \mathcal{P}(2^\Phi) = (Q,\Sigma,\delta_{hidden},q_0)
\end{equation*}

\begin{definition}[g-LTS composition]
Let $G = (Q_1,\Sigma_1,\Phi_1,\delta_1,q_{1},C_{1})$ and $H = (Q_2,\Sigma_2,\Phi_2,\delta_2,q_{2},C_{2})$ denote two g-LTS. Their \emph{composition} is another g-LTS, namely,
\begin{equation*}
G \parallel H = (S_1 \times S_2,\Sigma_1\cup\Sigma_2,\Phi_1\cup\Phi_2,\delta,(q_1,q_2),C_1 \wedge C_2)
\end{equation*}
\noindent where $\delta$ is the smallest relation satisfying the following rules:

\centering
\begin{tabular}{cl}
$\frac{\displaystyle G \stackrel{l}{\longrightarrow} G'}{\displaystyle G \parallel H \stackrel{l}{\longrightarrow} G' \parallel H}$ & $l \notin \mathcal{P}(2^\Phi)$, $l \notin \Sigma_2$ \\[20pt]

$\frac{\displaystyle H \stackrel{l}{\longrightarrow} H'}{\displaystyle G \parallel H \stackrel{l}{\longrightarrow} G \parallel H'}$ & $l \notin \mathcal{P}(2^\Phi)$, $l \notin \Sigma_1$ \\[20pt]

$\frac{\displaystyle G \stackrel{l}{\longrightarrow} G',~H \stackrel{l}{\longrightarrow} H'}{\displaystyle G \parallel H \stackrel{l}{\longrightarrow} G' \parallel H'}$ & $l \notin \mathcal{P}(2^\Phi)$, $l \neq \tau$ \\[20pt]

$\frac{\displaystyle G \stackrel{g}{\longrightarrow} G',~H \stackrel{h}{\longrightarrow} H'}{\displaystyle G \parallel H \stackrel{g \wedge h}{\longrightarrow} G' \parallel H'}$ & $g \in \mathcal{P}(2^{\Phi_1})$, $h \in \mathcal{P}(2^{\Phi_2})$, $g \wedge h \nvDash false$ 
\end{tabular}
\label{definition:glts-composition}
\end{definition}

The first three rules are the same as the LTS composition ones (see Section \ref{subsection:lts-composition}); here, they explicitly exclude guards. The last rule states that the two LTSs must transit on guards together, provided that the guard conjunction is satisfiable. The resulting transition is then labeled with this conjunction.

\section{From guarded hMSC to pure LTS\label{section:deductive-glts-to-lts}}

This section details algorithms to explicitly capture the traces of a guarded hMSC through a pure LTS. Section \ref{subsection:from-ghmsc-to-glts} provides an algorithm to rewrite a g-hMSC to a g-LTS. Section \ref{subsection:from-glts-to-lts} then shows the set of event traces admitted by a g-LTS can be captured through a LTS.

\subsection{From guarded hMSC to guarded LTS\label{subsection:from-ghmsc-to-glts}}

A guarded hMSC can be rewritten as a g-LTS. The latter abstracts from the agents and captures the set of global behaviors covered by the g-hMSC. The rewriting algorithm extends the technique presented in Section \ref{subsection:background-hmsc} that synthesizes a LTS capturing the traces of hMSC \cite{Uchitel:2004}. The algorithm may
be outlined as detailed below. It is illustrated in Fig.~\ref{image:scheduler-ghmsc-glts} for the guarded hMSC in Fig.~\ref{image:scheduler-ghmsc}.

\noindent \textbf{Handling nodes} -- Every hMSC node yields a behaviorally equivalent sub-LTS.
\begin{itemize}
\item A MSC node is rewritten as a sub-LTS collecting the linear event sequences from the scenario. For a MSC $M$, the LTS captures the set of traces $\mathcal{L}_{total}(M)$ defined in Section \ref{subsection:background-positive-scenarios}.
\item For a node expanded into a finer-grained hMSC, the procedure is applied recursively to obtain the corresponding sub-LTS.
\item A decision node is rewritten as a sub-LTS having only one state and no event; the same applies to the start and end nodes of the hMSC.
\end{itemize}
In the first two cases, the $Task_{start}$ and $Task_{end}$ special events are added to the corresponding sub-LTS.

\noindent \textbf{Handling edges} -- The edges in a guarded hMSC yield transitions between the terminal and initial states of the sub-LTS corresponding to their source and target nodes, respectively.
\begin{itemize}
\item An outgoing edge of a decision node is labeled by a guard and yields a guarded transition in the g-LTS.
\item Any other edge is simply converted as a $\tau$ transition.
\end{itemize}

\begin{figure}[H]\centering
\scalebox{0.57}{
  \includegraphics[trim=2mm 2mm 2mm 2mm, clip]{src/3-deductive/images/ghmsc-glts-scheduler}
}
\caption{Rewriting a g-hMSC as a g-LTS, on the meeting scheduler process in Fig.~\ref{image:scheduler-ghmsc}.\label{image:scheduler-ghmsc-glts}}
\end{figure}

Obtained g-LTS can be further optimized by coalescing states separated by $\tau$ transitions. Doing so reduces the g-LTS state space, speeding up the synthesis of a trace-equivalent LTS and thus the model-checking procedure described in Section \ref{section:tool-model-checker}. However, doing so also hurts the traceability between g-hMSC tasks and g-LTS states; such traceability is useful in practice for providing users with feedback on the process model itself (see Section \ref{section:tool-clinical-pathway-analyzer}). 

\subsection{From guarded LTS to pure LTS\label{subsection:from-glts-to-lts}}

The set of traces accepted by a g-LTS may be captured by building a trace-equivalent LTS. For this, a parallel composition of g-LTSs if first computed so as to meet the various acceptance conditions in Section \ref{subsection:glts-trace-semantics}. The first g-LTS in this composition is a super g-LTS meeting the \emph{trace inclusion} and \emph{admissible start} condition. To meet the \emph{guard satisfaction} condition, the set of traces of the super g-LTS is pruned further by composing it with fluent automata. Let us make each of them in the composition further precise.

\noindent \textbf{Super g-LTS} -- By definition, the input g-LTS already meets the \emph{trace inclusion} condition. In order to meet the \emph{admissible start} condition, it is extended by converting the initial condition $C_0$ as an explicit guard from the initial state. 

Let denote the input g-LTS by $G = (Q,\Sigma,\Phi,\delta,q_{0},C_{0})$; the super LTS is defined as:
\begin{equation*}
Super~LTS = (Q \cup \{ q_{start} \}, \Sigma, \Phi, \delta \cup \{(q_{start},C_0,q_0)\},q_{start},true)
\end{equation*}

\noindent \textbf{Fluent g-LTS} -- The \emph{guard satisfaction} condition is enforced by pruning all traces violating guards in the super g-LTS. For this we compose the latter with fluent automata. The latter keep track of the current fluent values; guard events are constrained to happen only when the corresponding guard is true.

A fluent $Fl = \textless Init_{Fl}, Term_{Fl} \textgreater $ yields a g-LTS $(Q,\Sigma,\Phi,\delta,q_{0},C_{0})$ where
\begin{align*}
Q      &= \{q_u,q_t,q_f\}            \\
\Sigma &= Init_{Fl} \cup Term_{Fl}   \\
\Phi   &= \{ Fl \} \\
\delta &=    \{~(q_f,e,q_t) \mid e \in Init_{Fl}~\}~\cup \{~(q_t,e,q_t) \mid e \in Init_{Fl}~\} \\
       &\cup~\{~(q_t,e,q_f) \mid e \in Term_{Fl}~\} \cup \{~(q_f,e,q_f) \mid e \in Term_{Fl}~\} \\
       &\cup~\{~(q_u, [Fl], q_t),~(q_u, [\neg Fl], q_f)~\} \\
       &\cup~\{~(q_t, [Fl], q_t),~(q_f, [\neg Fl], q_f)~\} \\
q_0    &= q_u \\
C_0    &= true
\end{align*}

This resulting g-LTS is illustrated in Fig.~\ref{image:fluent-glts} for a generic fluent definition. A transition labeled with a set name between $\{$ and $\}$ brackets is a shortcut denoting one transition for each element of the set.

\begin{figure}[H]\centering
\scalebox{0.75}{
  \includegraphics[trim=2mm 2mm 2mm 2mm, clip]{src/3-deductive/images/fluent-glts}
}
\caption{A generic fluent g-LTS\label{image:fluent-glts}}
\end{figure}

For example, consider the following fluent below for the meeting scheduler exemplar. The corresponding fluent automaton is shown in Fig.~\ref{image:second-cycle-fluent-glts}.
\begin{center}
fluent $second\_cycle = \textless \{ ExtendDateRange_{end}, \newline WeakenConstraints_{end} \},
 \{ InitiateMeeting_{end} \} \textgreater $\\
\end{center}

\begin{figure}[H]\centering
\scalebox{0.75}{
  \includegraphics[trim=2mm 2mm 2mm 2mm, clip]{src/3-deductive/images/second-cycle-fluent-glts}
}
\caption{Fluent g-LTS for $second\_cycle$\label{image:second-cycle-fluent-glts}}
\end{figure}

\noindent \textbf{Synthesized LTS} -- Putting pieces together, the trace-equivalent LTS of a g-LTS is obtained through the following computation:
\begin{align*}
\left((Super~LTS \parallel G_{Fl_1} \parallel \ldots \parallel G_{Fl_n}) \setminus \mathcal{P}(2^\Phi)\right)^\Delta
\end{align*}
\noindent where $G_{Fl_i}$ denotes the fluent automaton of the i-th fluent.

That is, a g-LTS is first obtained through the composition of the Super LTS with fluent automata; all guards are then hidden, resulting in a pure LTS which can be further minimized.

Figure \ref{image:scheduler-lts} shows the equivalent LTS obtained on the meeting scheduler process of Fig.~\ref{image:scheduler-ghmsc}. Task refinements are not entirely unfolded; only events used in fluents definitions have been kept. The \artifact{no\_remaining\_solution} event, for example, denotes an initiating event of the \artifact{date\_conflict} fluent and belongs to the refinement of the \artifact{AcquireConstraints} tasks. The events \artifact{click\_extend} and \artifact{click\_weaken} suggest buttons allowing the meeting initiator to select resolution heuristics when a conflict occurs. The appear in the definition of the fluent \artifact{resolve\_by\_weakening}. A transition labeled with `$\ldots$' denotes a sequence of events hidden for readability.  

Unlike the original g-hMSC and its g-LTS rewriting, this LTS does not contain any loop. This happens because the $second\_cycle$ fluent prevents looping in meeting arbitrations indefinitely, as required by soft goals. In other words, in the example at hand the technique unfolds the process models as three possible scenarios. In the right most one, no date conflict arises and the meeting is scheduled immediately after having acquired the constraints. In the middle one, a conflict occurs and is resolved by weakening the participant constraints. In the third one, the resolution consists in extending the date range and collecting constraints one again. In both cases, even if the absence of an optimal solution triggers a new date conflict, the meeting is scheduled. 

\begin{figure}\centering
\scalebox{0.16}{
  \includegraphics[trim=2mm 2mm 2mm 2mm, clip]{src/3-deductive/images/scheduler-lts}
}
\caption{Trace-equivalent LTS for the scheduler process in Fig.~\ref{image:scheduler-ghmsc}.\label{image:scheduler-lts}}
\end{figure}

Note that the composition technique described always considers $2^{\mid\Phi\mid}$ initial fluent assignments, pruning those that do not satisfy $C_0$. In practice, certain fluents have a specific initial value; in this case, the corresponding fluent automaton can be optimized by removing a transition from its initial state. This allows to avoid the computation of some of the unsatisfiable conjunctions on the global initial state. The approach might even be updated so as to consider only fluent initializations satisfying $C_0$. If such optimization speeds up the computation in practice, the exponential blow of the resulting trace LTS is unavoidable. It naturally results from the ability of models with guards to cover numerous behaviors in an implicit, compact way. 


\section*{Summary}

This chapter has shown how state machines can be derived from processes in the guarded hMSC modeling language. The main aim of such transformation is to make such processes amenable to analysis such as model-checking. For this, a trace-based semantics for guarded hMSC has been defined. This semantics is operationalized through labeled transition systems. A guarded flavor of LTS has been introduced as a powerful intermediate form in the transformation from guarded hMSC to pure LTS. Indeed, guarded LTS are a structured form of LTS avoiding state explosion and enabling efficient symbolic analyses of the corresponding process model (e.g.~\cite{Damas:2011}). Synthesis algorithms from guarded hMSC to guarded LTS to LTS have been discussed.
