\chapter{Derivation of State Machine Models from Process Models\label{chapter:deductive}}

This chapter shows how state machines can be synthesised from process models. They capture the trace semantics of process models, making them amenable to formal analysis such as model-checking.

\section{Motivation}

\section{Introducing guarded transition systems}

This section introduces guarded LTS (g-LTS) as an intermediate formalism between guarded hMSCs and LTS. Roughly, a g-LTS is a transition system with guards or events on transitions. It provides a convenient milestone on the way from a guarded hMSC to the corresponding LTS, in particular, for determining the set of traces accepted by the guarded hMSC. As a structured form of LTS, a g-LTS representation avoids state explosion. It is easier to understand and facilitates code generation. Moreover, interresting analyses may be performed at g-LTS level, see \cite{Damas:2011}.

\begin{definition}[Guarded LTS]
\noindent A guarded LTS is defined as a structure $(Q,\Sigma,\Phi,\delta,q_{0},C_{0})$ where 
\begin{itemize}
\item $Q$ is a finite set of states,
\item $\Sigma$ is a set of event labels, 
\item $\Phi$ is a set of fluents defined on $\Sigma$,
\item $\delta$ is a transition relation $Q \times \Sigma_{\tau}\cup\mathcal{P}(2^\Phi) \times Q$,
\item $q_{0} \in Q$ is the initial state,
\item $C_{0} \in \mathcal{P}(2^\Phi)$ is an initial condition. 
\end{itemize}
\end{definition}

In a guarded LTS, transitions are labeled either by a guard or by an event. $\tau$ transitions are also allowed and are used in a similar way than in simple LTS. 

A guard is a propositional formula over fluents in $\Phi$. The set of such formula is denoted by $\mathcal{P}(2^\Phi)$, as in Section \ref{section:background-fluents}. Intuitively, the guard must be evaluated to true for its transition to be activated.

The condition $C_0$ plays the same role as in guarded hMSC. It is a boolean formula on fluents that constraints their initial values.

\subsection{Trace semantics of guarded LTSs} 

The semantics of g-LTS is defined in terms of event traces involving no guards at all. 

\begin{definition}[g-LTS execution]
An \emph{execution} of a g-LTS $G = (Q,\Sigma,\Phi,\\\delta,q_{0},C_{0})$ from $q_0$ is a pair $(Init, \textless l_0, \ldots \textgreater)$, where 
\begin{itemize}
\item $Init$ is an initial fluent value assignment, mapping every fluent in $\Phi$ to true or false,
\item $\textless l_0, \ldots \textgreater$ is an finite sequence of labels $l_i \in \Sigma_{\tau}\cup\mathcal{P}(2^\Phi)$, some of them being events (including $\tau$) and others being guards.
\end{itemize}
\end{definition}

Only certain executions are considered valid from the initial state of the g-LTS. This is captured by the following definition:

\begin{definition}[Valid g-LTS execution]
An execution $S = (Init,\textless l_0,\ldots \textgreater)$ of a g-LTS $G = (Q,\Sigma,\Phi,\delta,q_{0},C_{0})$ is \emph{valid} from its initial state $q_0$ iif the following \emph{acceptance conditions} are met for every $i$:\\
\indent trace inclusion: $\exists q_{i+1} \in Q$ such that $(q_i,l_i,q_{i+1}) \in \delta$\\
\indent admissible start: $Init \models C_0$ \\
\indent guard satisfaction: $S_i \models l_i$ if $l_i \in \mathcal{P}(2^\Phi)$\\
where $S_i$ is the fluent value assignment after the i-th event in the trace, with $S_0 = Init$
\end{definition}

The first condition states that the label sequence is accepted by the automaton. The second condition states that the initial fluent value assignment must meet the initial condition $C_0$. The third condition ensures that all guards are met along the sequence.

The trace semantics of a g-LTS is defined as a set of event traces. Roughly, it consists of all valid executions where $\tau$ labels as well as guards have been removed. This is precisely captured by the the following definition.

\begin{definition}[g-LTS trace semantics]
The set of traces accepted by a g-LTS $G = (Q,\Sigma,\Phi,\delta,q_{0},C_{0})$ is defined as:
\begin{align*}
\mathcal{L}(G) &= \{~w|_{\Sigma} \mid w~is~a~valid~execution~for~G~\}
\end{align*}

\end{definition}
\subsection{Hiding and composition of guarded LTSs}

\section{From guarded hMSC to pure LTS}

\subsection{From guarded hMSC to guarded LTS}
\subsection{From guarded LTS to pure LTS}
