\section{Introducing guarded transition systems\label{section:deductive-glts}}

This section introduces guarded LTS (g-LTS) as an intermediate formalism between guarded hMSCs and LTS. It provides a convenient milestone on the way from a guarded hMSC to the corresponding LTS, in particular, for determining the set of traces accepted by the guarded hMSC. 

\begin{definition}[Guarded LTS]
\noindent A guarded LTS is defined as a structure $(Q,\Sigma,\Phi,\delta,q_{0},C_{0})$ where 
\begin{itemize}
\item $Q$ is a finite set of states,
\item $\Sigma$ is a set of event labels, 
\item $\Phi$ is a set of fluents defined on $\Sigma$,
\item $\delta$ is a transition relation $Q \times \Sigma_{\tau}\cup\mathcal{P}(2^\Phi) \times Q$,
\item $q_{0} \in Q$ is the initial state,
\item $C_{0} \in \mathcal{P}(2^\Phi)$ is an initial condition. 
\end{itemize}
\end{definition}

In a guarded LTS, transitions are labeled either by a guard or by an event. Non-observable $\tau$-transitions are also allowed. 

A guard is a propositional formula over fluents in $\Phi$. The set of such formula is denoted by $\mathcal{P}(2^\Phi)$, as in Section \ref{section:background-fluents}. Intuitively, the guard must be evaluated to true for its transition to be activated. This is made precise is the next section.

The condition $C_0$ plays the same role as in guarded hMSCs. It is a boolean formula that constraints the acceptable initial values of fluents. 

\subsection{Trace semantics\label{subsection:glts-trace-semantics}} 

The semantics of g-LTS is defined in terms of event traces involving no guards at all.

\begin{definition}[g-LTS execution]
An \emph{execution} of a g-LTS $G = (Q,\Sigma,\Phi,\\\delta,q_{0},C_{0})$ from $q_0$ is a pair $(Init, \textless l_0, \ldots,l_n\textgreater)$, where 
\begin{itemize}
\item $Init \in 2^\Phi$ is an initial fluent value assignment, mapping every fluent in $\Phi$ to true or false,
\item $\textless l_0,\ldots,l_n \textgreater$ is an finite sequence of labels $l_i \in \Sigma_{\tau}\cup\mathcal{P}(2^\Phi)$, some of them being event labels (including $\tau$) and others being guards.
\end{itemize}
\end{definition}

Only certain executions are considered valid from the initial state of the g-LTS. This is captured by the following definition:

\begin{definition}[Valid g-LTS execution]
An execution $S = (Init,\textless l_0,\ldots,l_n\textgreater)$ of a g-LTS $G = (Q,\Sigma,\Phi,\delta,q_{0},C_{0})$ is \emph{valid} from its initial state $q_0$ iif the following \emph{acceptance conditions} are met for every $i$ such that $0 \leqslant i < n$:\\
\vspace{-0.8cm}
\begin{tabbing}
\indent trace inclusion:~~~~~~~\= $\exists q_{i+1} \in Q$ such that $(q_i,l_i,q_{i+1}) \in \delta$\\
\indent admissible start:      \> $Init \models C_0$ \\
\indent guard satisfaction:    \> $S_i \models l_i$ if $l_i \in \mathcal{P}(2^\Phi)$\\
\end{tabbing}
\vspace{-0.8cm}
where $S_i$ is the fluent value assignment after the i-th event in the trace, with $S_0 = Init$.
\end{definition}

\noindent A few remarks are in order here about the three acceptance conditions:
\begin{description}
\item[trace inclusion] states that the label sequence denotes an existing path in the automaton. Note that the condition $(q_i,l_i,q_{i+1}) \in \delta$ is such that guard labels are not considered syntactically, but semantically. This is because guards are mathematically defined as sets of fluent assignments, that is, as elements of $\mathcal{P}(2^\Phi)$.
\item[admissible start] states that the initial fluent value assignment must meet the initial condition $C_0$.
\item[guard satisfaction] ensures that all guards are met along the sequence. The fluent value assignment $S_i$ follows from fluent definitions, as explained in Section \ref{subsection:background-fluents-single-traces}. The presence of guards in the label sequence can simply be ignored.
\end{description}

The trace semantics of a g-LTS is defined as a set of event traces. Roughly, it consists of all valid executions where $\tau$ labels as well as guards have been removed. This is precisely captured by the the following definition.

\begin{definition}[g-LTS trace semantics]
The set of traces accepted by a g-LTS $G = (Q,\Sigma,\Phi,\delta,q_{0},C_{0})$ is defined as:
\begin{align*}
\mathcal{L}(G) &= \{~w|_{\Sigma} \mid \exists~Init \in 2^\Phi~s.t.~(Init,w)~is~a~valid~execution~for~G~\}
\end{align*}
\end{definition}

Section \ref{subsection:from-glts-to-lts} provides a composition algorithm to capture this set of traces through a pure LTS, that is, a LTS with event labels only. It relies on g-LTS variants of the LTS hiding and composition operators defined in Section \ref{section:background-state-machines}. These two operators are introduced in the next section.

\subsection{Hiding and composition in presence of guards}

The hiding operator on g-LTSs is very similar to the one on LTSs, defined in Section \ref{subsection:lts-hiding}. 

\begin{definition}[g-LTS hiding]
The \emph{hiding} of a set of labels $I$ in a g-LTS $G = (Q,\Sigma,\Phi,\delta,q_{0},C_{0})$ defines the g-LTS
\begin{equation*}
G \setminus I = (Q,\Sigma \setminus I,\Phi,\delta_{hidden},q_{0},C_0)
\end{equation*}
\noindent where $\delta_{hidden}$ is the smallest relation satisfying the following rules:
\begin{center}
\begin{tabular}{cc}
$\frac{\displaystyle G \stackrel{l}{\longrightarrow} G'}{\displaystyle G \setminus I \stackrel{l}{\longrightarrow} G' \setminus I}~~l \notin I$ & 
$\frac{\displaystyle G \stackrel{l}{\longrightarrow} G'}{\displaystyle G \setminus I \stackrel{\tau}{\longrightarrow} G' \setminus I}~~l \in I$ \\
\end{tabular}
\end{center}
\end{definition}

As with LTSs, the hiding operator makes a set of labels invisible by replacing them by $\tau$ transitions. Note that the definition above allows hiding both guards and events. 

The special case where $I = \mathcal{P}(2^\Phi)$ yields a g-LTS with no guard remaining. In this case, we will consider that the hiding operator actually returns a LTS defined as defined below, where $\delta_{hidden}$ is as given above:
\begin{equation*}
G \setminus \mathcal{P}(2^\Phi) = (Q,\Sigma,\delta_{hidden},q_0)
\end{equation*}

\noindent We also define a composition operator on g-LTS, as follows:

\begin{definition}[g-LTS composition]
Let $G = (Q_1,\Sigma_1,\Phi_1,\delta_1,q_{1},C_{1})$ and $H = (Q_2,\Sigma_2,\Phi_2,\delta_2,q_{2},C_{2})$ denote two g-LTS. Their \emph{composition} is another g-LTS 
\begin{equation*}
G \parallel H = (S_1 \times S_2,\Sigma_1\cup\Sigma_2,\Phi_1\cup\Phi_2,\delta,(q_1,q_2),C_1 \wedge C_2)
\end{equation*}
\noindent where $\delta$ is the smallest relation satisfying the following rules:

\centering
\begin{tabular}{cl}
$\frac{\displaystyle G \stackrel{l}{\longrightarrow} G'}{\displaystyle G \parallel H \stackrel{l}{\longrightarrow} G' \parallel H}$ & $l \notin \mathcal{P}(2^\Phi)$, $l \notin \Sigma_2$ \\[20pt]

$\frac{\displaystyle H \stackrel{l}{\longrightarrow} H'}{\displaystyle G \parallel H \stackrel{l}{\longrightarrow} G \parallel H'}$ & $l \notin \mathcal{P}(2^\Phi)$, $l \notin \Sigma_1$ \\[20pt]

$\frac{\displaystyle G \stackrel{l}{\longrightarrow} G',~H \stackrel{l}{\longrightarrow} H'}{\displaystyle G \parallel H \stackrel{l}{\longrightarrow} G' \parallel H'}$ & $l \notin \mathcal{P}(2^\Phi)$, $l \neq \tau$ \\[20pt]

$\frac{\displaystyle G \stackrel{g}{\longrightarrow} G',~H \stackrel{h}{\longrightarrow} H'}{\displaystyle G \parallel H \stackrel{g \wedge h}{\longrightarrow} G' \parallel H'}$ & $g \in \mathcal{P}(2^{\Phi_1})$, $h \in \mathcal{P}(2^{\Phi_2})$, $g \wedge h \nvDash false$ 
\end{tabular}
\end{definition}

The first three rules are the same as with LTS composition (see Section \ref{subsection:lts-composition}); they explicitly exclude guards here. The last rule states that the two LTSs must transit on guards together, provided that their conjunction is satisfiable. The resulting transition is then labeled with this conjunction.

Note that this composition operator is not intended to provide a counterpart to LTS composition for agents whose behaviors would be modeled with g-LTSs. Among others, forcing agents to synchronize together on guards is arbitrary. It is mainly introduced here for the specific case of the composition algorithm given in the next section.
