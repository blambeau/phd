\section{Introducing guarded transition systems\label{section:deductive-glts}}

This section introduces guarded LTS (g-LTS) as an intermediate formalism between guarded hMSCs and LTS. This formalism provides a convenient milestone on the way from a guarded hMSC to the corresponding LTS, in particular, for determining the set of traces accepted by the guarded hMSC. 

\begin{definition}[Guarded LTS]
\noindent A guarded LTS is defined as a structure $(Q,\Sigma,\Phi,\delta,q_{0},C_{0})$ where 
\begin{itemize}
\item $Q$ is a finite set of states,
\item $\Sigma$ is a set of event labels, 
\item $\Phi$ is a set of fluents defined on $\Sigma$,
\item $\delta$ is a transition relation $Q \times L \times Q$,
\item $q_{0} \in Q$ is the initial state,
\item $C_{0} \in \mathcal{P}(2^\Phi)$ is an initial condition on fluents.
\end{itemize}
where the set of transition labels is defined as $L = \Sigma \cup \{\tau\} \cup \mathcal{P}(2^\Phi)$ 
\end{definition}

In a guarded LTS, transitions are labeled either by a guard, an event, or $\tau$. As with LTS, $\tau$-transitions are non-observable from the environment (see Section \ref{section:background-state-machines}). In the sequel, $\Sigma_{\tau}$ will denote $\Sigma \cup \{\tau\}$.

A guard is a propositional formula over fluents in $\Phi$. The set of such formula is denoted by $\mathcal{P}(2^\Phi)$, as in Section \ref{section:background-fluents}. Intuitively, the guard must evaluate to true for its transition to be activated. This is made more precise is the next section.

The condition $C_0$ plays the same role as in guarded hMSCs. It is a Boolean formula that constrains the acceptable initial values of fluents. 

\subsection{Trace semantics\label{subsection:glts-trace-semantics}} 

The semantics of g-LTS is defined in terms of event traces involving no guards at all.

\begin{definition}[g-LTS execution]
An \emph{execution} of a g-LTS $G = (Q,\Sigma,\Phi,\\\delta,q_{0},C_{0})$ from $q_0$ is a pair $(Init, \textless l_0, \ldots,l_n\textgreater)$, where 
\begin{itemize}
\item $Init \in 2^\Phi$ is an initial fluent value assignment mapping every fluent in $\Phi$ to true or false,
\item $\textless l_0,\ldots,l_n \textgreater$ is an finite sequence of labels $l_i \in \Sigma_{\tau}\cup\mathcal{P}(2^\Phi)$, each of them being either an event label, $\tau$, or a guard.
\end{itemize}
\end{definition}

Due to guards, only certain executions are to be considered valid from the initial state of the g-LTS. This is captured by the following definition.

\begin{definition}[Valid g-LTS execution]
An execution $S = (Init,\textless l_0,\ldots,\\l_n \textgreater)$ of a g-LTS $(Q,\Sigma,\Phi,\delta,q_{0},C_{0})$ is \emph{valid} from its initial state $q_0$ iff the following \emph{acceptance conditions} are met for every $i$ $(0 \leqslant i < n)$:\\
\vspace{-0.8cm}
\begin{tabbing}
\indent trace inclusion:~~~~~~~\= $\exists q_{i+1} \in Q$ such that $(q_i,l_i,q_{i+1}) \in \delta$\\
\indent admissible start:      \> $Init \models C_0$ \\
\indent guard satisfaction:    \> $S_i \models l_i$ if $l_i \in \mathcal{P}(2^\Phi)$\\
\end{tabbing}
\vspace{-0.8cm}
where $S_i$ is the fluent value assignment after the i-th event in the trace, with $S_0 = Init$.
\label{definition:valid-glts-execution}
\end{definition}

\begin{description}
\item[The trace inclusion condition] states that the label sequence must denote an existing path in the automaton.
\item[The admissible start condition] states that the initial fluent value assignment must meet the initial condition $C_0$.
\item[The guard satisfaction condition] ensures that all guards are met along the sequence. The fluent value assignment $S_i$ is determined by fluent definitions, as explained in Section \ref{subsection:background-fluents-single-traces}. 
\end{description}

As with LTS, the trace semantics of g-LTS is defined in terms of accepted \emph{event} traces. Roughly, it consists of all valid executions where $\tau$ labels as well as guards are removed. This is precisely captured by the the following definitions.

\begin{definition}[g-LTS trace]
A g-LTS trace is a pair $(Init,\textless e_0,\ldots,e_n\textgreater)$ where 
\begin{itemize}
\item $Init \in 2^\Phi$ is an initial fluent value assignment mapping every fluent in $\Phi$ to true or false,
\item $\textless e_0,\ldots,e_n \textgreater$ is an finite sequence of event labels, with $e_i \in \Sigma$.
\end{itemize}
\end{definition}

Unlike an execution, a g-LTS trace only contains event labels, but no $\tau$ nor guards. Such trace still includes an fluent assignment $Init$ that associates a truth-value to each fluent in the initial state.

\begin{definition}[Valid g-LTS traces]
The set of g-LTS traces accepted by a g-LTS $G = (Q,\Sigma,\Phi,\delta,q_{0},C_{0})$ is defined as:
\begin{align*}
\{~(Init, s) \mid \exists~(Init, w) \mbox{~a valid execution of $G$ such that~} s = w|_{\Sigma}~\}
\end{align*}
where $w|_{\Sigma}$ denotes the projection of $w$ over the alphabet $\Sigma$ (see Section \ref{section:background-state-machines}).
\end{definition}

In other words, the set of valid g-LTS traces corresponds to valid executions where $\tau$ and guards have been removed. 

The traces considered in the definition above are g-LTS traces, that is, they explicitely define an initial fluent assignment \emph{Init}. The set of pure event traces can be defined through existential quatification over such a fluent assignment:

\begin{definition}[Trace semantics of g-LTS]
The set of pure event traces accepted by a g-LTS $G$ is defined as:
\begin{align*}
\{~ s \mid \exists~Init \mbox{~such that~} (Init,s) \mbox{~is a valid g-LTS trace for G}~\}
\end{align*}
\label{definition:glts-trace-semantics}
\end{definition}

In the definitions above, g-LTS traces are used as a convenient milestone between g-LTS executions and pure event traces. Indeed, the explicit initial fluent assignment $Init$ proves useful for user feedback when model-checking g-hMSC and g-LTS (see Section \ref{section:tool-model-checker}). 

The next section provides g-LTS with hiding and composition operators similar to those defined on LTS in Section \ref{section:background-state-machines}.

\subsection{Hiding and composition in the presence of guards\label{subsection:glts-operators}}

The hiding operator on g-LTS is very similar to the one on LTS defined in Section \ref{subsection:lts-hiding}. 

\begin{definition}[g-LTS hiding]
The \emph{hiding} of a set of labels $L$ in a g-LTS $G = (Q,\Sigma,\Phi,\delta,q_{0},C_{0})$ yields the g-LTS
\begin{equation*}
G \setminus L = (Q,\Sigma \setminus L,\Phi,\delta_{hidden},q_{0},C_0)
\end{equation*}
\noindent where $\delta_{hidden}$ is the smallest relation satisfying the following rules:
\begin{center}
\begin{tabular}{cc}
$\frac{\displaystyle G \stackrel{l}{\longrightarrow} G'}{\displaystyle G \setminus L \stackrel{l}{\longrightarrow} G' \setminus L}~~l \notin L$ & 
$\frac{\displaystyle G \stackrel{l}{\longrightarrow} G'}{\displaystyle G \setminus L \stackrel{\tau}{\longrightarrow} G' \setminus L}~~l \in L$ \\
\end{tabular}
\end{center}
where $G \stackrel{l}{\longrightarrow} G'$ denotes that the g-LTS $G = (Q,\Sigma,\Phi,\delta,q_{0},C_{0})$ may transits into a g-LTS $G' = (Q,\Sigma,\Phi,\delta,q_{1},C_{0})$ through a transition labeled $l$ from its initial state, provided that $(q_0,l,q_1) \in \delta$. 
\label{definition:glts-hiding}
\end{definition}

As with LTS, the hiding operator makes a set of labels invisible by replacing them by $\tau$ transitions. Note that the definition here allows hiding both guards and events. 

The special case where $L = \mathcal{P}(2^\Phi)$ yields a g-LTS with no guard remaining. We will then consider that the hiding operator actually returns the LTS defined as follows:
\begin{equation*}
G \setminus \mathcal{P}(2^\Phi) = (Q,\Sigma,\delta_{hidden},q_0)
\end{equation*}

\begin{definition}[g-LTS composition]
Let $G = (Q_1,\Sigma_1,\Phi_1,\delta_1,q_{1},C_{1})$ and $H = (Q_2,\Sigma_2,\Phi_2,\delta_2,q_{2},C_{2})$ denote two g-LTS. Their \emph{composition} is another g-LTS, namely,
\begin{equation*}
G \parallel H = (S_1 \times S_2,\Sigma_1\cup\Sigma_2,\Phi_1\cup\Phi_2,\delta,(q_1,q_2),C_1 \wedge C_2)
\end{equation*}
\noindent where $\delta$ is the smallest relation satisfying the following rules:

\centering
\begin{tabular}{cl}
$\frac{\displaystyle G \stackrel{l}{\longrightarrow} G'}{\displaystyle G \parallel H \stackrel{l}{\longrightarrow} G' \parallel H}$ & $l \notin \mathcal{P}(2^\Phi)$, $l \notin \Sigma_2$ \\[20pt]

$\frac{\displaystyle H \stackrel{l}{\longrightarrow} H'}{\displaystyle G \parallel H \stackrel{l}{\longrightarrow} G \parallel H'}$ & $l \notin \mathcal{P}(2^\Phi)$, $l \notin \Sigma_1$ \\[20pt]

$\frac{\displaystyle G \stackrel{l}{\longrightarrow} G',~H \stackrel{l}{\longrightarrow} H'}{\displaystyle G \parallel H \stackrel{l}{\longrightarrow} G' \parallel H'}$ & $l \notin \mathcal{P}(2^\Phi)$, $l \neq \tau$ \\[20pt]

$\frac{\displaystyle G \stackrel{g}{\longrightarrow} G',~H \stackrel{h}{\longrightarrow} H'}{\displaystyle G \parallel H \stackrel{g \wedge h}{\longrightarrow} G' \parallel H'}$ & $g \in \mathcal{P}(2^{\Phi_1})$, $h \in \mathcal{P}(2^{\Phi_2})$, $g \wedge h \nvDash false$ 
\end{tabular}
\label{definition:glts-composition}
\end{definition}

The first three rules are the same as the LTS composition ones (see Section \ref{subsection:lts-composition}); here, they explicitly exclude guards. The last rule states that the two LTSs must transit on guards together, provided that the guard conjunction is satisfiable. The resulting transition is then labeled with this conjunction.
