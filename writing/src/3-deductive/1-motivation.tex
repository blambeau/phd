\section{Motivation\label{section:deductive-motivation}}

Building adequate, complete and consistent process models is not necessarily an easy task. Flawed models can have dreadful consequences in safety-critical areas. The recent usage of process modeling technologies to improve medical safety provides a good example \cite{Clarke:2008, Grando:2009, Damas:2011}. In such areas, models should be as free of errors as possible; techniques should therefore be available for helping building such models, in particular for systematically detecting and fixing severe flaws.

An analysis technique that immediately comes in mind is of course \emph{model checking} \cite{Clarke:1989}. This allows checking that a given safety property is satisfied in all process instances, for example. Other kinds of analysis on guarded hMSCs are discussed in \cite{Damas:2011}; this includes checking the satisfiability of guards; verifying that temporal constraints are met; detecting inaccurate process decisions; and so on.

Verification techniques require providing a formal semantics to the process models. As guarded hMSC are an extension of simple hMSCs, providing them a trace semantics appears natural. Moreover, explitely capturing the set of traces of a g-hMSC through a LTS enables reusing associated tools and techniques \cite{Magee:1999, Giannakopoulou:2003}. These two objectives are achieved in the following sections.

Instead of directly rewriting guarded hMSC as LTS, however, we will use an intermediate level through guarded labeled transition systems (g-LTS). Roughly, a g-LTS is a transition system with guards or events on transitions. It is mainly introduced for providing a structured form of LTS that avoids state explosion. It is therefore easier to understand and facilitates code generation. Moreover, most of the aforementioned analyses may be performed at g-LTS level, see \cite{Damas:2011}.

Section \ref{section:deductive-glts} formally introduce g-LTS and defines their trace semantics. An operational semantics is then given to guarded hMSC in Section \ref{section:deductive-glts-to-lts}, through two rewriting algorithms. The overal approach is illustrated on Fig.~\ref{X}.


