\chapter{Towards an Evaluation Platform for Inductive Model Synthesis\label{chapter:stamina}}

Grammar induction algorithms are commonly evaluated by reporting accuracy measures on the learned model for increasing sizes of the training sample. For example, plots in the previous chapter illustrated the convergence of QSM and ASM towards a ``good'' model when their input becomes rich enough. 

This kind of setup can also be used to compare various induction techniques by evaluating them on common benchmarks of increasing difficulty. Along this line, the Abbadingo contest had a notable impact~\cite{Lang:1998}. A grid of grammar induction challenges unbroken at that time were proposed online. The competition resulted in the discovery of the Blue-Fringe heuristics used by QSM for selecting state pairs to merge (see Section~\ref{BlueFringe}). Since then, Abbadingo has been frequently used as a reusable benchmark and evaluation protocol for grammar induction, e.g., \cite{Lucas:2003, Bongard:2005, Lucas:2005, Adriaans:2006, Dupont:2008, Lambeau:2008, Heule:2010}.

A serious weakness of the Abbadingo benchmark is its restriction to induction problems for automata defined on alphabets of two symbols only. Freezing certain parameters, such as the alphabet size, is necessary for keeping a competition accessible to participants. As a side effect, it limits the generalization of the conclusions and the reusability of the competition protocol. 

This effect has been observed when conducting the evaluations reported in the thesis. On one side, Abbadingo allowed our results to be compared on a sound and agreed benchmarking protocol. On the other side, limiting the evaluations to binary alphabets lacks credibility from a software engineering standpoint. Behavior models are commonly defined on thirty events or more. One might thus reasonably question the representativeness of synthetic machines used in the previous chapter for capturing behavior models.

Extending the Abbadingo protocol itself to take larger alphabets into account appears inappropriate. By design, the generation of learning samples in Abbadingo is independent from the target automaton. This feature inspires from \emph{probably approximately correct} (PAC) learning, a sound theoretical framework from machine learning \cite{Valiant:1984}. However, this independence assumption between that target automaton and the learning sample does not accurately reflect the way samples are obtained from end-user scenarios or automated testing of distributed systems. In particular, generating strings with a distribution independent of the state machine would lead to an overwhelming amount of negative scenarios for typical behavior models. Abbadingo is further discussed in Section~\ref{section:stamina-abbadingo}.

For that reason, an alternative benchmark is proposed here, called Sta-mina (for \emph{State Machine Inference Approaches}). Stamina inspires from Abbadingo but focuses on the complexity of the learning with respect to the alphabet size. Our protocol relies on adapted procedures for generating automata and samples. 

Stamina has initially taken the form of an induction competition, in a similar spirit to Abbadingo. The competition has officially ran between March and December 2010 and was organized in collaboration with the universities of Sheffield and Leicester; it has been officially sponsored by the Engineering and Physical Sciences Research Council\footnote{http://www.epsrc.ac.uk/}. The detailed setup of Stamina is explained in section~\ref{section:stamina-setup}. 

The winning algorithm, DFASAT, as well as a few additional competitors have significantly outperformed the Blue-fringe baseline on a variety of problems. DFASAT mixes SAT solving and state merging and uses a new scoring heuristics that proved especially useful to infer the kind of state machines considered in the competition. Section~\ref{section:stamina-results} presents the main results of the competition and a brief description of the winning technique. 

Further to fostering efforts around the induction problem, the competition raised interest from at least three communities, namely, Machine Learning, Software Engineering and Formal Methods. The Stamina website\footnote{http://stamina.chefbe.net/} is still accessible; it has been updated to serve as an online benchmark and evaluation platform for model synthesis, as discussed in Section~\ref{section:stamina-platform}.

\section{The Abbadingo benchmark\label{section:stamina-abbadingo}}

Abbadingo is designed as a grid of 16 induction problems \cite{Lang:1998}. A competing learner gets a set of training strings labeled as positive or negative by an hidden DFA. It is required to predict the labels that the DFA would assign to a set of unlabeled testing strings. The 16 problems are classified in a grid along two dimensions of difficulty. 
\begin{itemize}
\item The first is the size of the target automata, as measured by the number of states (64, 128, 256 or 512). 
\item The second is the sparsity of the training sample, in terms of four sparsity levels. These levels have been tuned by manually inspecting the learning curves of the Trakhtenbrot-Barzdin algorithm \cite{Trakhtenbrot:1973, Lang:1992}. This algorithm was a state-of-the-art induction algorithm in 1997; the problem grid has been adjusted in such a way that it solves the four problems with the largest samples.
\end{itemize}

The target automata, training sets and test strings of Abbadingo were all drawn from uniform random distributions.
\begin{itemize}
\item Random automata were generated by constructing and minimizing directed graphs of binary degree. Edge and state labels were chosen by flipping a fair coin. 
\item A training set for a target automaton $A$ were made of a random sample drawn without replacement from a uniform distribution over the collection of all binary strings. The average length of those strings were chosen in order to have a good chance of reaching the deepest state of the automaton, a necessary criteria toward a structurally complete sample (see Section~\ref{subsection:gi-background-search-space}). 
\item A testing set were composed of 1800 strings drawn from the remaining strings. Training and test sets did not overlap.
\end{itemize}

The testing protocol of Abbadingo runs as follows. The learner labels each string of the test set and submits this labeling to an online oracle\footnote{available at http://abbadingo.cs.nuim.ie/}. To avoid hill climbing, where the feedback of the competition server could be used by the learner to iteratively optimize a first solution, this oracle only provides a single bit of feedback. The latter tells whether or not the problem is broken. 

A problem is considered broken if the accuracy of the labeling reached at least 99\%. The accuracy is computed as the proportion of the 1800 test strings correctly labeled. During the competition itself, the first participant to break a given problem gained credit for it. Abbadingo allowed multiple winners by defining a partial order on problem difficulty: a problem $A$ is considered harder than a problem $B$ if its DFA has more states \emph{and} its training sample is sparser. 

Two winners, Rodney Price and Hugues Julli\'e, won the competition with similar algorithms relying on what has since been called \emph{evidence-driven state merging} (EDSM). When performing generalization of a sample through state merging (see Chapter~\ref{chapter:inductive-synthesis}), this term captures the strategy of first performing state merges that are supported by the most evidence, according to a specific scoring heuristics. Among other contributions, the competition helped finding the Blue-fringe heuristics to refine which state pairs to merge preferably (see Section~\ref{BlueFringe}).

\subsection{Limitations for evaluating inductive model synthesis}

After the formal competition, Abbadingo evolved into a reusable protocol and online benchmark for induction techniques. The evaluations in Chapter~\ref{chapter:inductive-synthesis} were conducted on a similar protocol. However, Abbadingo sets the size of the alphabet to two symbols only. This limits the relevance of reusing its protocol for evaluating behavior model synthesis techniques. Behavior models are commonly defined on larger sets of events. For instance, the small phone case study studied in the previous chapter uses 16 distinct events for a state machine of only 23 states. Moreover, the automata and samples generated with Abbadingo are notable different from those commonly found in software engineering problems:

\begin{description}
\item[Automata] The automata randomly generated by Abbadingo have a small, quasi-constant, state degree (in number of incident edges). Being defined on a binary alphabet, a canonical automaton of $n$ states has at least $n$ and at most $2n$ edges; these edges are uniformly distributed over all states. 

Automata modeling software systems involve transitions that may be triggered by any of a large number of events. The latter may capture mouse clicks, function names, IO events, etc. The number of outgoing transitions for a given state can be very large and vary significantly from state to state. In the process of designing Stamina, a review of state machines found in the literature has shown the following~\cite{Walkinshaw:2008}: 
\begin{itemize}
\item most states have in- and out-degrees of one or two transitions,
\item a few states have a high in-degree, e.g. those modeling exception handling,
\item a few states have a high out-degree, e.g. an \emph{idle} state where a software agent waits for external stimuli in terms of input events,
\item a few states are \emph{sink} accepting states, that is, they have an out-degree of 0, e.g., explicitly modeling the ability of a system to halt.
\end{itemize}

The automata generated with Abbadingo do not fit such characteristics, due to the small alphabet size. Moreover, because of the uniform edge distribution implied, the protocol would not naturally lead to automata presenting aforementioned characteristics, even if adapted for handling larger alphabets.

\item[Samples] Abbadingo was inspired from the PAC framework for generating samples \cite{Valiant:1984}. In this setting the distribution of the observed strings is external to the target machine. Samples are therefore simply drawn from uniform random distribution over the collection of binary strings up to a some prescribed length. The target machine is only used to classify them as positive or negative. The respective number of positive and negative strings is naturally balanced.

This procedure is no longer adequate when working with larger alphabets and state machines presenting the characteristics previously described. Because of the small average out-degree of states, an overwhelming majority of random sequences from $\Sigma^*$ would be negative strings. Moreover, the few positive strings that would be made available would probably not provide a sufficient coverage of the target machine for obtaining good induction results (see Section~\ref{section:inductive-background}).

For behavior models, a \emph{generative} procedure seems more adequate, where the target machine itself is used to generate the learning sample. This is representative of the way such samples are obtained in the software engineering research: end-users scenarios come from a mental model of the software that is, a variant of the target machine~\cite{Combefis:2011}; execution traces are sometimes obtained through software testing and monitoring; and so on.

\item[Scoring] The accuracy measure is defined in Abbadingo as the proportion of test strings that are correctly classified by the learned automaton. This choice is questionable if one relaxes the assumption of balanced test samples~\cite{Walkinshaw:2008}. In the extreme case of a test set largely overwhelmed by negative strings, for example, a learner classifying all strings as negative would still obtain a comfortable score. A better measure should consider the acceptance of positive strings and the rejection of negative ones as equally important. 
\end{description}

Conducting sound evaluations requires rethinking important parts of the underlying protocol. To capitalize over and share the cost of such work, we designed Stamina as both a public complement and an alternative to Abbadingo. Focusing on characteristics of behavior models, it also acts as a call to crossfertilization between the machine learning and software engineering communities. To achieve this, Stamina first took the form of an online induction competition before evolving into an online benchmark, in a similar spirit to Abbadingo. The next section details the Stamina setup as designed for the competition. Changes made to the platform since the end of the competition will be explained in Section~\ref{section:stamina-platform}.

\section{Stamina setup\label{section:stamina-setup}}

The competition scenario chosen for Stamina is very similar to the one of Abbadingo.

\begin{quotation}\emph{
A learner downloads a training set made of positive and negative strings, and induces a model using her induction technique. The learned model is then used to label strings of a test sample; these labels are submitted to the competition server. The latter scores the submission and provides a binary feedback as to whether the problem is considered broken or not.
}\end{quotation}

While the competition scenario is similar, Stamina differs from Abbadingo in its focus on the complexity of the learning with respect to the alphabet size. It therefore relies on an adapted generation protocol for target automata and samples.

\subsection{Competition grid}

As in Abbadingo, induction problems are classified in a grid. Here, the competition grid is divided into cells of five problems each, where each cell corresponds to a particular combination of sparsity and alphabet size. Table~\ref{stamina:table:problem-grid} shows how the 100 competition problems are distributed over cells. Easier problems (smaller alphabet and larger sample) are toward the upper-left of the table whereas the harder problems (larger alphabet and smaller sample) are toward the bottom-right.

\begin{table}[h]
\begin{center}
\begin{tabular}{c|c c c c}
&\multicolumn{4}{|c}{Sparsity}\\ 
\textbf{$|\Sigma|$} & \textbf{100\%} & \textbf{50\%} & \textbf{25\%} & \textbf{12.5\%}\\
\hline
\textbf{2}  & 1-5   & 6-10  & 11-15 & 16-20 \\
\textbf{5}  & 21-25 & 26-30 & 31-35 & 36-40 \\
\textbf{10} & 41-45 & 46-50 & 51-55 & 56-60 \\
\textbf{20} & 61-65 & 66-70 & 71-75 & 76-80 \\
\textbf{50} & 81-85 & 86-90 & 91-95 & 96-100\\
\end{tabular}
\end{center}
\caption{\label{stamina:table:problem-grid}Grid of 100 problems distributing the induction difficulty along two dimensions: sparsity of the learning sample and alphabet size.}
\end{table}

With respect to Abbadingo, the following similarities and differences should be noted:

\begin{itemize}

\item An increasing size of the alphabet forms a first difficulty dimension, ranging from 2 to 50 letters. The lower bound allows result comparisons with Abbadingo on easiest problems; the upper bound is representative of behavior models found in the literature.

\item In Abbadingo, the varying automaton size is one difficulty dimension (ranging from 64 to 512). In contrast, Stamina considers automata of roughly 50 states only. These automata provide characteristics of behavior models in terms of the variance of their state degree, among other differences (see Section~\ref{subsection:stamina-machines}).

\item The second difficulty dimension is the decreasing size of the training sample; it is similar to Abbadingo. Nevertheless, samples in Stamina are generated ``from the machine'' by a random walk procedure, instead of randomly drawn from all possible strings (see Section~\ref{subsection:stamina-samples}).

\item Instead of an accuracy measure, submissions in Stamina are scored using a \emph{balanced classification rate} (BCR). BCR places an equal emphasis on the accuracy of an inferred model in terms of acceptance of positive sequences and rejection of negative ones. Obtaining a BCR score of at least 99\% is required to consider a problem broken. A cell is broken if its five problems are broken by the same learner (see Section~\ref{subsection:stamina-scoring})

\item Unlike Abbadingo, Stamina allowed one winner only; the latter is the first learning technique to break a hardest cell among those broken during the competition. To adjust the grid and choose cell difficulties, Blue-fringe has been ran and scored on each problem (see Section~\ref{subsection:stamina-baseline}). The adjustement consisted in ensuring that Blue-fringe almost breaks the problems of the very first cell (alphabet of 2 letters and sparsity of 100\%)\footnote{3 of the 5 problems in the cell are broken by Blue-fringe, which obtains a BCR score of 98\% on the two remaining problems} while obtaining a decent score on cells immediately adjacent. An opensource implementation of this baseline algorithm has been made available for download.

\end{itemize}

\subsection{State Machines\label{subsection:stamina-machines}}

For state machine generation, a quick review of software models has been conducted. Observations were made on a small sample of case-study models found in research publications (about 20 systems). State machine models were analyzed in terms of number of states their states, transitions, alphabet sizes, in-/out degree and depth. Although the sample is too small to form any authoritative conclusion, findings can be interpreted as indicative. Based on these observations, Stamina machines were generated using a variant of the Forest-Fire algorithm~\cite{Leskovec2007}. The algorithm has been tuned to generate state machines presenting the following characteristics.
\begin{description}

\item[Number of states] All state machines have approximately 50 states. This is somewhat larger than the conventional state machines identified in the literature; the aim is to ensure that a well-performing technique could scale to infer models for reasonably complex software systems. It has also been decided not to consider state machines of exactly 50 states in order to avoid a strong bias in the benchmark. Automaton sizes actually range from 41 to 59 states. The latter information was not disclosed during the competition.

\item[Accepting state ratio] A roughly equal proportion of accepting and rejecting states has been chosen, a similarity to Abbadingo. States of most behavior models, notably LTS, are accepting states. It has however been decided to keep non-accepting states as well. This keeps our setup sufficiently close to former competitions such as Abbadingo; in particular, it avoids restricting the problem to the inference of prefix-closed regular languages (see Section~\ref{section:background-lts-and-regular-languages}).

\item[Degree distribution] Following observations from the literature, state machines exhibit an important variance of their state degree, especially on largest alphabets. They may also have sink accepting states, that is, states with no outgoing transition.

\item[Determinism and minimality] Following a common setup of regular inference experiments, all Stamina machines are both deterministic and minimal.

\end{description}

\subsection{Training and test samples\label{subsection:stamina-samples}}

Training and test samples were generated using a dedicated generation procedure. This procedure aims at simulating the way examples of system behavior are usually obtained in the software engineering community (e.g. a collection of program traces at implementation level, the generation of scenarios at design level, and so on). Stamina samples present the following characteristics.

\begin{description}

\item[Generated by the target] A dedicated algorithm were implemented to generate positive strings by walking through the automaton. From the initial state it randomly selects outgoing transitions with a uniform distribution. When an accepting state $v$ is reached, the generation ends with a probability of $1.0/(1 + 2*outdegree(v))$. This procedure simulates an ``end-of-string'' transition from state $v$ with half the probability of an existing transition. The length distribution of the strings generated is approximately centered on $5 + depth(automaton)$. As in Abbadingo, this provides a good chance of reaching the deepest state of the automaton. However, no guarantee is given of having a structurally complete sample.

\item[Negative strings] Negative strings were generated by randomly perturbing positive strings. Three kinds of edit operation are considered here: substituting, inserting, or deleting a symbol. The editing position is randomly chosen according to a uniform distribution over the string length. Substitution and insertion also use a uniform distribution over the alphabet. The number of editing operations is chosen according to a Poisson distribution (with a mean of 3); the three editing operations have an equal change of being selected. The randomly edited string is included in the negative sample provided it is indeed rejected by the target machine. Otherwise, it is simply discarded.

\end{description}

The random walk algorithm and perturbation procedure serve as building blocks for the generation of training and test samples for each problem. A set of 20.000 strings is first generated as described above; it contains a roughly equal number of positive and negative strings and contains duplicates. The distinct strings are then equally partitioned into two sets, $S_0$ and $S_1$, taking care of respecting the positive and negative balance in each one.
\begin{itemize}
\item A test sample contains 1500 strings randomly drawn from $S_0$ without replacement. It never contains duplicates in order to avoid favoring repeated strings in the scoring metric.
\item The official training sets are sampled from $S_1$ with different levels of sparsity (100\%, 50\%, 25\%, 12.5\%). They usually contain duplicates, as a consequence of random walk generation from the target machine. 
\item Training and test sets do never intersect.
\end{itemize}

\subsection{Submission and Scoring\label{subsection:stamina-scoring}}

Solutions to Stamina problems must be submitted as binary strings to the competition server. The learner is expected to produce a binary sequence of labels where, for each test string, a ``1'' is added to the sequence if the string is considered to be accepted, a ``0'' otherwise. 

To assess solution accuracy, the sequence is compared with a reference string representing the correct labeling of the test set by the target model. The overlap between the two binary strings is measured through the \emph{balanced classification rate (BCR)}. The harmonic BCR measure was chosen because it places equal emphasis on the accuracy of the inferred model in terms of acceptance of positive sequences and rejection of negative ones. Moreover, it does not require the test set to be balanced in terms of number of positive and negative sequences. 

Harmonic BCR is the harmonic mean of two factors. \emph{Sensitivity} is the proportion of positive matches that are predicted to be positive and \emph{Specificity} is the proportion of true negatives that are predicted to be negative. Usually, BCR is computed as arithmetic mean of sensitivity and specificity; the harmonic mean is preferred here because it favors balance between the two factors. Let $TP$, $FP$, $TN$ and $FN$ denote the sets of true positives, false positives, true negatives and false negatives, respectively. Sensitivity and specificity are defined as follows:

$$Sensitivity=\frac{|TP|}{|TP \cup FN|}$$ 

$$Specificity=\frac{|TN|}{|TN \cup FP|}$$

$$BCR=\frac{2*Sensitivity*Specificity}{Sensitivity + Specificity}$$

A problem is considered broken if the BCR score obtained is greater than or equal to 0.99. As in Abbadingo, hill-climbing is made almost impossible by a binary feedback from the oracle, depending on whether the problem is broken or not. A cell is considered broken if all its five problems are broken. The Stamina website dedicates a private section to each registered participant that provides visual feedback about the performance of her algorithm(s). In this section, problems and cells of the submission grid turn to green when broken.

\subsection{Blue-fringe baseline\label{subsection:stamina-baseline}}

Blue-fringe has been ran on all Stamina problems in order to adjust the grid for feasibility and fix the difficulty level of each cell.

Adjusting the grid is similar in spirit to what has been done for Abbadingo. The idea is to adjust free parameters of the competition design in such a way that a state-of-the-art algorithm, here Blue-fringe, breaks the easiest problems. Adjusted parameters were the sizes of the training and test sets and the average length of the strings with respect to the automaton depth. This trial-and-error process converged with three problems broken in the easiest cell; reasonable scores were also observed for Blue-fringe on the two remaining problems and in adjacent cells.

The performance of the Blue-fringe baseline is summarized in Table~\ref{table:stamina-baseline} and illustrated with convergence curves in Fig.~\ref{stamina:image:bluefringe-performance}. As shown there, the BCR score decreases along each of the two difficulty dimensions; this experimentally confirms the expected effect of an increasing alphabet size on the induction difficulty. 

\begin{table}
\begin{center}
\begin{tabular}{c|c c c c}
&\multicolumn{4}{|c}{\textbf{Sparsity}}\\ 
\textbf{$|\Sigma|$} & \textbf{100\%} & \textbf{50\%} & \textbf{25\%} & \textbf{12.5\%}\\
\hline
\textbf{2}  & 0.99 (1) & 0.95 (1) & 0.67 (3) & 0.66 (3)\\
\textbf{5}  & 0.97 (1) & 0.78 (2) & 0.59 (4) & 0.52 (4)\\
\textbf{10} & 0.93 (1) & 0.64 (3) & 0.51 (4) & 0.50 (4)\\
\textbf{20} & 0.91 (1) & 0.63 (3) & 0.54 (4) & 0.51 (4)\\
\textbf{50} & 0.81 (2) & 0.64 (3) & 0.57 (4) & 0.50 (4)\\
\end{tabular}
\end{center}
\caption{Average BCR of Blue-fringe in each cell; the difficulty level is shown in parenthesis.\label{table:stamina-baseline}}
\end{table}

\begin{table}
\begin{center}
\begin{tabular}{c|c}
Difficulty level & Score\\
\hline
1&$0.9 \leq score \leq 1$\\
2&$0.7 \leq score < 0.9$\\
3&$0.6 \leq score < 0.7$\\
4&$0 \leq score < 0.6$\\
\end{tabular}
\end{center}
\caption{\label{stamina:table:calibration}Calibrating cell difficulties, based on the scores given in Table~\ref{table:stamina-baseline}}
\end{table}

\begin{figure}
\centering\scalebox{.4}{
  \includegraphics*{src/6-stamina/images/bluefringe-performance}}
  \caption{Performance curves of Blue-fringe\label{stamina:image:bluefringe-performance}.}
\end{figure}

Thanks to those scores, the difficulty level of each cell has been calibrated using the rules defined in Table~\ref{stamina:table:calibration}. The notion of cell difficulty proved useful for driving the competition. Indeed, the winner was the first technique to have broken a hardest cell. It is not used anymore since the competition has evolved into an evaluation platform with exact scoring (see Section~\ref{section:stamina-platform}).

\section{Competition results\label{section:stamina-results}}

The Stamina competition started on March 2010 with the 31th of December announced as an official deadline for submitting results. Between these two dates, 1856 submissions have been made by 11 challengers. Among them, 61 have a BCR score of at least 99\%, breaking 42 different problems in 10 different cells.

The competition hall of fame is shown in Fig.\ref{table:stamina-hall-of-fame}. The easiest cell (alphabet 2, sparsity 100\%) has been quickly broken --~five days after the competition start~-- by Manuel V\'azquez de Parga Andrade with the Equipo algorithm, that learns automata teams~\cite{Garcia:2010}). This first result has been followed by an apparent lull in the competition. Some challengers were actually submitting at that time without successfully breaking new problems and cells. Therefore, in the absence of a more detailed hall of fame, this activity was not directly visible on the competition website. A few weeks after, Marijn Heule and Sicco Verwer, with the DFASAT algorithm (described later), started solving all cells of difficulty~1 in the left-most column. After a new apparent lull, they eventually broke the first cell of difficulty~2 (alphabet 50, sparsity 100\%) therefore taking the head of the competition. They eventually won the competition with an extra cell broken (alphabet 50, sparsity 50\%), the only cell of difficulty~3 broken during the competition. Individual problems have also been broken in cells of the second column (sparsity 50\%), as shown by numbers in Table~\ref{table:stamina-hall-of-fame}.

\begin{table}
\begin{center}
\begin{tabular}{c|c c c c}
&\multicolumn{4}{|c}{\textbf{Sparsity}}\\ 
\textbf{$|\Sigma|$} & \textbf{100\%} & \textbf{50\%} & \textbf{25\%} & \textbf{12.5\%}\\
\hline
\textbf{2}  & Equipo (1) & \emph{4 broken} (1)  & - (3) & - (3) \\
\textbf{5}  & DFASAT (1) & \emph{1 broken} (2)  & - (4) & - (4) \\
\textbf{10} & DFASAT (1) & \emph{3 broken} (3)  & - (4) & - (4) \\
\textbf{20} & DFASAT (1) & \emph{4 broken} (3)  & - (4) & - (4) \\
\textbf{50} & DFASAT (2) & DFASAT (3) & - (4) & - (4) \\
\end{tabular}
\end{center}
\caption{Stamina hall of fame. Broken cells are annotated with the winner name. In other cells, the number in italics indicates how many of the five problems were broken the 31th of December, if any. Difficulty levels in parenthesis are recalled from Table~\ref{table:stamina-baseline}.\label{table:stamina-hall-of-fame}}
\end{table}

The fact that DFASAT has broken the cell (alphabet 50, sparsity 50\%) but not easier cells in the same column might appear strange at first glance. This result must first be slightly weakened, then explained. First, one can verify in Table~\ref{table:stamina-hall-of-fame} that 17 of the 25 problems of the column have been actually broken (all of them are broken by DFASAT). Strategic participant choices must also be taken into account. After having solved cells of difficulty 1 and 2, they probably focused on solving cells of higher difficulty. This precaution taken, the fact that DFASAT relies on a useful scoring heuristics for large alphabets is certainly important for explaining their success on that hardest cell (see Section~\ref{subsection:stamina-winning}).

As the same table shows, no problem has been broken in the last two columns, that is with a sample sparsity of 25\% or less. Here again, this fact must be interpreted with caution. First, while fewer activity has been monitored on such cells, a few approaches actually performed quite well -- above the baseline in particular. Also, because of specific strategies used by the different participants, only partial results were available at the end of the competition. In other words, participants have not submitted results on all available problems, so that making comparisons and drawing conclusions is made difficult. Before discussing trends on those hardest cells, however, let be fair and present key ingredients used by the winning algorithm.

\subsection{The winning algorithm\label{subsection:stamina-winning}}

DFASAT reuses the baseline algorithm in an appropriate way but relies on additional key ingredients: an adapted heuristics for scoring candidate merging operations, a random perturbation of such heuristics and intense search after an original reduction to a satisfiability problem (hence the DFA\textbf{SAT} name). Each of them is described in turn in the following sections.

\subsubsection*{Regular induction seen as a SAT problem}

The idea behind the satisfiability part of DFASAT is to first reduce the problem of finding the minimal DFA consistent with a labeled sample to a graph coloring problem, to encode the latter into a satisfiability problem, and finally to solve it using an efficient SAT solver. 

The principle of reducing DFA induction to graph coloring dates back to 1997 (more or less as the Abbadingo contest itself, see e.g.~\cite{Coste:1997}) and roughly proceeds as follows. Consider an undirected graph made of one node for each state of the augmented PTA encoding the training sample. Two nodes of this graph are connected if they cannot be merged, that is, if their respective state in the PTA have different labels (i.e. accepting vs. error state) or if merging them would lead to merging states of different labels in virtue of the determinization process. These transitions can be computed in a way similar to the pre-computation of incompatibilities between PTA states which is sometimes used in classical state merging algorithms, e.g.~\cite{Coste:1998, Coste:2004}. Coloring the graph obtained with different colors for adjacent nodes is equivalent to finding a consistent DFA. Indeed, used colors defines a partition of the PTA states (see Section~\ref{section:inductive-background}) so that merging states of the same color leads to a DFA, which is consistent with the sample by construction. The amount of colors being equal to the size of the DFA, finding the minimum leads to solving the problem of minimal consistent DFA.

The main drawback of such an approach is that the number of graph nodes, and of edges capturing coloring constraints, quickly becomes huge for big samples. This can lead to coloring problem instances that are impractical in practice. In spite of a compact boolean encoding of the graph coloring problem and huge enhancements in SAT solving in recent years, some of the Abbadingo and Stamina problems remain intractable with this technique only. See~\cite{Heule:2010}, by the same authors as DFASAT, for details about their graph coloring and SAT translations.

\subsubsection*{Mixing state-merging and SAT solving}

To overcome this tractability problem, the authors reuse classical state-merging upstream their SAT solving technique. The idea here is to execute the first steps of Blue-fringe (or another EDSM variant) to first reduce the PTA to a partially identified DFA. Indeed, the SAT encoding of the authors does not depend on a tree-shaped automaton and is sufficiently flexible to work with any DFA. However, if executing even the first steps of a state-merging algorithm drastically reduce the number of states, and hence the number of constraints, this state-merging step implies that the solution provided by the SAT solver is no longer exact. Nevertheless the first few merges performed by Blue-fringe are normally supported by a lot of evidence. Consequently, they are likely to be correct and are then expected to lead to an optimal solution in practice~\cite{Heule:2010}, provided that a good scoring heuristics is used. 

Interestingly, according to the winners themselves, the EDSM + SAT technique so far was not sufficient to break Stamina cells above the first difficulty level. As another technique already had gained a cell of same difficulty they have had to further optimize their technique, as explained in the next section.

\subsubsection*{A few additional ingredients required}

In order to obtain convincing results on those harder cells, a few additional ingredients to the procedure described so far are required:

The first one is a different scoring heuristics than the one used by EDSM/Blue-fringe. Indeed, since Abbadingo it is commonly admitted that the heuristics must be based on the number of merged states sharing the same label. To win the Stamina competition, however, DFASAT had to use a different one, related to the number of merged transitions sharing the same symbol. The explanation relies on the kind of automata considered, especially on large alphabets (see Section~\ref{subsection:stamina-machines}): as two different states of the target DFA are unlikely to have many outgoing transitions in common, two PTA states having more than a few of them are very likely to correspond to the same target state.

Also, Instead of considering only one solution, as commonly done with greedy state-merging algorithms like Blue-fringe, DFASAT includes a search strategy over multiple candidates. While this is natural with SAT-based graph coloring applied to grammar induction (since all possible solutions are captured in a compact form) the authors have also intentionally introduced search in two other places. The first one involves a random perturbation of the scoring heuristics, to reduce the criticality of wrong initial merges potentially made during the first phase. The second one was to also look for non-minimal consistent automata by considering more colors than actually required for coloring the graph. While doing so appears in contradiction with the Occam's razor principle at first glance, it actually makes sense in the context of Stamina. Indeed, the actual problem to solve in the competition is not strictly speaking the identification of target automata but the learning of good approximate solutions reaching 99\% of BCR score. In view of the competition setup and generation protocols, looking for automata of about 50 states is probably a better criteria than looking for the minimal consistent one.

\subsection{Trends on hardest cells\label{subsection:stamina-trends-on-hardest-cells}}

Table~\ref{table:stamina-hall-of-fame} only reveals competition activity according to its winning criteria, that is, to be the first to break the hardest cell among those broken. Interesting hidden activity (so far) may be revealed and trends drawn if we turn to a different criteria for presenting the hall of fame. Consider the Table~\ref{table:stamina-hall-of-fame-2}. In contrast to Table~\ref{table:stamina-hall-of-fame} that considers broken cells only, this one shows for each cell the name of the best technique together with its average score on cell problems. To keep a fair comparison criteria by cell, only participants that have submitted for the five problems it contains are taken into account. Also, only the best participant score for each problem is kept for computing the average score of the cell. Last, only results above the baseline are taken into account. In other words, empty cells may be interpreted as being owned by the baseline itself. 

Results in Table~\ref{table:stamina-hall-of-fame-2} have been compiled according to that criteria, from available data at competition termination. Two observations are straightforward: 

\begin{itemize}

\item First, the pbc algorithm (\emph{pcb} stands for \emph{pattern based classification}, a modification of the technique presented in~\cite{Lo:2009}) performs well on hardest problems in comparison with the baseline which obtains around 0.52 in average for the same cells (baseline scores have been given in Table~\ref{table:stamina-baseline}). 

\item Second, DFASAT is near of breaking a cell of difficulty 4 (alphabet 50, sparsity 25\%), with an average score of 0.967. Interestingly, the variance is small (the best score being 0.985 and the worse 0.929) which supports our claim. The other score obtained on this column (0.788 on alphabet 2, sparsity 25\%) tend to confirm the usefulness of DFASAT's scoring heuristics discussed in the previous section, since better results are obtained on large alphabets than on small ones (here also, strategic choices made by DFASAT might also partly explain this difference).

\end{itemize}

\begin{table}[H]
\begin{center}
\begin{tabular}{c|c c c c}
&\multicolumn{4}{|c}{\textbf{Sparsity}}\\ 
\textbf{$|\Sigma|$} & \textbf{100\%} & \textbf{50\%} & \textbf{25\%} & \textbf{12.5\%}\\
\hline
\textbf{2}  & DFASAT (\emph{br.}) & DFASAT (0.987)      & DFASAT (0.788) &  \\
\textbf{5}  & DFASAT (\emph{br.}) & DFASAT (0.968)      &                &  \\
\textbf{10} & DFASAT (\emph{br.}) & DFASAT (0.979)      & pbc (0.680)    & pbc (0.688) \\
\textbf{20} & DFASAT (\emph{br.}) & DFASAT (0.991)      & pbc (0.734)    & pbc (0.726) \\
\textbf{50} & DFASAT (\emph{br.}) & DFASAT (\emph{br.}) & DFASAT (0.967) & pbc (0.750) \\
\end{tabular}
\end{center}
\caption{Hall of fame based on average scoring by cell, provided that the participant has submitted for all problems in the cell (\emph{br.} stands for \emph{broken}).\label{table:stamina-hall-of-fame-2}}
\end{table}

The fact that only partial results are available do not allow fully drawing comparisons between scores obtained by the different approaches. For example, pbc did not submit on smallest alphabets while DFASAT did not submit on sparsest samples, as illustrated in Fig.~\ref{image:stamina-winners-performance-comparison} where their respective performance curves are plotted. Also, a few other techniques significantly outperformed the baseline on a variety of problems, but have not submitted for all problems of a given cell and are therefore not taken into account in Table~\ref{table:stamina-hall-of-fame-2}. Even when restricting our attention to pbc and DFASAT, the table shows that, while pbc performs well, it is very likely that DFASAT would outperform it on sparsest samples. Two observations support this claim. First, cells where both algorithms have submitted are credited to DFASAT (sparsity of 50\% on largest alphabets, for instance). Also, with a sparsity of 25\%, DFASAT already outperforms pbc in average on the two cells with smallest and largest alphabets. 

\begin{figure}[ht]
\centering\scalebox{.26}{
  \includegraphics[trim=20mm 0mm 25mm 0mm, clip]{src/6-stamina/images/DFASAT-performance}
  \includegraphics[trim=20mm 0mm 25mm 0mm, clip]{src/6-stamina/images/pbc-performance}}
  \caption{Performance curves of DFASAT and pbc\label{image:stamina-winners-performance-comparison}.}
\end{figure}

In any case, a quick comparison of the curves in Fig.~\ref{image:stamina-winners-performance-comparison} with the baseline curves in Fig.~\ref{stamina:image:bluefringe-performance} perfectly illustrates the overall contribution of the competition and of these two approaches in particular: on the kind of problems considered in Stamina, the state-of-the-art in grammar induction has been pushed significantly forward.


\section{Evolution towards an evaluation platform\label{section:stamina-platform}}

In the spirit of Abbadingo, the competition website is still available online. It is aimed at becoming an online benchmark for evaluating novel induction techniques. To achieve this goal a few changes were made on the competition server:
\begin{itemize}

\item The average score obtained by Blue-fringe on each cell has been published in the documentation section of the website. Moreover, the cell difficulty level has been made obsolete; it is only kept for documentation of the competition itself.

\item The public hall-of-fame has been updated. Instead of displaying the winners of broken cells only, each cell is annotated with the three best challengers in descending order of their average score for that cell. Their score is also disclosed. 

\item The oracle has been modified to provide the exact score obtained when attempting to break a problem. The private submission grid of each participant displays the best score obtained on each problem for which she submitted. 

\end{itemize}

These choices were made to provide a more transparent feedback while keeping a competitive aspect to the platform. Future work is worth considering:
\begin{itemize}

\item Interactive inductive techniques, e.g. QSM, could hardly compete so far due to the absence of an online oracle answering membership queries. Implementing such an oracle presents no particular difficulty; it would however require generating fresh new problems to avoid interfering with the current grid and challengers already competing on it. 

\item White box benchmarks, where target machines would be disclosed together with samples, might also be envisaged. This would help evaluating inductive techniques that use domain knowledge; the target machines prove useful for simulating such information. Alternatively, sharing binaries for generating new target machines and samples could help reusing the Stamina protocol. 

\end{itemize}



\section*{Summary}

This chapter discussed \emph{Stamina}, an evaluation platform for inductive model synthesis. Stamina follows a previous platform from the machine learning community known as Abbadingo. Our work on Stamina sets a basis for overcoming limitations of Abbadingo for evaluating inductive model synthesis techniques such as QSM and ASM. In particular, random automata in Stamina are more representative of the state machines found in software engineering problems.

Stamina was first designed to be a formal competition in automaton induction before evolving into an online benchmark. The competition was won by DFASAT, a novel induction algorithm that significantly pushed forward the state-of-the-art by outperforming Blue-fringe used as baseline. The design of Stamina make the competition results especially relevant for our software engineering standpoint. In particular, DFASAT opens important perspectives for inductive model synthesis in our application domain.
