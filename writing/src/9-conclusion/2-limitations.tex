\section{Limitations and perspectives\label{section:conclusion-limitations}}

The formal definition of the multi-view modeling framework used in the thesis raises a certain number of issues in terms of behavior model expressiveness.
\begin{itemize}

\item Labeled transition systems do not distinguish between input and output events. Such distinction might be necessary for specific analyses, such as implied scenario detection \cite{Letier:2005b}. This could be addressed either by considering the use of input-output automata \cite{Lynch:1987} or explicitly introducing models for capturing the structural system dimension \cite{Jackson:1995, Magee:1995}.

\item The behavior models capturing agent and system behaviors define any prefix of an accepted trace as an accepted trace. While keeping our framework simple to use, it also slightly limits the class of systems that can be captured.

\item The used framework does not distinguish between successful and unsuccessful system executions. Among others, this does not allow reasoning in terms of deadlocks, an important aspect when analyzing distributed systems. Labeled transition systems can easily be extended to support this, see e.g. \cite{Uchitel:2003}. Enriching scenario models might be necessary to benefit of deadlock analysis in multi-view modeling.

\end{itemize}

About the semantics of guarded hMSC, vertical synthesis algorithms and the model-checker:
\begin{itemize}

\item The expressiveness of guarded hMSC as a process language is rather limited. In particular, guards are limited to Boolean variables such as fluents. So called \emph{tracking  variables} are also introduced in \cite{Damas:2011}, but are limited to Boolean variables as well. Higher-level modeling abstractions could possibly be defined and translated to our trace semantics, such as \emph{counters} or \emph{resources}.

\item When a safety property is violated by a process, our model checker returns a counterexample as a pure event-trace. Works remains to be done so as to provide an interpretation of such counterexample on the process model itself. 

\item The composition operator of guarded LTS raises issues about the intended compositional semantics. In particular, whereas the semantics of guarded LTS is in terms of LTS, the thesis does not provide guarantees for a roundtrip between guarded LTS composition and LTS composition.

\end{itemize}

About our horizontal inductive synthesis technique and related grammar induction contributions:
\begin{itemize}

\item When using the interactive feature of QSM, an important open question is the robustness to possible misclassification of scenarios questions by the end-user. 

Traditional ways of dealing with noisy inputs include probabilistic learning methods, which are not necessary relevant here. The availability of domain knowledge could help detecting and/or correcting such mistakes.

\item QSM raises an issue about the number of scenario submitted for classification by the end-user. As illustrated in experiments, additional system knowledge such as fluents and goals helps reducing the number of questions to be rejected. However, the number of accepted scenarios does not similarly reduce. This might lead to usability issues for large systems.

One way to tackle this problem would be to explore new ways for interacting with the user. The latter could for example request to terminate the induction process early. Visual inspection of generated state machines would yield a form of equivalence queries. 

\item The projection of the synthesized LTS on local agents is known to possibly introduce additional behaviors, the so-called implied scenarios. Negative implied scenarios, i.e. undesirable system behaviors, were shown to hurt multi-model consistency such as possibly violating goals. 

Refactoring techniques to help eliminating undesired implied scenarios still need to be developed.

\item Work remains to be done to let QSM and ASM evolve towards a larger and more flexible induction approach to multi-view behavior modeling.

The so far partial transition from QSM to ASM in our implementation already raises questions and open perspectives about the availability of a superseding induction algorithm, ASM$^*$. If the importance of a convergence criterion were discussed, the notion of characteristic sample needs to be revisited.

Relying on regular language theory and grammar induction algorithms for behavior model synthesis were shown useful, both from practical and theoretical standpoints. Integrating such approaches in a broader, truly incremental approach to behavior model synthesis for requirement elaboration and the exploration of system designs were discussed as opening promising perspectives but requiring further research.

\end{itemize}
