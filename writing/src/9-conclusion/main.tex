\chapter{Conclusion\label{chapter:conclusion}}

Models play a significant role for elaborating requirements and exploring designs of software systems. They help abstracting from multiple details in order to focus on key system aspects. This abstraction process naturally leads capturing complex systems through multiple models. Each of them focusses on particular facets of the system along its intentional, structural, operational and behavioral dimensions.

Building high-quality models for software system is not an easy task. To play a significant role, models should adequately represent the essence of the target system; they should be complete enough to capture all its pertinent facets; high-quality models should be precise when taken in isolation; they should be consistent with each other in multi-view modeling. 

Adequate modeling languages, tools and techniques are therefore required to help building them. This thesis investigated model synthesis as a promising formal support along these requirements. Model synthesis has been articulated along two main axes in the thesis:
\begin{description}
\item[Vertical synthesis] To capture the operational semantics of high-level models by deriving lower-level ones. This kind of synthesis also makes high-level models amenable to analysis.

Along this axe, the thesis defined an operational semantics for guarded high-level Message Sequence Charts (hMSC), a process modeling language used for capturing critical processes involving decisions \cite{Damas:2011}. This semantics has been operationalized through labeled transition systems (LTS) and synthesis algorithms have be described. The implementation of a model-checker for guarded hMSCs has been described; it adapts a compositional LTS technique borrowed from \cite{Giannakopoulou:2003}.

\item[Horizontal synthesis] To produce missing models in multi-view descriptions, or complete existing ones. Horizontal synthesis uses the inherent overlap in multi-view models to semi-automatically produce or complete model fragments from various sources of knowledge.

Along this axe, the thesis showed how grammar induction can be used to synthesize state machines from positive and negative scenarios illustrating interactions among system agents. This technique operates in a golden triangle where declarative goals can be used to constrain the induction process so as to avoid undesired system behaviors. It also interacts with an end-user who classifies additional scenarios generated by the process as examples and counterexamples of desired system behaviors.
\end{description}

While distinguishing between horizontal and vertical model synthesis may sometimes be a slightly arbitrary, these two axes prove fairly different:
\begin{itemize}
\item Vertical synthesis compiles high-level models into lower-level abstractions; horizontal synthesis does not. 

For example, guarded process models made of refinable tasks and decision nodes were derived as a structure form of transition systems involving guards, then as flat LTS involving events only. In contrast, the interactive synthesis of state machines from scenarios triggers for the identification of declarative goals.

\item Vertical synthesis is derivational by nature whereas horizontal synthesis may be inductive. In the former, the user has a passive role whereas she may have an active role in the latter.

For example, the analysis of guarded process models in the Gisele toolset was shown to benefit from running as an fully automated background process. In contrast, our interactive state machine synthesizer implements an horizontal inductive synthesis where the end-user was seen to play an important role to avoid overgeneralizations.

\item Models derived in vertical synthesis are hidden from the user and thrown away once the analysis results have been obtained. Horizontal synthesis yields models requiring user validatation; such models also have a long lifecycle, such as models involved in requirements or used in documentation.

The thesis discussed how the Gisele toolset proposes various checks on process models without keeping nor even disclosing the synthesized transition systems to the user. In contrast, the ISIS synthesizer was shown to generate scenario questions to the user, and to infer safety properties and annotate synthesized state machines in order for analysts to validate them.

\end{itemize}

\section{Contributions\label{section:conclusion-contributions}}

Contributions along the first axe of model synthesis follows from our work on guarded hMSC:

\begin{itemize}

\item This thesis gave a formal semantics to guarded hMSC. This semantics is defined through the introduction of guarded LTS, a kind of labeled transition systems allowing events and guards on transitions. Guarded LTS are given a declarative trace semantics as well as composition and hiding operators.

\item A synthesis algorithm has been proposed that operationalizes the semantics of guarded hMSC and guarded LTS in terms of pure LTS. Among others, this semantics makes guarded models amenable to trace-based model checking, requiring only a few adaptations of existing compositional techniques.

\end{itemize}

Contributions along the second axe, i.e. model synthesis as a support for multi-view modeling, are:

\begin{itemize}

\item An interactive procedure for synthesizing LTS agent state machines, from positive and negative MSC scenarios as well as their flowcharting in high-level MSCs. This procedure uses grammar induction for generalizing behaviors and may be guided by an end-user who classifies additional scenarios as positive or negative examples of system behaviors.

\item A technique to constrain and prune the search space of inductive LTS synthesis by incorporating various sources of knowledge about the target system. This technique is instantiated to take into account fluent definitions, legacy components, goals and domain properties. This guarantees the consistency of the synthesized state machines with other models, notably goals. 

\end{itemize}

The techniques listed above are supported by libraries and tools, available under open-source licences. Techniques and tools have been evaluated on typical case studies and synthetic data. In the latter case, an novel evaluation protocol for inductive model synthesis has been proposed and ran as a formal competition.

\section{Limitations and future directions\label{section:conclusion-limitations}}

