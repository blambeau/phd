\chapter{Conclusion\label{chapter:conclusion}}

Models are increasingly recognized as an effective means for elaborating requirements, exploring designs of software systems, analyzing and setting up work processes, and so on. Models help abstracting from multiple details in order to focus on key system aspects.

This abstraction process naturally leads to capture complex systems with multiple models. Each of them focuses on particular facets of the system, i.e. its intentional, structural, operational and behavioral dimensions. For example, scenarios illustrate interactions between agents forming the system whereas goals make the underlying objectives precise.

If building software systems is known to be an difficult task, one can hardly argue that building quality models for them is easier.  To play a significant role, models should adequately represent the essence of the target system; they should be complete enough to capture all its pertinent facets; models should also be precise when taken in isolation; they should be consistent with each other in multi-view modeling; and so on. Adequate modeling languages, tools and techniques are therefore required to help building them.

Two distinct yet intertwined activities can be formaly supported: model \emph{analysis} and model \emph{synthesis}. Model analysis mostly aims at querying and verifying models against various kinds of properties; this typically involves techniques such as model checking. Model synthesis aims at producing models for various source of knowledge, typically other models.

This thesis has investigated model synthesis techniques more particularly; its companion by Christophe Damas investigates model analysis \cite{Damas:2011}. Model synthesis can be articulated along two main axes, according to the underlying objective:

\begin{itemize}

\item Model synthesis may be used for defining the operational semantics of high-level models, in terms of lower-level ones. This kind of synthesis is also motivated by the wish to make high-level models amenable to analysis, through techniques available on the lower-level ones.

%Along this axe, the thesis has defined an operational semantics for guarded high-level Message Sequence Charts (g-hMSC), the process modeling language used in \cite{Damas:2011}. This semantics is defined in terms of labeled transition systems (LTS), through a compositional synthesis algorithm. Among other, such choice allows model checking g-hMSC processes through a slight adaptation of the technique from \cite{Giannakopoulou:2003}.

\item Model synthesis also proves useful for producing missing models in multi-view descriptions, or completing existing ones. The idea here is to use the inherent overlap of multi-model system descriptions to semi-automatically produce or complete models from various sources of knowledge.

%Along this axe, the thesis showed how agent state machines can be interactively synthesized from positive and negative scenarios, under the control of goals and domain properties. This technique reuses and extends grammar induction algorithms from \cite{Oncina:1992, Lang:1998}. To avoid over-generalizing scenario behaviors, the synthesis process may be guided by an end-user. The latter then classifies scenario questions as positive or negative examples of system behaviors.

\end{itemize}

Note that, in addition to persuing different goals, the two usages of model synthesis above differ in important ways.

\begin{itemize}

\item Semantic-driven synthesis is derivative by nature. In contrast, the completion of multi-view system descriptions benefits from inductive synthesis processes such as grammar induction or inductive logic programming.

\item When used for triggering analyses, produced models have a short lifetime, generally limited to the analysis itself. Models in multi-view descriptions have a different lifecycle as they are often kept as software documentation or for requirements traceability. 

\item In the first case, models are generally kept hidden from the user who has a passive role in the generation process. In the second case, models must generally be validated by the end-user. The latter may also play an active role in the synthesis process, such as the oracle who classifies scenarios as positive of negative system behaviors.

\end{itemize}

Two sections conclude the thesis. Section \ref{section:conclusion-contributions} summarizes our contributions to model synthesis along the two axes discussed above. Section \ref{section:conclusion-limitations} discusses limitations and possible future directions. 

\section{Contributions\label{section:conclusion-contributions}}

Contributions along the first axe of model synthesis follows from our work on guarded hMSC:

\begin{itemize}

\item This thesis gave a formal semantics to guarded hMSC. This semantics is defined through the introduction of guarded LTS, a kind of labeled transition systems allowing events and guards on transitions. Guarded LTS are given a declarative trace semantics as well as composition and hiding operators.

\item A synthesis algorithm has been proposed that operationalizes the semantics of guarded hMSC and guarded LTS in terms of pure LTS. Among others, this semantics makes guarded models amenable to trace-based model checking, requiring only a few adaptations of existing compositional techniques.

\end{itemize}

Contributions along the second axe, i.e. model synthesis as a support for multi-view modeling, are:

\begin{itemize}

\item An interactive procedure for synthesizing LTS agent state machines, from positive and negative MSC scenarios as well as their flowcharting in high-level MSCs. This procedure uses grammar induction for generalizing behaviors and may be guided by an end-user who classifies additional scenarios as positive or negative examples of system behaviors.

\item A technique to constrain and prune the search space of inductive LTS synthesis by incorporating various sources of knowledge about the target system. This technique is instantiated to take into account fluent definitions, legacy components, goals and domain properties. This guarantees the consistency of the synthesized state machines with other models, notably goals. 

\end{itemize}

The techniques listed above are supported by libraries and tools, available under open-source licences. Techniques and tools have been evaluated on typical case studies and synthetic data. In the latter case, an novel evaluation protocol for inductive model synthesis has been proposed and ran as a formal competition.

\section{Limitations and future directions\label{section:conclusion-limitations}}

