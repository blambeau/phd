\chapter{Conclusion\label{chapter:conclusion}}

Models play a significant role for elaborating requirements and exploring designs of software systems. They help abstracting from multiple details in order to focus on key system aspects. This abstraction process naturally leads capturing complex systems through multiple models. Each of them focusses on particular facets of the system along its intentional, structural, operational and behavioral dimensions.

Building high-quality models for software system is not an easy task. To play a significant role, models should adequately represent the essence of the target system; they should be complete enough to capture all its pertinent facets; high-quality models should be precise when taken in isolation; they should also be consistent with each other. 

This thesis investigated model synthesis as a promising approach for building high-quality models meeting these requirements. Model synthesis were articulated along two main axes in the thesis:
\begin{description}
\item[Vertical synthesis] To capture the operational semantics of high-level models by deriving lower-level ones. This kind of synthesis makes high-level models amenable to formal analysis.

Along this axe, the thesis defined an operational semantics for guarded high-level Message Sequence Charts (guarded hMSC), a process modeling language used for capturing critical processes involving decisions \cite{Damas:2011}. This semantics is operationalized through guarded labeled transition systems (g-LTS); synthesis algorithms were described to derive guarded hMSC as g-LTS, and to derive the latter as pure event-based LTS. The implementation of a model-checker for guarded hMSCs were also described; it adapts a compositional LTS technique borrowed from \cite{Giannakopoulou:2003}.

\item[Horizontal synthesis] To produce missing model fragments in multi-view descriptions, or complete existing models. Horizontal synthesis uses the rules of consistency between the models to produce or complete model fragments from the  system knowledge available in other views.

Along this axe, the thesis showed how grammar induction methods can be used to synthesize state machines from scenarios illustrating examples and counterexamples of desired system behavior. Our technique interacts with an end-user who accepts or rejects additional scenarios generated by the synthesizer. It operates in a golden triangle between scenarios, state machines and declarative goals; the latter indeed prove an effective way of constraining the induction process so as to avoid undesired system behaviors. The induction process may be pruned with various other sources of system knowledge, resulting in a faster process, a reduced number of user interactions and a strong consistency of the synthesized state machines with such knowledge.
\end{description}

Horizontal and vertical model synthesis present significant differences that were illustrated in various places in the thesis.
\begin{itemize}
\item Vertical synthesis compiles high-level models into lower-level abstractions; horizontal synthesis does not. 

Guarded process models made of refinable tasks and decision nodes were derived as a structure form of transition systems involving guards, then as flat LTS involving events only. In contrast, scenarios, state machine and goals are seen at an equivalent level of abstraction by our interactive synthesis technique.

\item Vertical synthesis is derivational by nature whereas horizontal synthesis may be inductive. In the former, the user has a passive role whereas she may have an active role in the latter.

The analysis of guarded process models in the Gisele toolset was shown running as a background process; the latter automatically compiles process models in state machines when needed. In contrast, our state machine synthesizer is both inductive and interactive; the end-user was seen to play an important role to avoid overgeneralization.

\item Models derived in vertical synthesis are hidden from the user and often thrown away once the analysis results obtained. Horizontal synthesis yields models requiring user validation; such models are often used for requirements traceability and documentation.

The thesis discussed how the Gisele toolset implements a variety of checks on process models without saving nor disclosing the synthesized transition systems to the user. In contrast, the ISIS synthesizer was shown to generate scenario questions to the user, and to infer safety properties and annotate synthesized state machines in order for analysts to validate them.

\end{itemize}

\section{Contributions\label{section:conclusion-contributions}}

Contributions along the first axe of model synthesis follows from our work on guarded hMSC:

\begin{itemize}

\item This thesis gave a formal semantics to guarded hMSC. This semantics is defined through the introduction of guarded LTS, a kind of labeled transition systems allowing events and guards on transitions. Guarded LTS are given a declarative trace semantics as well as composition and hiding operators.

\item A synthesis algorithm has been proposed that operationalizes the semantics of guarded hMSC and guarded LTS in terms of pure LTS. Among others, this semantics makes guarded models amenable to trace-based model checking, requiring only a few adaptations of existing compositional techniques.

\end{itemize}

Contributions along the second axe, i.e. model synthesis as a support for multi-view modeling, are:

\begin{itemize}

\item An interactive procedure for synthesizing LTS agent state machines, from positive and negative MSC scenarios as well as their flowcharting in high-level MSCs. This procedure uses grammar induction for generalizing behaviors and may be guided by an end-user who classifies additional scenarios as positive or negative examples of system behaviors.

\item A technique to constrain and prune the search space of inductive LTS synthesis by incorporating various sources of knowledge about the target system. This technique is instantiated to take into account fluent definitions, legacy components, goals and domain properties. This guarantees the consistency of the synthesized state machines with other models, notably goals. 

\end{itemize}

The techniques listed above are supported by libraries and tools, available under open-source licences. Techniques and tools have been evaluated on typical case studies and synthetic data. In the latter case, an novel evaluation protocol for inductive model synthesis has been proposed and ran as a formal competition.

\section{Limitations and future directions\label{section:conclusion-limitations}}

