\chapter{Conclusion\label{chapter:conclusion}}

Models play a significant role for elaborating requirements and exploring designs of software systems. They help abstracting from multiple details in order to focus on key aspects of the target system. Complex systems should be captured through multiple models. Each of these focusses on particular facets of the system along its intentional, structural, operational and behavioral dimensions.

Building high-quality models for software systems is not an easy task. To play a significant role, models should adequately represent the essence of the target system; they should be complete enough to capture all its pertinent facets; they should be precise when taken in isolation; they should also be consistent with each other. 

This thesis investigated model synthesis as a promising approach for building high-quality models meeting those requirements. Our model synthesis techniques were articulated along two main dimensions:
\begin{description}
\item[Vertical synthesis] yields the operational semantics of high-level models by deriving lower-level ones. This kind of synthesis makes high-level models amenable to formal analysis.

Along this dimension, the thesis defined an operational semantics for guarded high-level Message Sequence Charts (g-hMSC), a process modeling language used for capturing critical processes involving decisions \cite{Damas:2011}. This semantics is defined through guarded labeled transition systems (g-LTS); synthesis algorithms were described to derive g-LTS from g-hMSC, and to derive pure event-based LTS from g-LTS. The intermediate g-LTS language proves convenient for avoiding state explosion and for enabling a variety of analyses at that level~\cite{Damas:2011} The implementation of a model-checker for g-hMSC was also described; it adapts a compositional LTS technique borrowed from \cite{Giannakopoulou:2003}.

\item[Horizontal synthesis] yields missing model fragments in multi-view models. It uses the rules of inter-model consistency to produce or complete model fragments from the other views.

Along this dimension, the thesis showed how grammar induction methods can be adapted to synthesize state machines from scenarios that illustrate examples and counterexamples of desired system behavior. Our technique interacts with an end-user who accepts or rejects additional scenarios generated by the synthesizer. Declarative properties such as goals were seen to be an effective way of constraining the induction process so as to avoid undesired system behaviors. The induction space may be pruned thanks to such knowledge, resulting in a faster process, a reduced number of user interactions and a guaranteed consistency of the synthesized state machines with respect to such knowledge.
\end{description}

The model synthesis techniques proposed in the thesis aim at supporting the incremental building of system models and thereby the exploration of requirements and system designs. This objective is met in different but complementary ways.
\begin{itemize}

\item Vertical synthesis enables model analysis which in turn proves very effective for systematic model building.

The operational semantics of g-hMSCs paves the way for systematic construction through model analysis. The thesis proposed a model-checking procedure for process models. In addition, it enables a variety of analyses detailed in~\cite{Damas:2011}.

The potential benefits of such model building approach is illustrated with the Gisele process model analyzer described in Section~\ref{section:tool-clinical-pathway-analyzer}. This tool has been successfully used for modeling real, safety-critical care processes.

\item Horizontal synthesis achieves similar objectives but in a constructive way.

Our inductive synthesis technique provides flexible and effective support for behavior model building. It requires only a few scenarios in the first place but also supports structured forms of scenarios in input. In addition to synthesizing state machines that are consistent with the input scenarios, the technique calls for the identification of state variables and goals. It allows the initial scenario specification to be iteratively completed with new examples and counterexamples of system behavior.

Our approach is supported by the ISIS tool, described in Section~\ref{section:tool-support-isis}. Its effectiveness has been assessed through various evaluations, both on case studies and on synthetic datasets (see Chapter~\ref{chapter:evaluation}).

\end{itemize}

As illustrated in the thesis, our specific analysis and synthesis techniques already prove effective for model building when taken in isolation. However, their strengths appear even more clearly when they are combined in an integrated environment. This has been observed when using the ISIS tool on case studies. In addition to inductively synthesizing state machines from scenarios, the tool may infer goals from scenarios, decorate synthesized state machines with state invariants, and perform various consistency checks. A similar effect has been observed in the Gisele toolset which integrates a variety of analyses on process models -- see~\cite{Damas:2011} for a convincing case.

\section{Contributions\label{section:conclusion-contributions}}

Contributions along the first axe of model synthesis follows from our work on guarded hMSC:

\begin{itemize}

\item This thesis gave a formal semantics to guarded hMSC. This semantics is defined through the introduction of guarded LTS, a kind of labeled transition systems allowing events and guards on transitions. Guarded LTS are given a declarative trace semantics as well as composition and hiding operators.

\item A synthesis algorithm has been proposed that operationalizes the semantics of guarded hMSC and guarded LTS in terms of pure LTS. Among others, this semantics makes guarded models amenable to trace-based model checking, requiring only a few adaptations of existing compositional techniques.

\end{itemize}

Contributions along the second axe, i.e. model synthesis as a support for multi-view modeling, are:

\begin{itemize}

\item An interactive procedure for synthesizing LTS agent state machines, from positive and negative MSC scenarios as well as their flowcharting in high-level MSCs. This procedure uses grammar induction for generalizing behaviors and may be guided by an end-user who classifies additional scenarios as positive or negative examples of system behaviors.

\item A technique to constrain and prune the search space of inductive LTS synthesis by incorporating various sources of knowledge about the target system. This technique is instantiated to take into account fluent definitions, legacy components, goals and domain properties. This guarantees the consistency of the synthesized state machines with other models, notably goals. 

\end{itemize}

The techniques listed above are supported by libraries and tools, available under open-source licences. Techniques and tools have been evaluated on typical case studies and synthetic data. In the latter case, an novel evaluation protocol for inductive model synthesis has been proposed and ran as a formal competition.

\section{Open issues and perspectives}

This section discusses open issues with our techniques and states a few perpectives for future work.

\subsection*{Incremental building of multi-view models}

For incremental model building to converge towards the desired system model, we need to assume that the latter actually exists, at least theoretically. This is arguable, since the system is unknown. During modeling, different alternatives are explored; some design decisions are rejected whereas others are selected on the way towards a better model.

In practice, this means that system modeling is a highly non linear, trial-and-error process. Model refactoring techniques seem therefore important to complement synthesis and analysis techniques. The availability of all of such techniques in integrated environments would support a more effective exploration and design process. 

For example, a modeling bottleneck in the ISIS tool is the lack of support for scenario refactoring. This may prevent some design explorations and may even hurt the modeling process when such refactoring is required -- due to the presence of negative implied scenarios for example. If such refactoring were to be formally supported, our  synthesis technique would allow the user to effectively rebuild state machines from refactored scenarios. The investigation of the intertwined usage of refactoring, synthesis and analysis techniques is thus an interesting perspective for additional support.

This raises the question of what makes user guidance effective in an environment integrating multiple techniques. As discussed in Chapter~\ref{chapter:tool-support}, the ISIS and Gisele tools differ in the way such guidance is implemented. In the former case, the available techniques and analyses are made contextually available and executed on demand. In the latter case, the analyses run in the background; real-time feedback is provided as the user navigates through the process model. Both approaches were shown to present advantages and drawbacks.

\subsection*{Convergence and scalability of the synthesis process}

Our choice of grammar induction for incremental state machine synthesis is partly motivated by the convergence criterion it provides towards adequate models of the target system (see Section~\ref{section:inductive-discussion}). The soundness of such theoretical convergence has been argued to be for the synthesis of behavior models from both scenarios and goals (see Section~\ref{related-for-requirements-2}). As discussed in Section~\ref{section:inductive-discussion}, work remains to be done to root our ASM$^*$ algorithm on such a sound theoretical framework.

Even if convergence guarantees are provided, QSM raises an issue about the number of scenarios submitted for classification by the end-user. As seen in the experiments, additional knowledge such as fluent definitions and goal specifications help reducing the number of questions to be rejected. However, the number of accepted scenarios does not similarly reduce. This may lead to a scalability problem for interactive synthesis on very large models.

One way to tackle this problem would be to explore new ways of interacting with the user. The latter might for example ask to terminate the induction process earlier. The visual inspection of generated state machines would yield a form of equivalence queries. Otherwise, we might investigate ways of mitigating the lack of user control by injecting further domain knowledge or automating the oracle (see e.g., \cite{Walkinshaw:2007}).

\subsection*{Lack of knowledge and classification errors}

One difficulty with incremental system modeling is related to the lack of system knowledge. When using the interactive feature of QSM for example, the user might be faced with a scenario question for which she does not have a clear classification answer. In such situation, ``don't know'' answers are worth investigating. Another approach would consist of allowing the end-user to defer some scenario questions so that they can be answered later as more system knowledge is being gained.

An open issue related to the previous one is the robustness against possible misclassification of scenario questions by the end-user. Traditional ways of dealing with noisy inputs include probabilistic learning methods. Such methods could be adapted in our context, especially in presence of multiple stakeholders.

Multi-view modeling actually provides another way of dealing with classification errors. For example, domain knowledge and goals may help detecting and/or correcting such mistakes. Indeed, misclassification typically leads to consistency checks failing if domain properties and goals are available. Those failures can be effectively fixed provided they are detected soon enough. Once again, a better integration of model analysis and model synthesis techniques might prove very useful here.

\subsection*{Implied scenarios}

The potential nuisance of implied scenarios has been discussed in Section~\ref{section:inductive-correctness}. Implied scenarios occur when a distributed system is designed globally while implemented component-wise. In our framework, problematic implied scenarios are system behaviors correctly rejected by the system inductively synthesized, as required by a goal for instance, but exhibited in the system when individial agent state machines are composed.

Existing techniques, such as ~\cite{Uchitel:2004}, may be adapted to our framework in order to detect implied scenarios and submit them as additional scenario questions to the end-user (see Section~\ref{section:related-for-analysis-3}). Even with the availability of such a detection technique, implied scenarios still raise important open issues.
\begin{itemize}
\item The presence of negative implied scenarios requires refactoring of the system decomposition and agent interfaces. An important issue with implied scenarios is the lack of scenario refactoring techniques to systematically eliminate them.
\item Another important problem occurs when we don't know whether an implied scenario should be accepted or rejected. This is related to the problem of lack of knowledge previously discussed. Implied scenarios whose status is unclear stay potentially harmful until a decision is made about them.
\item The late discovery of harmful implied scenarios may be due to an inadequate use of the multi-view modeling framework or to weaknesses of the formal framework itself. For example, our definition of negative scenario borrowed from \cite{Uchitel:2002} should probably be revisited. Indeed, such scenarios define negative system traces only; in particular, they do not define a negative trace in the state machine of the agent that sends the proscribed event. In other words, negative scenarios do not capture a restriction on the behavior of a single agent. Similarly, goals are considered system-wide. Considering only goals that are realizable by at least one agent is often sufficient to avoid related implied scenario problems in the first place.
\end{itemize}

While those restrictions and open issues might limit the applicability of our techniques in specific cases, they also pave the way for numerous improvements and research in multi-view model synthesis.


