\chapter{Conclusion\label{chapter:conclusion}}

Models play a significant role for elaborating requirements and exploring designs of software systems. They help abstracting from multiple details in order to focus on key system aspects. This abstraction process naturally leads capturing complex systems through multiple models. Each of them focusses on particular facets of the system along its intentional, structural, operational and behavioral dimensions.

Building high-quality models for software system is not an easy task. To play a significant role, models should adequately represent the essence of the target system; they should be complete enough to capture all its pertinent facets; high-quality models should be precise when taken in isolation; they should be consistent with each other in multi-view modeling. 

Adequate modeling languages, tools and techniques are therefore required to help building them. This thesis investigated model synthesis as a promising formal support along these requirements. Model synthesis has been articulated along two main axes in the thesis:
\begin{description}
\item[Vertical synthesis] To capture the operational semantics of high-level models by deriving lower-level ones. This kind of synthesis also makes high-level models amenable to analysis.

Along this axe, the thesis defined an operational semantics for guarded high-level Message Sequence Charts (hMSC), a process modeling language used for capturing critical processes involving decisions \cite{Damas:2011}. This semantics has been operationalized through labeled transition systems (LTS) and synthesis algorithms have be described. The implementation of a model-checker for guarded hMSCs has been described; it adapts a compositional LTS technique borrowed from \cite{Giannakopoulou:2003}.

\item[Horizontal synthesis] To produce missing models in multi-view descriptions, or complete existing ones. Horizontal synthesis uses the inherent overlap in multi-view models to semi-automatically produce or complete model fragments from various sources of knowledge.

Along this axe, the thesis showed how grammar induction can be used to synthesize state machines from positive and negative scenarios illustrating interactions among system agents. This technique operates in a golden triangle where declarative goals can be used to constrain the induction process so as to avoid undesired system behaviors. It also interacts with an end-user who classifies additional scenarios generated by the process as examples and counterexamples of desired system behaviors.
\end{description}

While distinguishing between horizontal and vertical model synthesis may sometimes be a slightly arbitrary, these two axes prove fairly different:
\begin{itemize}
\item Vertical synthesis compiles high-level models into lower-level abstractions; horizontal synthesis does not. 

For example, guarded process models made of refinable tasks and decision nodes were derived as a structure form of transition systems involving guards, then as flat LTS involving events only. In contrast, the interactive synthesis of state machines from scenarios triggers for the identification of declarative goals.

\item Vertical synthesis is derivational by nature whereas horizontal synthesis may be inductive. In the former, the user has a passive role whereas she may have an active role in the latter.

For example, the analysis of guarded process models in the Gisele toolset was shown to benefit from running as an fully automated background process. In contrast, our interactive state machine synthesizer implements an horizontal inductive synthesis where the end-user was seen to play an important role to avoid overgeneralizations.

\item Models derived in vertical synthesis are hidden from the user and thrown away once the analysis results have been obtained. Horizontal synthesis yields models requiring user validatation; such models also have a long lifecycle, such as models involved in requirements or used in documentation.

The thesis discussed how the Gisele toolset proposes various checks on process models without keeping nor even disclosing the synthesized transition systems to the user. In contrast, the ISIS synthesizer was shown to generate scenario questions to the user, and to infer safety properties and annotate synthesized state machines in order for analysts to validate them.

\end{itemize}

\section{Contributions\label{section:conclusion-contributions}}

The contributions of the thesis are summarized along these two model synthesis axes.

\subsection*{Horizontal synthesis}

\begin{itemize}
\item The thesis has proposed an interactive procedure for synthesizing agent state machines from positive, negative and structured forms of scenarios. This procedure uses grammar induction for generalizing scenario behaviors. It is interactive with an end-user who classifies additional scenarios as positive or negative examples of system behaviors.
\item A technique has been discussed to constrain and prune the search space of inductive state machine synthesis. This allows incorporating various sources of knowledge about the target system. This technique has been instantiated to take fluent definitions, legacy components, and goals and domain properties into account. The pruning technique provide guarantees of the consistency of the synthesized state machines with all such system knowledge. It also speeds up the induction search space for better performances and reduces the number of user interactions.
\item Our induction technique is supported by a tool. This tool also integrates various checks for the consistency of scenarios, state machines and goals as well as a technique to infer safety properties from scenarios. 
\item The techniques and tools have been evaluated on case studies and synthetic datasets. In the latter case, a novel evaluation protocol for inductive model synthesis has been proposed and ran as a formal competition.
\end{itemize}

\subsection*{Vertical synthesis}

\begin{itemize}
\item This thesis gave a formal semantics to guarded hMSC. This semantics is defined through the introduction of guarded LTS, a kind of labeled transition systems allowing events and guards on transitions. Guarded LTS have been given a declarative trace semantics as well as composition and hiding operators.
\item Two synthesis algorithms have been proposed to derive guarded hMSC to guarded LTS to pure LTS. Those algorithms have been shown to make guarded process models amenable to a variety of model analyses.
\item The implementation of a model-checker for guarded hMSC and guarded LTS has been discussed. The thesis also discussed the architecture and important design decisions in the implementation of another toolset aimed at analyzing mission-critical process models.
\end{itemize}



\section{Limitations and perspectives\label{section:conclusion-limitations}}

The formal definition of the multi-view modeling framework used in the thesis raises a certain number of issues in terms of model expressiveness.
\begin{itemize}

\item Labeled transition systems do not distinguish between input and output events. Such distinction may be necessary for specific analyses, such as implied scenario detection \cite{Letier:2005b}. This could be addressed either by considering the use of input-output automata \cite{Lynch:1987} or explicitly introducing models for capturing the structural system dimension \cite{Jackson:1995, Magee:1995}.

\item The behavior models capturing agent and system behaviors define any prefix of an accepted trace as an accepted trace. While keeping our framework simple to use, it also slightly limits the class of systems that can be captured.

\item Our framework does not distinguish between successful and unsuccessful system executions. Among others, this does not allow reasoning in terms of deadlocks, an important aspect when analyzing distributed systems. Labeled transition systems can easily be extended to support this, see e.g. \cite{Uchitel:2003}. Here as well, enriching scenario models would be necessary to benefit of deadlock analysis in multi-view modeling.

\end{itemize}

About the semantics of guarded hMSC, synthesis algorithms and the model-checker:
\begin{itemize}

\item The expressiveness of guarded hMSC as a process language is rather limited. In particular, guards are limited to Boolean variables such as fluents. So called \emph{tracking  variables} are also introduced in \cite{Damas:2011}, but are limited to Boolean variables as well. Higher-level modeling abstractions could possibly be defined and translated to our trace semantics, such as \emph{counters} or \emph{resources}.

\item When a safety property is violated by a process, our model checker returns a counter-example as a pure event-trace. Works remains to be done so as to provide an interpretation of such counter-example on the process model itself. 
\end{itemize}

About our inductive LTS synthesis technique and related grammar induction contributions:
\begin{itemize}

\item When using the interactive feature of QSM, an important open question is the robustness to possible misclassification of scenarios questions by the end-user. 

Traditional ways to deal with noisy inputs include probabilistic learning methods, which are not necessary relevant here. The availability of domain knowledge could help detecting and/or correcting such mistakes.

\item QSM raises an issue about the number of scenario submitted for classification by the end-user. As illustrated in experiments, additional system knowledge such as fluents and goals helps reducing the number of questions to be rejected. However, the number of accepted scenarios does not similarly reduce. This might lead to usability issues for large systems.

One way to tackle this problem would be to explore new ways for interacting with the user. The latter could for example request to terminate the induction process early. Visual inspection of generated state machines would yield a form of equivalence queries. 

\item The projection of the synthesized LTS on local agents is known to possibly introduce additional behaviors, the so-called implied scenarios. Negative implied scenarios, i.e. undesirable system behaviors, may hurt multi-model consistency such as violating goals. 

Refactoring techniques to help eliminating undesired implied scenarios still need to be developed.

\item Work remains to be done to let QSM evolve towards a larger and more flexible induction approach to multi-view modeling.

The so far uncompleted transition from QSM to ASM already raises questions and open perspectives about the availability of a superseding induction algorithm. If the importance of a convergence criteria has been discussed, the notion of characteristic sample would need to be revisited.

While relying on grammar induction algorithms for behavior model synthesis has been proven useful, integrating such approaches in a larger and incremental vision of behavior model synthesis has also been discussed as requiring further research.

\end{itemize}

