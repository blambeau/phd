\chapter{Conclusion\label{chapter:conclusion}}

Models are increasingly recognized as an effective means for elaborating requirements, exploring designs of software systems, analyzing and setting up work processes, and so on. Models help abstracting from multiple details in order to focus on key system aspects.

This abstraction process naturally leads to capture complex systems with multiple models. Each of them focuses on particular facets of the system, i.e. its intentional, structural, operational and behavioral dimensions. For example, scenarios illustrate interactions between agents forming the system whereas goals make the underlying objectives precise.

If building software systems is known to be an difficult task, one can hardly argue that building quality models for them is easier.  To play a significant role, models should adequately represent the essence of the target system; they should be complete enough to capture all its pertinent facets; models should also be precise when taken in isolation; they should be consistent with each other in multi-view modeling; and so on. Adequate modeling languages, tools and techniques are therefore required to help building them.

Albeit intertwined, two particular activities can be formaly supported: model \emph{analysis} and model \emph{synthesis}. Model analysis mostly aims at querying and verifying models against various kinds of properties; this typically involves techniques such as model checking. Model synthesis aims at producing models for various source of knowledge, typically other models.

This thesis has investigated model synthesis techniques more particularly, that can be classified along two main underlying objectives:

\begin{itemize}

\item Model synthesis may be used for defining the operational semantics of high-level models, in terms of low-level ones. This kind of model synthesis is also motivated by the wish to make high-level models amenable to analysis, through techniques available on lower-level models.

For example the thesis defines an operational semantics for guarded high-level Message Sequence Charts (g-hMSC), the process modeling language used in \cite{Damas:2011}. This semantics is defined in terms of labeled transition systems, through a compositional synthesis algorithm. This allows using a slight adaptation of the technique from \cite{Giannakopoulou:2003} to model check g-hMSC processes.

\item Model synthesis also proves useful for producing missing models in multi-view descriptions, or completing existing ones. The idea here is to use the inherent overlap of multi-model system descriptions to semi-automatically produce or complete models from various sources of knowledge.

For example, the thesis shows how agent state machines can be interactively synthesized from positive and negative scenarios, under the control of goals and domain properties. This technique reuses and extends grammar induction algorithms from \cite{Oncina:1992, Lang:1998}.

\end{itemize}

Note that, in addition to persuing different goals, these two uses of model synthesis differ in important ways.

\begin{itemize}

\item Semantic-driven synthesis is derivative by nature. In contrast, the completion of multi-view system descriptions benefits from inductive synthesis processes such as grammar induction or inductive logic programming.

\item When used for triggering analyses, produced models have a short lifetime, generally limited to the analysis itself. Models in multi-view descriptions have a different lifecycle as they are often kept as software documentation or for requirements traceability. 

\item In the first case, models are generally kept hidden from the user who has a passive role in the generation process. In the second case, models must generally be validated by the end-user. The latter may also play an active role in the synthesis process, such as the oracle who classifies scenarios as positive of negative system behaviors.

\end{itemize}

We concludes this thesis with two sections. Section \ref{section:conclusion-contributions} summarizes our contributions to model synthesis along the two axes discussed above. Section \ref{section:conclusion-limitations} then discusses limitations and possible future directions. 

\section{Contributions\label{section:conclusion-contributions}}

The contributions of the thesis are summarized along these two model synthesis axes.

\subsection*{Horizontal synthesis}

\begin{itemize}
\item The thesis has proposed an interactive procedure for synthesizing agent state machines from positive, negative and structured forms of scenarios. This procedure uses grammar induction for generalizing scenario behaviors. It is interactive with an end-user who classifies additional scenarios as positive or negative examples of system behaviors.
\item A technique has been discussed to constrain and prune the search space of inductive state machine synthesis. This allows incorporating various sources of knowledge about the target system. This technique has been instantiated to take fluent definitions, legacy components, and goals and domain properties into account. The pruning technique provide guarantees of the consistency of the synthesized state machines with all such system knowledge. It also speeds up the induction search space for better performances and reduces the number of user interactions.
\item Our induction technique is supported by a tool. This tool also integrates various checks for the consistency of scenarios, state machines and goals as well as a technique to infer safety properties from scenarios. 
\item The techniques and tools have been evaluated on case studies and synthetic datasets. In the latter case, a novel evaluation protocol for inductive model synthesis has been proposed and ran as a formal competition.
\end{itemize}

\subsection*{Vertical synthesis}

\begin{itemize}
\item This thesis gave a formal semantics to guarded hMSC. This semantics is defined through the introduction of guarded LTS, a kind of labeled transition systems allowing events and guards on transitions. Guarded LTS have been given a declarative trace semantics as well as composition and hiding operators.
\item Two synthesis algorithms have been proposed to derive guarded hMSC to guarded LTS to pure LTS. Those algorithms have been shown to make guarded process models amenable to a variety of model analyses.
\item The implementation of a model-checker for guarded hMSC and guarded LTS has been discussed. The thesis also discussed the architecture and important design decisions in the implementation of another toolset aimed at analyzing mission-critical process models.
\end{itemize}



\section{Limitations and perspectives\label{section:conclusion-limitations}}

The formal definition of the multi-view modeling framework used in the thesis raises a certain number of issues in terms of model expressiveness.
\begin{itemize}

\item Labeled transition systems do not distinguish between input and output events. Such distinction may be necessary for specific analyses, such as implied scenario detection \cite{Letier:2005b}. This could be addressed either by considering the use of input-output automata \cite{Lynch:1987} or explicitly introducing models for capturing the structural system dimension \cite{Jackson:1995, Magee:1995}.

\item The behavior models capturing agent and system behaviors define any prefix of an accepted trace as an accepted trace. While keeping our framework simple to use, it also slightly limits the class of systems that can be captured.

\item Our framework does not distinguish between successful and unsuccessful system executions. Among others, this does not allow reasoning in terms of deadlocks, an important aspect when analyzing distributed systems. Labeled transition systems can easily be extended to support this, see e.g. \cite{Uchitel:2003}. Here as well, enriching scenario models would be necessary to benefit of deadlock analysis in multi-view modeling.

\end{itemize}

About the semantics of guarded hMSC, synthesis algorithms and the model-checker:
\begin{itemize}

\item The expressiveness of guarded hMSC as a process language is rather limited. In particular, guards are limited to Boolean variables such as fluents. So called \emph{tracking  variables} are also introduced in \cite{Damas:2011}, but are limited to Boolean variables as well. Higher-level modeling abstractions could possibly be defined and translated to our trace semantics, such as \emph{counters} or \emph{resources}.

\item When a safety property is violated by a process, our model checker returns a counter-example as a pure event-trace. Works remains to be done so as to provide an interpretation of such counter-example on the process model itself. 
\end{itemize}

About our inductive LTS synthesis technique and related grammar induction contributions:
\begin{itemize}

\item When using the interactive feature of QSM, an important open question is the robustness to possible misclassification of scenarios questions by the end-user. 

Traditional ways to deal with noisy inputs include probabilistic learning methods, which are not necessary relevant here. The availability of domain knowledge could help detecting and/or correcting such mistakes.

\item QSM raises an issue about the number of scenario submitted for classification by the end-user. As illustrated in experiments, additional system knowledge such as fluents and goals helps reducing the number of questions to be rejected. However, the number of accepted scenarios does not similarly reduce. This might lead to usability issues for large systems.

One way to tackle this problem would be to explore new ways for interacting with the user. The latter could for example request to terminate the induction process early. Visual inspection of generated state machines would yield a form of equivalence queries. 

\item The projection of the synthesized LTS on local agents is known to possibly introduce additional behaviors, the so-called implied scenarios. Negative implied scenarios, i.e. undesirable system behaviors, may hurt multi-model consistency such as violating goals. 

Refactoring techniques to help eliminating undesired implied scenarios still need to be developed.

\item Work remains to be done to let QSM evolve towards a larger and more flexible induction approach to multi-view modeling.

The so far uncompleted transition from QSM to ASM already raises questions and open perspectives about the availability of a superseding induction algorithm. If the importance of a convergence criteria has been discussed, the notion of characteristic sample would need to be revisited.

While relying on grammar induction algorithms for behavior model synthesis has been proven useful, integrating such approaches in a larger and incremental vision of behavior model synthesis has also been discussed as requiring further research.

\end{itemize}

