\addcontentsline{toc}{chapter}{Abstract}
\begin{center}
\textbf{\large Abstract}
\end{center}

Models play a significant role for analyzing requirements and exploring design of software systems. Quality models focus on key aspects of the target system while abstracting from numerous details. The conjoint use of multiple modeling languages and artifacts allows capturing complementary views of the system in a complete, coherent and consistent way. Reaching quality models along these requirements, however, proves difficult in practice. Tools and techniques are therefore required to support users and analysts in this task.

Model synthesis covers a set of techniques that help systematically building models from various source of knowledge about the target system. Synthesizing models may be motivated by the need of performing specific checks and analyses on models. Among others, this allows detecting modeling errors and correcting them early. Model synthesis can also be used to derive missing models and complete existing ones in multi-view descriptions, to generate the documentation and/or the code of specific software components, and so on.

This thesis makes two main contributions in terms of tool-supported synthesis techniques in a multi-view modeling framework involving scenarios, state machines, process models and goals. 

The first one is the precise definition of a trace semantics for process models as well as algorithms to derive them as state machines. In addition to a various collection of analyses contributed in a companion thesis, this makes model-checking available on such process models using a slight adaptation of an existing procedure.

The second one is a interactive technique to synthesize agent state machines from scenarios. Grammar induction is used to generalize the examples of system behavior from the scenarios under the control of counterexamples and goals to be respected by the system. The induction process is guided by an end-user who classifies additional scenarios generated by the synthesizer as positive or negative examples of system behavior.

Our techniques are supported by tools, whose implementation is described. Our inductive synthesizer has been evaluated in depth, both on case-studies and on synthetic datasets. These evaluations have led to the recent setup of an online protocol and plateform for evaluating inductive synthesis techniques.
