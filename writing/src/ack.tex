\addcontentsline{toc}{chapter}{Acknowledgements}
\begin{center}
\textbf{\large Acknowledgements}
\end{center}

Je ne pourrais commencer autrement cette th\`ese qu'en remerciant tout 
particuli\`erement Axel van Lamsweerde, Pierre Dupont et Christophe Damas.

Axel van Lamsweerde a suivi et soutenu ce travail tout au long de pas moins de 
trois projets de recherche, multipliant ses commentaires, conseils et critiques 
tant sur le fond que sur la forme, partageant \'egalement ses propres intuitions 
et id\'ees. Sa vision et son enseignement guident aujourd'hui une grande partie 
de mes connaissances.

Pierre Dupont m'a offert le premier de transformer une bonne intuition de 
recherche en th\'eorie solide et d'en entrevoir les nombreux avantages. Sa 
pr\'ecision, ses encouragements parfois soutenus, son travail et sa 
disponibilit\'e m'auront ensuite suivi tout au long lors de notre travail 
conjoint en induction de grammaire et de mon propre travail de th\`ese.

Christophe Damas est mon coll\`egue depuis 8 ans maintenant, tant pour la 
recherche que pour l'encadrement des \'etudiants ou lors de d\'eplacements en 
conf\'erence. Je lui dois nombre d'enseignements tant sur le plan technique 
qu'humain.

Warmest thanks also go to my jury members, Olivier Bonaventure, Jeff Kramer, 
Emmanuel Letier, Charles Pecheur for both their high-level and technical 
comments.

Ce travail a \'et\'e soutenu financi\`erement par la R\'egion Wallone (projet 
ReQuest, RW Conv. 315592, projet Gisele RW Conv. 616425, projet PIPAS RW Conv. 
XXX) ainsi que par le programme PAI du gouvernement f\'ed\'eral belge (P\^oles 
d'attraction Interuniversitaire, projet MoVES).

