\addcontentsline{toc}{chapter}{Acknowledgements}
\begin{center}
\textbf{\large Acknowledgements}
\end{center}

Je ne pourrais commencer autrement qu'en remerciant Axel van Lamsweerde, Pierre 
Dupont et Christophe Damas.

Axel van Lamsweerde a suivi et soutenu ce travail au long de pas moins de 
trois projets de recherche, multipliant ses commentaires, conseils et critiques 
tant sur le fond que sur la forme, partageant ses propres intuitions, ses id\'ees,
son exp\'erience. Sa vision et son enseignement guident aujourd'hui une 
grande partie de mes connaissances.

Pierre Dupont m'a offert le premier de transformer une bonne intuition de 
recherche en th\'eorie solide et d'en entrevoir les nombreux avantages. Sa 
pr\'ecision, ses encouragements parfois soutenus, son travail et sa 
disponibilit\'e m'auront accompagn\'e tout au long de notre travail en 
inf\'erence grammaticale, l'un des piliers de cette th\`ese.

Christophe Damas est mon coll\`egue depuis 8 ans maintenant, tant dans la 
recherche que dans l'encadrement des \'etudiants, lors de d\'eplacements en 
conf\'erence, lors de la r\'edaction de nos th\`eses respectives, ou de leur(s) 
pr\'esentation(s). Je lui dois nombre d'enseignements tant sur le plan technique 
qu'humain.

My examination board deserves special thanks too. Thanks to Olivier Bonaventure, 
Jeff Kramer, Emmanuel Letier and Charles Pecheur both for their high-level and 
technical comments. Let me also thank Jeff Magee, Sebastian Uchitel, Dalal 
Alrajeh and Alessandra Russo from Imperial College London for the fruitful 
meetings and discussions organized there. On that side of the Channel, special 
thanks also go to Neil Walkinshaw and Kirill Bogdanov from the University of 
Sheffield for co-organizing the Stamina competition.

De nombreux coll\`egues ont partag\'e avec moi de nombreux avis lors de 
nombreuses discussions, d\'epassant largement le cadre de ce travail ou 
de l'informatique. Je pense en particulier \`a Jos\'e Vander Meulen, 
Fran\c{c}ois Roucoux (mais aussi ses coll\`egues de Palantiris), Alain Pirotte,
Antoine Cailliau, Renaud De Landtsheer, Jean-No\"el Monette, Pierre Schaus, 
Gr\'egoire Dooms, St\'ephane Zampelli, S\'ebastien Comb\'efis et Damien Leroy.

Avant tout, je remercie mes parents, mes fr\`eres et soeurs, ma compagne Elodie, 
et sa famille \`a elle. Pour leur patience, leur int\'er\^et, leur soutien et 
leurs nombreux encouragements bien s\^ur, mais en fait et surtout pour le reste.

Ce travail a \'et\'e soutenu financi\`erement par la R\'egion Wallone lors des 
projets ReQuest (Programme WIST 1.0, RW Conv. 315592), Gisele (Programme WIST 2.0, 
RW Conv. 616425) et PIPAS (Programme WIST 3.0, RW Conv. 1017087) ainsi que par 
le programme PAI du gouvernement f\'ed\'eral belge (P\^oles d'attraction 
Interuniversitaire, projet MoVES).
