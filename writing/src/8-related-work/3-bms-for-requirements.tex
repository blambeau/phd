\section{Eliciting requirements from scenarios\label{section:related-for-requirements}}

This section addresses the problem of eliciting requirements from scenarios. Section~\ref{related-for-requirements-1} reviews approaches for inferring goals from scenarios. Section~\ref{related-for-requirements-2} then discusses how scenarios and safety properties can be used jointly for incremental requirement elicitation. 

\subsection{Goal inference from scenarios\label{related-for-requirements-1}}

A technique for synthesizing goals from scenario collections was presented in~\cite{Damas:2006, Damas:2011}. This technique is integrated in the ISIS tool (see Section~\ref{section:tool-support-isis}). It consists in decorating MSC timelines with invariants on fluents monitored and controlled by the associated agent. From these invariants, goals are induced according to two kinds of specification patterns, namely \emph{Maintain goals} taking the form $\square(P \rightarrow Q)$ and \emph{Immediate Achieve goals} taking the form $\square(P \rightarrow \circ Q)$. The inferred goals are submitted to the user for validation.

This synthesis technique helps enriching a multi-view system description while guaranteeing multi-view consistency. An inferred goal is stated as a generalization of the available examples. The accepted goals enrich the goal model accordingly; the rejected ones call for enriching the scenario collection with a counterexample. 

The technique relies on the availability of state invariants on fluents; it might be extended to infer goals from any annotated LTS, and hence, from agent state machines or from a hMSC. 

Van Lamsweerde and Willemet developed an inductive learning technique for generating goal specifications in linear temporal logic (LTL) from positive and negative scenarios expressed in MSC-like form \cite{VanLamsweerde:1998}. The user here has to provide state-based annotations of scenario interactions. These are the domain pre- and postconditions on the operations operationalizing goals; they are found in the corresponding operation model. Compared with our fluent-based approach, the technique requires more input to be provided to the synthesizer; it requires pre- and postconditions whereas our fluent definitions encode postconditions only.

\cite{Alrajeh:2007} uses inductive logic programming for extracting requirements from example scenarios and a partial requirements specification. Scenarios capture positive and negative system behaviors whereas the requirements specification captures an initial but incomplete description of the envisioned system. The specification is completed by learning a set of missing requirements that cover all the desirable scenarios and none of the undesirable ones.

\subsection{Incremental requirement elicitation from scenarios and safety properties\label{related-for-requirements-2}}

In a multi-view behavior modeling framework like ours, scenarios capture a ``lower-bound'' on system behaviors whereas goals capture an ``upper bound''. Scenarios capture behaviors that the system must exhibit whereas goals prune the space of all acceptable behaviors captured by the consistency conditions in Chapter \ref{chapter:inductive-synthesis}, namely,
\begin{align*}
&\mathcal{L}^+(\me{Scenarios}) \subseteq \mathcal{L}(\agentscomposed) \\
&\mathcal{L}^-(\me{Goals}) \cap \mathcal{L}(\agentscomposed) = \emptyset
\end{align*}

In \cite{Uchitel:2007, Uchitel:2009}, Uchitel et al. developed a technique for capturing within a single behavior model both the lower bound from the scenarios and the upper bound from goals. Modal Transition Systems (MTS) \cite{Larsen:1988} are used instead of LTS for capturing trace-based system behaviors; MTSs are transition systems that distinguish between required, possible and proscribed behaviors.

MTSs support incremental system design through user interactions aimed at eliciting missing requirements. Roughly, the techniques from \cite{Uchitel:2003} and \cite{Giannakopoulou:2003} are adapted to capture through MTSs (1) a structured form of MSC scenarios and (2) safety properties expressed in a 3-valued variant of FLTL. The obtained MTSs are merged so as to capture the lower and upper bounds of system behaviors in a single MTS; the latter preserves the semantics of both scenarios and goals. 

So-called \emph{maybe} traces are extracted from this MTS and submitted for classification by the user (we call them ``maybe queries'' in the sequel). Such traces denote system behaviors that do not violate safety properties but were not explored in the scenario specification. Accepted traces denote missing scenarios in the specification whereas rejected ones help identifying missing safety properties. 

In terms of our formal framework, \emph{maybe} queries belong to the following set:
\begin{align}
\left( \Sigma^* \setminus \mathcal{L}^-(\me{Goals}) \right) \setminus \mathcal{L}^+(\me{Scenarios})
\label{relation:uchitel-mts-questions}
\end{align}
that is, the set of system behaviors that are not explicitly rejected by goals, excluding those already required by scenarios.

When all goals denote safety properties, this language is regular and can be captured through standard automata. Our toolset could therefore be complemented with this interactive synthesis technique.

Even though they rely on a form of scenario question, the approach from Uchitel et al. is not equivalent to our inductive synthesis approach. In particular, no generalization of scenario behaviors occurs in \cite{Uchitel:2007, Uchitel:2009}. This actually raises questions about the convergence of the underlying elicitation process. 

Let us assume that we know the expected target system. Let $T$ denote an hypothetical state machine capturing the system behaviors. Multiple scenarios tend to cover overlapping paths of $T$. However the coverage of each scenario is intrinsically incomplete in terms of event-continuations of $T$ states. When these scenarios are captured through a trace preserving transition system $S$, some of its states are trace-equivalent with respect to the target system, but are not equivalent in $S$. The latter tends to grow when new scenarios are added since no state merging occurs. If $T$ grows, new questions appear and the process is diverging. This phenomenon seems independent of the \emph{kind} of transition system used to capture behaviors, e.g., LTS, MTS, standard automata. Indeed, all these automaton flavors have their equivalence relation defined in terms of state continuations.

To sum up, to our understanding, \cite{Uchitel:2009} offers no guarantee for the convergence of $\mathcal{L}^+(\me{Scenarios})$ towards the behaviors of the composed system $\mathcal{L}(\agentscomposed)$. As a consequence, even if the safety properties converge towards an adequate set capturing all counterexamples of desired system behavior, it seems to us that the language denoted by (\ref{relation:uchitel-mts-questions}) itself will never get empty. 

In contrast, in the same conditions where the target system is fixed, grammar induction provides convergence guarantees towards the target machine $T$ when the input scenario collection is sufficiently growing \cite{Oncina:1993} (see Chapter~\ref{section:inductive-background}). In particular, the identification-in-the-limit framework guarantees that, when its input is rich enough, RPNI induces the target system\footnote{We restrict our attention to RPNI instead of QSM or ASM here to avoid mixing two different kinds of scenario questions in the discussion.}. Moreover, available domain knowledge, goals in particular, tend to enrich their input.

These observations open interesting perspectives for future research. The two approaches seem indeed complementary. In particular, $\mathcal{L}(System)$, the language of the system LTS induced with RPNI, tends to converge towards $\mathcal{L}(\agentscomposed)$. In other words, maybe queries from the language:
\begin{align*}
\left( \Sigma^* \setminus \mathcal{L}^-(\me{Goals}) \right) \setminus \mathcal{L}(\me{System})
\end{align*}
appear promising as an elicitation technique, \`a la \cite{Uchitel:2007, Uchitel:2009}, but then with a convergence criterion.
