\section{Inductive synthesis of behavior models from scenarios\label{section:related-from-scenarios}}

This section compares existing techniques for synthesizing state machines from scenarios with the inductive approach introduced in Chapter~\ref{chapter:inductive-synthesis}. A more complete survey of such techniques can be found in \cite{Liang:2006}.

%%%

\subsection{Statecharts synthesis from sequence diagrams}

Whittle and Schumann proposed a technique for generating UML statecharts from sequence diagrams that capture positive scenarios (and positive scenarios only) \cite{Whittle:2000}. Their technique requires scenario interactions to be explicitly annotated with pre- and post-conditions on global state variables expressed in the Object Constraint Language (OCL). 

In a similar spirit, Kruger et al. proposed a technique for translating MSCs into statecharts \cite{Kruger:2000}. Their algorithm also requires state information, through MSC conditions, as additional input.

Those techniques generalize behaviors described in input scenarios by introducing sequencing, alternatives and loops. State annotations provide the semantic basis for guiding the generalization process. In both cases, those annotations are required for the technique to work; the user of the technique is expected to provide them. 

This contrasts with a grammar induction approach since, in that case, behavior generalization is driven by compatible state merging, which is semantically rooted in \emph{event-based} continuations of PTA states. In our approach, state-based information such as fluents is used only for pruning the induction process (see Section~\ref{section:inductive-mutliview-consistency}) and is not required for the technique to work.

State machine synthesis techniques from scenarios are often used in a requirements engineering context, when the target system is unknown \cite{Weidenhaupt:1998}. Indeed, scenarios are concrete illustrations of the envisionned system. They are easier to provide by end-users than operation specifications in terms of pre- and post- conditions, MSC conditions, goals, state machines, etc. Therefore, scenarios tend to appear early in the design process and form a natural but limited input for state machine synthesis algorithms. 

Our grammar induction approach only requires end-user scenarios in the first place and therefore fits nicely in such contexts. As discussed in Chapter~\ref{chapter:inductive-synthesis} and illustrated in evaluations (Chapter~\ref{chapter:evaluation}) and tool support discussions (Chapter~\ref{chapter:tool-support}), our algorithms also yield a more flexible approach. Indeed it already works with a limited scenario collection but adapts to richer inputs typically made available in subsequent analysis and design iterations. For example, our approach supports taking simple MSC scenarios as input but also works with structured forms of scenarios such as hMSCs. Similarly, fluents and goals are not required in the first place but are intended to smoothly integrate state-based knowledge in an incremental way.

%%%

\subsection{Minimally adequate teacher approach}

Makinen and Systa developed an interactive algorithm for synthesizing UML statecharts from sequence diagrams \cite{Makinen:2001}. It is another approach relying on grammatical inference techniques for generalizing examples of system behavior captured in scenarios. This work is inspired from the approach of minimally adequate teacher and the L$^*$ algorithm proposed by Angluin~\cite{Angluin:1987}. 

As with our approach, and unlike those discussed in previous section, the technique works with scenarios only and does not require additional state-based annotations. There are significant differences with our approach though.
\begin{itemize}

\item Their input sequence diagrams capture positive examples of system behavior only. The negative knowledge required to avoid overgeneralization comes from negative answers to membership and equivalence queries asked to an oracle (see below); such negative knowledge does not take the form of end-user negative scenarios.

\item Grammar induction is applied on a per agent basis. In other words, induction traces are sequences of events seen by a single agent along its scenario timeline. The alphabet of possible events and the learning strings are specific to each agent; the learning problem is tackled separately for each of them.

\item The interacting user is asked to answer both membership and equivalence queries. Membership queries are traces to be classified as positive or negative examples of agent behavior; they amount to our scenario queries. Equivalence queries, in contrast, require the ability to classify state machine candidates as correctly capturing the complete behavior of a specific agent.
\end{itemize}

As discussed in Section~\ref{section:inductive-discussion}, tackling the induction problem on a per agent basis is an interesting alternative to our approach. For end-user involvement, however, an per agent approach seems less appropriate than ours; it requires the user to be able to classify traces in a different target language for each agent. It appears much more convenient for the end-user to interact in the scenario language of the global system as this is the language she used in the first place.

Equivalence queries may also prove difficult to answer by an end-user in practice as the agent state machines are unknown. That being said, a certain form of equivalence query might be seen in our approach as well. With the ISIS tool, for example, a visual inspection of agent and system state machines is recommended for validation and triggering of a new modeling cycle if required. Such inspection may typically highlight negative scenarios in case of overgeneralization or new positive scenarios illustrating additional desired features.

As the approach is interactive and can be used with only a few scenarios, it proves useful in early phases of system design. Unlike ours, though, such approach does not adapt to richer scenario inputs nor does it take additional models into account such as fluents or goals. In other words, multi-view consistency is not guaranteed by the synthesis process.

%%%

\subsection{Statecharts synthesis from live sequence charts}

Harel et al. proposed a technique in \cite{Harel:2005} for synthesizing statecharts from Live Sequence Charts (LSCs) \cite{Damm:2001}. LSCs are an extension of MSCs that distinguish between allowed and required behaviors. MSCs are often seen as capturing \emph{existential} scenarios whereas LSCs allow for capturing \emph{universal} ones. 

Universal behaviors are related to liveness, that is, the notion that ``something good will eventually happen''; existential behaviors are related to the more common idea of \emph{examples} of desired system behavior. In view of the rich expressiveness of universal scenarios, behavior generalization is not necessarily relevant in the context of LSCs. On the other hand, LSCs are more difficult to provide by stakeholders involved in early phases of system design. 

