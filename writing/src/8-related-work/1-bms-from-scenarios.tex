\section{Behavior model synthesis from scenarios\label{section:related-from-scenarios}}

This section reports techniques for synthesizing state machines from scenarios; it also discusses the differences with our inductive approach introduced in Chapter~\ref{chapter:inductive-synthesis}. A more complete survey of such techniques can be found in \cite{Liang:2006}.

%%%

\subsection{Statecharts synthesis from sequence diagrams}

Whittle and Schumann proposed a technique for generating UML statecharts from sequence diagrams that capture positive scenarios and positive scenarios only \cite{Whittle:2000}. Their technique requires scenario interactions to be annotated with pre- and post-conditions on global state variables expressed in the Object Constraint Language (OCL). 

In a similar spirit, Kruger et al. proposed a technique for translating MSCs into statecharts \cite{Kruger:2000}. Their algorithm also requires state information, through MSC conditions, as additional input.

These techniques generalize behaviors described in input scenarios by introducing sequencing, alternatives and loops. In both cases, state annotations provide the semantics basis to guide the generalization process. 

Observe that this is rather different from our induction technique. With grammar induction, behavior generalization is induced by compatible state merging, which is semantically rooted in \emph{event-based} continuations of PTA states. State-based information, such as fluents, is only used to prune the induction process (see Section~\ref{section:inductive-mutliview-consistency}). 

The advantage of our approach in comparison with those is that it does not require state-based annotation of scenarios such as pre- and postconditions of MSC events. This kind of knowledge proves particularly difficult to provide by end-users in early steps of system design. Our use of fluents and goals tend to smoothly integrate state-based knowledge in an incremental way.

%%%

\subsection{A minimally adequate teacher inductive approach}

Makinen and Systa developed an interactive algorithm for synthesizing UML statecharts from sequence diagrams \cite{Makinen:2001}. It is another approach relying on grammatical inference techniques for generalizing system behaviors described in scenarios. More precisely, it is inspired from the model of minimally adequate teacher and L$^*$ algorithm proposed by Angluin in~\cite{Angluin:1987}. 

Important differences exist with our approach:
\begin{itemize}

\item Input sequence diagrams denote positive examples of system behavior only. The negative knowledge required to avoid overgeneralization comes from negative answers to membership and equivalence queries asked to an oracle (see below); this negative knowledge does not take the form of end-user negative scenarios.

\item Grammar induction is applied on an agent basis. In other words, induction traces are sequences of events seen by a single agent along its timeline in scenarios. The alphabet of possible events and the learning strings are specific to each agent; the learning problem is tackled separately for each of them.

\item The end-user is asked to answer both membership and equivalence queries. Membership queries are traces to be classified as positive or negative examples of agent behaviors, and are similar to our scenario queries. Equivalence queries, in contrast, denote the ability to classify state machines candidates as correctly capturing full agent behaviors. 
\end{itemize}

As discussed in Section~\ref{section:inductive-discussion}, tackling the induction problem on an agent basis is an interesting alternative to our approach. For end-user involvement, however, an per-agent approach seems less adequate than ours as it requires the user to be able to classify traces in a different target language for each agent. It is much more convenient to interact with the end-user in the scenario language of the global system because this is the language she used in the first place.

Equivalence queries might also prove difficult to answer by an end-user in practice as agent state machines are precisely unknown. That said, a form of equivalence queries is actually used in our approach as well. With the ISIS tool, for example, a visual inspection of agent and system state machines is often used to trigger a new modeling cycle. Such inspection typically results in a negative scenario in case of overgeneralization or in the definition of new positive scenarios illustrating additional system features. 

%%%

\subsection{Statecharts synthesis from live sequence charts}

Another approach is due to Harel which proposed a technique in \cite{Harel:2005} to synthesize statecharts from Live Sequence Charts (LSCs) \cite{Damm:2001}. LSCs are an extension of MSCs that distinguishes between possible and necessary behaviors. In other words, MSCs are often seen as capturing \emph{existential} scenarios whereas LSCs allow capturing \emph{universal} ones. 

Universal behavior is related to the notion of liveness, while existential behavior is related to the more common approach to scenarios, that of \emph{examples} of system behavior. Due to the rich expressiveness of universal scenarios, behavior generalization is not necessarily relevant in a LSC context. On the other hand, LSCs are more difficult to provide by end-users involved in early system design phases. 

