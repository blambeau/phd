\section{The ISIS Synthesizer\label{section:tool-support-isis}}

The interactive LTS synthesis technique described in Chapter \ref{chapter:inductive-synthesis} has been implemented in the ISIS tool (Interactive State machine Induction from Scenarios). The tool integrates the following features.

\begin{itemize}
\item Editing and visualizing scenarios in a graphical syntax \`a la MSC (see Section \ref{section:background-scenarios}).
\item Definition of fluents and goals, on a per-agent basis (see Sections \ref{section:background-fluents} and \ref{section:background-goals}).
\item Verification of a multi-view modeling specification, more specifically consisting in checking the consistency of scenarios, state machines and goals (see Sections \ref{subsection:background-scenario-consistency} and \ref{subsection:background-goals-consistency}).
\item Interactive induction of LTS state machines from scenarios, taking fluents and goals into account (see Chapter \ref{chapter:inductive-synthesis}).
\item Decoration of scenario timelines and state machines with state invariants (see Section \ref{section:background-fluents}).
\item Induction of safety goals from scenarios (see \cite{Damas:2011}).
\end{itemize}

Figure \ref{image:isis-tool} shows a snapshot of the ISIS tool. On the left, the current scenario specification is outlined: the agents forming the system are listed together with associated fluent and goal definitions; positive and negative MSC scenarios are listed below. On the right, two scenarios are shown for the train system case study. The scenario ``Normal journey'' in the background is annotated with computed state invariants.

\begin{figure}
\centering\scalebox{.545}{
  \includegraphics*{src/7-tool-support/images/isis}}
  \caption{The ISIS tool, scenarios of the train system case-study\label{image:isis-tool}}
\end{figure}

The analysis and synthesis techniques in ISIS are made available ``on demand'' by right-clicking the name of a specification (see Fig.~\ref{image:isis-tool-context}).
\begin{description}
\item[Check specification] allows the user to check if the collection of positive and negative scenarios is consistent (as defined in Section~\ref{subsection:background-scenario-consistency}) and that the scenarios do not violate safety goals (as defined in Section~\ref{subsection:background-goals-consistency}).
\item[Synthesize LTSs] launches the interactive induction process described in Chapter~\ref{chapter:inductive-synthesis}. ISIS implements the QSM algorithm; the scenario specification is automatically completed with the user answers to generated scenario questions (see Section~\ref{section:lts-induction-from-mscs}). Moreover, the induction is automatically constrained with available fluents and goals (see Section~\ref{section:inductive-mutliview-consistency}). The tool does not support hMSC models; therefore it does not integrate ASM either (see Section~\ref{section:inductive-from-hMSC}).
\item[Compose System LTS] builds the LTS state machine of the composed system and opens it in the right pane.
\item[Discover goals] launches the techniques allowing to discover of goals from scenarios as discussed in \cite{Damas:2006, Damas:2011}.
\end{description}

\begin{figure}
\centering\scalebox{.675}{
  \includegraphics*{src/7-tool-support/images/isis-context}}
  \caption{Available analysis and synthesis techniques in the ISIS tool\label{image:isis-tool-context}}
\end{figure}

During inductive synthesis, scenario questions are submitted to the end-user for classification (see Fig.~\ref{image:isis-tool-scenario-question}). When rejecting a scenario, a ``No, why?'' button invites her to provide the rational for rejection by updating fluent and goal definitions. 

\begin{figure}
\centering\scalebox{.445}{
  \includegraphics*{src/7-tool-support/images/isis-question-1}}
  \caption{A scenario question submitted to the end-user during inductive LTS synthesis\label{image:isis-tool-scenario-question}}
\end{figure}

At the end of the synthesis process, new scenarios appear in the collection according to their classification by the end-user. Those scenarios are given comprehensible names, such as ``Opening doors while moving'' (see left pane in Fig.~\ref{image:isis-after}).

The LTS state machine for each agent also appears in the specification outline. State machines can be visualized with or without state assertions. The LTS state machine of the train controller is shown in both modes in Fig.~\ref{image:isis-after}. 

\begin{figure}
\centering\scalebox{.575}{
  \includegraphics*{src/7-tool-support/images/isis-after}}
  \caption{After synthesis, the annotated state machines can be visualized.\label{image:isis-after}}
\end{figure}

\subsubsection*{Discussion}

The ISIS tool has been used successfully on various case studies during the evaluation of the thesis (see Chapter \ref{chapter:evaluation}). Our integrated approach to multi-view model synthesis proves very effective in practice. The availability of \emph{multiple} synthesis techniques supports the incremental intertwined construction of each view by launching the adequate synthesis assistant.

Our approach could be better supported in practice by complementing the ``on demand'' approach with additional guidance. The idea here would be to run the different synthesis techniques in background. A list of suggestions could be maintained such as:
\begin{itemize}
\item ``\emph{new scenarios available for classification, run LTS synthesis!}'',
\item ``\emph{new goals have been inferred, click here}''. 
\end{itemize}
Consistency checking of the specification would certainly benefit from a similar feature. 

%The tool described in the next section makes an investigation into this kind of tool support. 
