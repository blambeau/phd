\chapter*{Recent changes}

\section*{Conclusion}

\begin{itemize}
\item The conclusion has been revisited in depth. Contributions, significance of 
      results and strenghts of our approach have all been clarified.
\item The conclusion now discusses open issues about robustness to
      misclassification of scenario questions by the end user.
\item The conclusion now discusses open issues about the convergence process and 
      the scalability of our approach on larger systems that those considered in
      the case studies.
\item The conclusion now discusses lack of knowledge issues and inability to 
      answer scenario questions.
\item This conclusion now discusses implied scenarios and related open issues.
\end{itemize}

\section*{Evaluation}

\subsection*{What do we evaluate?}

\begin{itemize}
\item The introduction section of Chapter 5 has been extended to better sell the 
    way evaluations have been conducted and why they have been conducted that 
    way. In particular, it is now much clearer that we evaluate the overal 
    effectiveness of inductive synthesis through controlled experiments. 
    Links with the thesis objectives in Introduction have been made explicit.
\item A fresh new section has been added when evaluating the approach on case 
    studies. A modeling session in the ISIS tool is illustrated step-by-step on
    the Mine Pump. The section show how the pump controller state machine is 
    inferred. Only two initial scenarios are given as input. Scenario questions
    trigger the identification of fluents, domain properties and goals. The pump
    controller is built in three small iterations in the tool, demonstrating the
    effectiveness of the proposed approach to build a multi-view, adequate, 
    complete and consistent specification of such system.
\item The conclusion/discussion section of the same chapter has been slighlty 
    extended and revisited to link technical evaluation results to the thesis
    objectives.
\end{itemize}

\subsection*{Scalability}

\begin{itemize}
\item The scalability of the approach has been further discussed in the conclusion
    section of Chapter 5. In particular it is now explicitely shown that the 
    approach scales when one considers model adequacy/accuracy, induction time 
    and real time usage. 
\item It has been made explicit that the number of positive scenario questions 
    might lead to a scalability problem when QSM is used on large systems.
    A solution to this problem is proposed, this problem is further examined in
    the thesis conclusion.
\item It should be noted that no new plot has been added that would show how the 
    approach scales (time or \# of scenario questions) when the size of the 
    target system increases. The reason is that it would lead to plots whose 
    soundness is arguable because of comparisons based on unrelated sample 
    richnesses (a learning sample of 100\% is richer for large target systems 
    than for small ones, in an unknown relation).
\end{itemize}

\section*{Related Work}

\begin{itemize}
\item Strenghts of our approach have been made clearer when compared to statechart 
  synthesis from Whittle and Schuman [WS00] as well as Kruger [Kr00] (Section 
  8.1.1)
\item Strenghts of our approach have been made clearer when compared to the minimaly
  adequate teacher approach [MS01] (Section 8.1.2).
\item A comparison has been added with Harel's Play in/play out approach [HM03] in
  Section 8.1.3. Strenghts and drawbacks of both approaches are discussed.
\end{itemize}

\section*{Miscellaneous}

\begin{itemize}
\item Introduction: references to OMT and Finkelstein's viewpoints have been added 
  when talking about Multi-view modeling.
\item Section 2.4.2: a remark has been added about our implicit use of the partial 
  ordering framework with negative scenarios.
\item Section 2.7: the \emph{may} vs \emph{must} of decision nodes in process models has been 
  made explicit.
\item Chapter 3: the advantages of introducing guarded LTS have been made more 
  explicit in Introduction and Summary sections.
\item Section 3.3.1: conditions under which the g-LTS resulting from g-hMSC can be
  simplified (tau transitions removal) have been clarified.
\item Section 3.3.2: a remark has been added about the reason why we call the Super 
  g-LTS "super" (superset of admissible traces).
\item Section 3.3.2: a remark has been added about fluents with undetermined initial
  values in fluent g-LTS construction.
\item Section 4.1.1: a remark has been added about our implicit assumption that the
  synthesized system is captured through a minimal automaton.
\item Section 4.1.1: a remark has been added about our implicit use of the general 
  MSC framework with partial event ordering.
\item Section 4.3: the fact that any quotient automaton still covers positive 
  scenarios by construction has been recalled.
\item Section 4.3.3 and others: incorrect references to the Compatible function have
  been replaced by references to the Consistent.
\item Section 5.2.7: a remark has been added about the small target state machines 
  used in case studies.
\item Section 5.3.1: the fact that induction samples are balanced because of the 
  kind of automata considered has been clarified.
\item Section 5.3.x: the absisse scale has been clarified in plots.
\end{itemize}

